Los resultados expuestos en las secciones anteriores se pueden generalizar a tiempo continuo, entendiendo al filtro de Kalman en tiempo continuo como límite de filtros de Kalman en tiempo discreto. Esto ya fue probado en dimensión finita por \cite{Shaid1999TheFilter, Kelly2014Well-posednessTime, Lange2022DerivationCoefficients}.

Se considera la dinámica a tiempo continuo en un horizonte finito de tiempo $t \in [0, T]$
\begin{align*}
    \mathbf{x}'(t) & = \mathbf{f}(\mathbf{x}(t), \mathbf{w}(t)) \\
    \mathbf{y}(t) & = \mathbf{g}(\mathbf{x}(t), \mathbf{v}(t))
\end{align*}
en donde $\mathbf{w}: \R \to \R^{n_\mathbf{w}}$, $\mathbf{v}: \R \to \R^{n_\mathbf{v}}$ son variables aleatorias tales que $\E[\mathbf{w}(t)] = 0$, $\E[\mathbf{v}(t)] = 0$ y tienen segundo momento finito en el sentido de que
\begin{equation*}
    \E[ \mathbf{w}(t)^\top \mathbf{w}(t) ] = \mathbf{Q}(t), \quad \E[ \mathbf{v}(t)^\top \mathbf{v}(t) ] = \mathbf{R}(t)
\end{equation*}
para ciertas funciones a valores matriciales $\mathbf{Q}: \R \to \R^{n_\mathbf{w} \times n_\mathbf{w}}$, $\mathbf{R}: \R \to \R^{n_\mathbf{v} \times n_\mathbf{v}}$.

Se consideran las siguientes dinámicas discretas, parametrizadas por el paso de tiempo $\Delta t$
\begin{align*}
    \mathbf{x}_{k+1} & = \Tilde{\mathbf{f}} (\mathbf{x}_k, \mathbf{w}_k) := \mathbf{x}_k + \Delta t \cdot \mathbf{f}(\mathbf{x}_k, \mathbf{w}_k )\\
    \mathbf{y}_k & = \mathbf{g}(\mathbf{x}_k, \mathbf{v}_k)
\end{align*}
en donde 
\begin{equation*}
    t_k = k\Delta t, \,  \mathbf{x}_k := \mathbf{x}(t_k), \,  \mathbf{y}_k := \mathbf{y}(t_k), \, \mathbf{w}_k := \mathbf{w}(t_k), \, \mathbf{v}_k := \mathbf{v}(t_k).
\end{equation*}

Se recuerda de secciones anteriores entonces el operador de Koopman asociado a $\Tilde{\mathbf{f}}$, que en este contexto quedará parametrizado por $\Delta t$,
\[
(\U_{\Delta t} \psi)(\mathbf{x}) = \E [\psi (\Tilde{\mathbf{f}} (\mathbf{x}, \cdot))] = \E [\psi (\mathbf{x} + \Delta t \cdot \mathbf{f} (\mathbf{x}, \cdot))].
\]
Esta definición motiva la del operador de Koopman en tiempo continuo.

\begin{defn}[Operador de Koopman estocástico, tiempo continuo]
    Se define el operador $\Tilde{\U}_{\Delta t}: \H_\X \to \H_\X$ mediante
    \[
    (\Tilde{\U}_{\Delta t} \psi)(\mathbf{x}_0) = \E \left [\psi \left ( \mathbf{x}_0 + \int_0^{\Delta t} \mathbf{f}(\mathbf{x}(s), \mathbf{w}(s)) ds \right ) \right ]
    \]
\end{defn}

Intuitivamente, este operador desplaza la dinámica continua en un tiempo $\Delta t$. 

\begin{prop}[Phillip et al. \cite{Philipp2024ErrorOperator}]
    $(\Tilde{\U}_{\Delta t})_{\Delta t \geq 0}$ es un $C_0$-semigrupo cuyo generador
    \[
    \Tilde{\mathcal{L}} \psi = \lim_{\Delta t \searrow 0} \frac{\Tilde{\U}_{\Delta t}\psi - \psi}{\Delta t}
    \]
    está bien definido en un denso de $\H_\X$, es cerrado y disipativo.
\end{prop}

\begin{lema}
    Sea $k$ el kernel de Matérn de parámetro $\nu = 1/2$ y ancho de banda $\gamma$, es decir,
    \[
    k(x, y) = \text{exp} \left ( -\frac{\| x- y \|}{\gamma} \right ).
    \]
    Sea $\H_\X$ su espacio de Hilbert asociado, luego si $\psi \in \H_\X$ se tiene que $\psi$ es $1/2$-Hölder de constante $\sqrt{2/\gamma} \| \psi \|_{\H_\X}$, esto es,
    \[
    |\psi(x) - \psi(y)| \leq \sqrt{\frac{2}{\gamma}} \| \psi \|_{\H_\X} \sqrt{\| x - y\|}, \quad \forall x, y \in \X.
    \]
    \label{lemma:matern_lipschitz}
\end{lema}
\begin{proof}
    Sean $\psi \in \H_\X$, $x, y \in \X$. Primero, por propiedad reproduciente notar que se tiene
    \[
    | \psi(x) - \psi(y) |^2 = | \langle k(x, \cdot), \psi \rangle - \langle k(y, \cdot), \psi \rangle  |^2.
    \]
    Luego, por bilinealidad del producto interno
    \[
    | \psi(x) - \psi(y) |^2 = | \langle k(x, \cdot) - k(y, \cdot), \psi \rangle  |^2.
    \]
    Por desigualdad de Cauchy-Schwarz se tiene
    \[
    \begin{aligned}
        | \psi(x) - \psi(y) |^2 &\leq \| \psi \|^2_{\H_\X} \| k(x, \cdot) - k(y, \cdot) \|^2_{\H_\X} \\
        &\leq \| \psi \|^2_{\H_\X} \langle k(x, \cdot) - k(y, \cdot), k(x, \cdot) - k(y, \cdot) \rangle \\
        &\leq \| \psi \|^2_{\H_\X} ( k(x, x) + k(y, y) - 2k(x,y) ).
    \end{aligned}
    \]
    Dado que 
    \[
    k(x, y) = \text{exp} \left ( -\frac{\| x- y \|}{\gamma} \right ),
    \]
    se tiene que $k(x,x) = k(y,y) = 1$, con lo que
    \[
    | \psi(x) - \psi(y) |^2 \leq 2 \| \psi \|^2_{\H_\X} \left ( 1 - \text{exp} \left ( -\frac{\| x- y \|}{\gamma} \right ) \right ).
    \]
    Utilizando que $1 - e^{-x} \leq x$ para $x \geq 0$, se llega a
    \[
    | \psi(x) - \psi(y) |^2 \leq \frac{2}{\gamma} \| \psi \|^2_{\H_\X} \| x - y \|,
    \]
    lo que implica que
    \[
    | \psi(x) - \psi(y) | \leq \sqrt{\frac{2}{\gamma}} \| \psi \|_{\H_\X} \sqrt{\| x - y \|}
    \]
    obteniendo el resultado.
\end{proof}

Esto tendrá como consecuencia una cota de error para el error que se comete al aproximar el operador de Koopman continuo por uno de dinámica discreta. 

\begin{prop}
    Bajo las hipótesis del lema \ref{lemma:matern_lipschitz}, se tiene que
    \[
    \| \Tilde{\U}_{\Delta t} - \U_{\Delta t} \| \leq 2 \sqrt{\frac{\| \mathbf{f} \|_\infty \Delta t }{\gamma}} 
    \]
    \label{prop:aprox_koop_cont_disc}
\end{prop}

\begin{proof}
    Sean $\psi \in \H_\X$, $\mathbf{x}_0 \in \X$, entonces
    \[
    \begin{aligned}
        \| (\Tilde{\U}_{\Delta t}\psi)(\mathbf{x}_0) - (\U_{\Delta t} \psi)(\mathbf{x}_0) \| &= \left  \| \E \left [\psi \left ( \mathbf{x}_0 + \int_0^{\Delta t} \mathbf{f}(\mathbf{x}(s), \mathbf{w}(s)) ds \right ) \right ] - \E [\psi(\mathbf{x}_0 + \Delta t \cdot \mathbf{f}(\mathbf{x}(t), \mathbf{w}(t)))] \right \| \\
        &\leq \E \left [ \left \| \psi \left ( \mathbf{x}_0 + \int_0^{\Delta t} \mathbf{f}(\mathbf{x}(s), \mathbf{w}(s)) ds \right ) - \psi(\mathbf{x}_0 + \Delta t \cdot \mathbf{f}(\mathbf{x}(t), \mathbf{w}(t))) \right \| \right ] \\
        &\leq \E \left [ \sqrt{\frac{2}{\gamma}} \| \psi \|_{\H_\X} \sqrt{ \left \| \mathbf{x}_0 + \int_0^{\Delta t} \mathbf{f}(\mathbf{x}(s), \mathbf{w}(s)) ds - \mathbf{x}_0 - \Delta t \cdot \mathbf{f}(\mathbf{x}(t), \mathbf{w}(t)))\right \|} \right ] \\
        &= \E \left [ \sqrt{\frac{2}{\gamma}} \| \psi \|_{\H_\X} \sqrt{ \left \| \int_0^{\Delta t} \mathbf{f}(\mathbf{x}(s), \mathbf{w}(s)) ds - \Delta t \cdot \mathbf{f}(\mathbf{x}(t), \mathbf{w}(t)))\right \|} \right ] \\
        &= \E \left [ \sqrt{\frac{2}{\gamma}} \| \psi \|_{\H_\X} \sqrt{ \left \| \int_0^{\Delta t} \mathbf{f}(\mathbf{x}(s), \mathbf{w}(s)) ds - \Delta t \cdot \mathbf{f}(\mathbf{x}(t), \mathbf{w}(t)) \right \|} \right ] \\
        &= \E \left [ \sqrt{\frac{2}{\gamma}} \| \psi \|_{\H_\X} \sqrt{ \left \| \int_0^{\Delta t} \mathbf{f}(\mathbf{x}(s), \mathbf{w}(s)) ds - \int_0^{\Delta t} \mathbf{f}(\mathbf{x}(t), \mathbf{w}(t)) ds \right \|} \right ] \\
        &\leq \E \left [ \sqrt{\frac{2}{\gamma}} \| \psi \|_{\H_\X} \sqrt{ 2 \| \mathbf{f} \|_\infty \Delta t } \right ] \\
        &= 2 \sqrt{\frac{\| \mathbf{f} \|_\infty \Delta t}{\gamma}} \| \psi \|_{\H_\X}  
     \end{aligned}
    \]
    Al ser esta cota uniforme en $\mathbf{x}_0$ se concluye que
    \[
    \| \Tilde{\U}_{\Delta t}\psi - \U_{\Delta t} \psi \| \leq 2 \sqrt{\frac{\| \mathbf{f} \|_\infty \Delta t}{\gamma}} \| \psi \|_{\H_\X}\] 
    con lo que
    \[
    \| \Tilde{\U}_{\Delta t} - \U_{\Delta t} \| \leq 2 \sqrt{\frac{\| \mathbf{f} \|_\infty \Delta t}{\gamma}}. \] 
\end{proof}

Primero, se deduce la dinámica continua del embedding
\[
\mu_{k+1} = \U^*_{\Delta t} \mu_k + \zeta_k = \E [X^+ | X = \mathbf{x}_k] + \Phi_\X (\mathbf{x}_{k+1}) - \E [X^+ | X = \mathbf{x}_k]  = \Phi_\X (\mathbf{x}_{k+1})
\]
con lo que
\[
\frac{\mu_{k+1} - \mu_k}{\Delta t} = \frac{\U^*_{\Delta t} \mu_k - \mu_k + \zeta_k}{\Delta t} = \frac{\Tilde{\U}^*_{\Delta t} \mu_k - \mu_k + \Tilde{\U}^*_{\Delta t} \mu_k - \U^*_{\Delta t} \mu_k + \zeta_k}{\Delta t}
\]
haciendo $\Delta t \searrow 0$ se obtiene
\[
\mu'(t) = \Tilde{\mathcal{L}}^* \mu(t) + \Tilde{\zeta} (t)
\]
con $\mu(t) = \Phi_\X (\mathbf{x}(t))$ y $\Tilde{\zeta}(t)$ algún proceso centrado con operador de covarianza $\mathcal{Q}(t)$.
Entonces se genera el siguiente sistema a tiempo continuo
\begin{equation*}
    \begin{aligned}
        \mu'(t) &= \Tilde{\mathcal{L}}^* \mu(t) + \Tilde{\zeta}(t) \\
        \mathbf{y}(t) &= \mathcal{G}^* \mu(t) + \nu (t)
    \end{aligned}
\end{equation*}
siendo $\nu$ un proceso centrado con función de covarianza $\mathcal{R}:[0, T] \to \R^{p \times p}$ que se supondrá invertible.

Luego, tiene el siguiente filtro de Kalman-Bucy asociado

\begin{equation*}
    \hat{\mu}'(t) = \Tilde{\mathcal{L}}^*\hat{\mu}(t)  + \K(t) \big( \mathbf{y}(t) - \mathcal{G}^*\hat{\mu}(t) \big),
\end{equation*}
\begin{equation*}
    \mathcal{P}'(t) = \Tilde{\mathcal{L}}^*\mathcal{P}(t) + \mathcal{P}(t)\Tilde{\mathcal{L}} - \mathcal{P}(t)\mathcal{G}\mathcal{R}^{-1}\mathcal{G}^*\mathcal{P}(t) + \mathcal{Q}(t).
\end{equation*}
\begin{equation*}
    \mathcal{K}(t) = \mathcal{P}(t)\mathcal{G}\mathcal{R}^{-1}(t).
\end{equation*}

Para $\Delta t > 0$ se propone aproximar la solución por la solución del siguiente sistema

\begin{equation*}
    \hat{\mu}'_{N, \Delta t}(t) = \Tilde{\mathcal{L}}_{N, \Delta t}^* \hat{\mu}_{N}(t)  + \K_{N, \Delta t}(t) \big( \mathbf{y}(t) - \mathcal{G}_N^*\hat{\mu}_{N, \Delta t}(t) \big),
\end{equation*}
\begin{equation*}
    \mathcal{P}_{N, \Delta t}'(t) = \Tilde{\mathcal{L}}_{N, \Delta t}^*\mathcal{P}_{N, \Delta t}(t) + \mathcal{P}_{N, \Delta t}(t)\Tilde{\mathcal{L}}_{N, \Delta t} - \mathcal{P}_{N, \Delta t}(t)\mathcal{G}_N\mathcal{R}_N^{-1}\mathcal{G}_N^*\mathcal{P}_{N, \Delta t}(t) + \mathcal{Q}_N(t).
\end{equation*}
\begin{equation*}
    \mathcal{K}_{N, \Delta t}(t) = \mathcal{P}_{N, \Delta t}(t)\mathcal{G}_N\mathcal{R}_N^{-1}(t).
\end{equation*}
\begin{equation*}
    \Tilde{\mathcal{L}}_{N, \Delta t} = \frac{\U_N - I}{\Delta t}
\end{equation*}

\begin{prop}
    Bajo las hipótesis del lema y del teorema \ref{teo:error_koop}, tomando $\gamma = N^{3/4}$, $\Delta t = N^{-1/4}$, se tiene que, para $\delta \in (0, 1)$, con probabilidad $1-\delta$ existe una constante $\Tilde{C}$ tal que
    \[
    \| \Tilde{\mathcal{L}}_{N, \Delta t} - \Tilde{\mathcal{L}}\| \leq \Tilde{C} N^{-1/4}
    \]
\end{prop}
\begin{proof}
    La demostración pasa por 3 aproximaciones intermedias
    \[
    \begin{aligned}
        \| \Tilde{\mathcal{L}} - \Tilde{\mathcal{L}}_{N, \Delta t}\| &= \left \| \Tilde{\mathcal{L}} - \left ( \frac{\Tilde{\U}_{\Delta t} - I}{\Delta t} \right ) + \left ( \frac{\Tilde{\U}_{\Delta t} - I}{\Delta t} \right ) - \left ( \frac{\U_{\Delta t} - I}{\Delta t} \right ) + \left ( \frac{\U_{\Delta t} - I}{\Delta t} \right ) - \left ( \frac{\U_N - I}{\Delta t} \right )  \right \| \\
        &\leq \left \| \Tilde{\mathcal{L}} - \left ( \frac{\Tilde{\U}_{\Delta t} - I}{\Delta t} \right ) \right \| + \left \| \left ( \frac{\Tilde{\U}_{\Delta t} - I}{\Delta t} \right ) - \left ( \frac{\U_{\Delta t} - I}{\Delta t} \right ) \right \| + \left \| \left ( \frac{\U_{\Delta t} - I}{\Delta t} \right ) - \left ( \frac{\U_N - I}{\Delta t} \right )  \right \| \\
        & \leq C \Delta t + \frac{1}{\Delta t} \left \|  \Tilde{\U}_{\Delta t} - \U_{\Delta t}  \right \| + \frac{1}{\Delta t} \left \| \U_{\Delta t} - \U_N \right \|.
    \end{aligned}
    \]
    Aplicando la proposición \ref{prop:aprox_koop_cont_disc} se tiene
        \[
    \begin{aligned}
        \| \Tilde{\mathcal{L}} - \Tilde{\mathcal{L}}_{N, \Delta t}\|
        & \leq C \Delta t + \frac{1}{\Delta t} \sqrt{\frac{\| \mathbf{f} \|_\infty \Delta t}{\gamma}} + \frac{1}{\Delta t} C N^{-1/2} \\
        & = C \Delta t + \frac{1}{\sqrt{\Delta t}} \sqrt{\frac{\| \mathbf{f} \|_\infty}{\gamma}} + \frac{1}{\Delta t} C N^{-1/2}.
    \end{aligned}
    \]
    Utilizando $\gamma = N^{3/4}$ y $\Delta t = N^{-1/4}$ se obtiene
    \[
    \begin{aligned}
        \| \Tilde{\mathcal{L}} - \Tilde{\mathcal{L}}_{N, \Delta t}\|
        & = C N^{-1/4} + \frac{1}{\sqrt{N^{-1/4}}} \sqrt{\frac{\| \mathbf{f} \|_\infty}{N^{3/4}}} + \frac{1}{N^{-1/4}} C N^{-1/2} \\
        & \leq C N^{-1/4} + \sqrt{\frac{\| \mathbf{f} \|_\infty}{N^{1/2}}} + C N^{-1/4} \\
        & \leq C N^{-1/4} + \sqrt{\| \mathbf{f} \|_\infty} N^{-1/4} + C N^{-1/4} \\
        & = \Tilde{C} N^{-1/4}
    \end{aligned}
    \]
\end{proof}

Entonces, se puede ver que
\[
\begin{aligned}
    \| \mathcal{P}_{N, \Delta t}(t) - \mathcal{P}(t) \| &= \left \| \mathcal{P}_{N, \Delta t}(0) - \mathcal{P}(0) + \int_0^t (\mathcal{P}'_{N, \Delta t}(s) - \mathcal{P}'(s))ds \right \| \\
    &\leq \| \mathcal{P}_{N, \Delta t}(0) - \mathcal{P}(0) \| + \int_0^t \| \mathcal{P}'_{N, \Delta t}(s) - \mathcal{P}'(s) \| ds \\
    &\leq C_1 N^{-1/2} + \int_0^t C_2(s) N^{-1/2} \|  \mathcal{P}_{N, \Delta t}(s) - \mathcal{P}(s) \| ds
\end{aligned}
\]
entonces por Gronwall-Bellman
\[
    \| \mathcal{P}_{N, \Delta t}(t) - \mathcal{P}(t) \| \leq C_1 N^{-1/2} \text{exp} \left ( N^{-1/4} \int_0^t C_2 (s) ds \right )
\]
con esta cota y haciendo lo análogo para $\hat{\mu}(t)$ se obtiene
\[
    \| \hat{\mu}_{N, \Delta t}(t) - \hat{\mu}(t) \| \leq C_3 N^{-1/2} \text{exp} \left ( N^{-1/4} \int_0^t C_4 (s) ds \right )
\]