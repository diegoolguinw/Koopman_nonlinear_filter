El problema de filtraje consiste en la reconstrucción de trayectorias a partir de observaciones parciales y ruidosas de un sistema dinámico. Este problema, recurrente en ingeniería y ciencias, ha captado la atención de diversas áreas de investigación y múltiples autores desde el siglo pasado. Su formalización inicial se atribuye a Wiener, mientras que Kalman proporcionó una solución óptima en el contexto lineal. Sin embargo, los trabajos de Stratonovich demostraron que, en el caso general, es imposible obtener soluciones óptimas y de dimensión finita, por lo que cualquier algoritmo solo puede devolver soluciones subóptimas.

Desde la época de Kalman, se han desarrollado diversos algoritmos para tratar dinámicas y observaciones no lineales, como el \textit{Extended Kalman Filter} y el \textit{Unscented Kalman Filter}. No obstante, estos métodos carecen de cotas de error asintóticas con respecto al óptimo del problema de filtraje. Por otro lado, se han propuesto soluciones asintóticamente óptimas, como los filtros de partículas y los \textit{Ensemble Kalman Filters}, los cuales, en un contexto general, presentan un error del orden de \( O(N^{-1/2}) \).

Estas técnicas son de particular interés en epidemiología, donde los modelos que surgen son intrínsecamente no lineales debido a las interacciones entre una población susceptible y una infectada. Por ello, todos los experimentos numéricos presentados en este trabajo se basan en modelos derivados del clásico modelo SIR de Kermack y McKendrick.

En este trabajo de tesis, se propone un algoritmo de filtraje no lineal para un caso general, el cual posee una cota de error de \( O(N^{-1/2}) \), lo que lo posiciona como una alternativa competitiva frente a los algoritmos existentes. Para su construcción, se emplea la teoría del operador de Koopman, cuyos orígenes se remontan a la década de 1930 con los trabajos de Koopman y Von Neumann. En la última década, esta teoría ha experimentado un resurgimiento significativo en las comunidades que estudian sistemas dinámicos basados en datos. La capacidad de aproximar el operador de Koopman permite transformar un sistema no lineal de dimensión finita en uno lineal, aunque de dimensión infinita.

La propuesta combina la teoría del operador de Koopman con la de los \textit{Reproducing Kernel Hilbert Spaces} (RKHS), ampliamente utilizada en el ámbito del aprendizaje de máquinas. A partir de las propiedades del operador de Koopman en estos espacios, se deduce una cota de error de aproximación del operador, se construye un filtro en dimensión infinita, probando su cota de error asociada.

Finalmente, se realizan experimentos numéricos para comparar el algoritmo propuesto con los filtros no lineales existentes. Además, se aplica el filtro en la estimación de parámetros, compitiendo con algoritmos de tipo MCMC que utilizan \textit{samplers} como \textit{Differential Evolution Metropolis} y \textit{No-U-Turn Sampler} (NUTS). Los resultados demuestran que el algoritmo propuesto no solo es competitivo, sino que en muchos casos supera a los métodos mencionados, incluso en términos de tiempo de ejecución.