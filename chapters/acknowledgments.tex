Primero, a mi pilar fundamental, mi familia, nada de lo que he logrado en mi vida sería posible sin ellos. A Rosa Cautivo, por ser mi sostén en todo momento, por su compañía incondicional, su enseñanza constante y su cariño de madre y abuela; cada palabra de esta tesis se debe a ella, y yo soy un simple escriba en este complejo proceso. A Matías Wende, por ser mi guía desde el inicio de este camino y por ser la voz de sabiduría cuando los caminos eran oscuros; sin duda, sin él hubiese errado mucho más de lo que lo he hecho en mi vida. También por su cariño de hermano, más necesario que nunca cuando todo se pone cuesta arriba.

A Evelyn Wende y Emilio Wende, quienes se durmieron en el camino de verme elegir mi carrera, por haberme formado con el espíritu firme de que lo único que toca después de caer es levantarse; espero que a través de los ojos de Matías y Rosa hayan podido ver cómo la semilla que plantaron creció en buena tierra. A Rosa María Wende y Pamela Poblete, por ser una compañía incondicional e inigualable y enseñarme que la unión nos hace más fuertes. Como familia, me han enseñado que las flores más bellas crecen en los ambientes más adversos.

A Andrés Olguín, Natalia Olguín y Luisa Pavéz por haber estado ahí durante estos años, a pesar de las distancias han podido estar presentes para mi. 

A mis amigos del colegio, Javier Rivera, Felipe Romero, Ignacio Galaz, Lucas Carnot, Fernanda Suárez, Josefa Peña y Hugo Marín, quienes pasaron de ser mis compañeros de puesto en kínder a ser mis compañeros para toda la vida. Gracias por enseñarme que las distancias y el tiempo no son nada en comparación con una amistad genuina.

A Camila Poblete, por acompañarme en todo este viaje, por vivir y comprender el frenesí de estos años, por soportarme, por ser mi cable a tierra cuando era necesario, por impulsar mis sueños y por apoyarme incondicionalmente. Agradezco a la vida que logró juntar nuestros caminos, que por más separados que se vieran, encontró un sendero por el cual hemos podido caminar juntos.

A las personas de Explora RM Sur Poniente 2019-2022, Luis Contreras, Sandra Rojas, Marianella Cofré, Lucía Nuñez, Paula Fredes, Paula Troncoso, Catalina Moya y Luz María Cortínez, por haber sido el primer equipo que formé en la facultad y por haberme confiado labores en épocas tan complejas como la pandemia y el regreso a la presencialidad. Lo que aprendí fue invaluable y resultó increíblemente útil a lo largo de mi carrera.

A mis amigos del RIP, Álvaro Márquez, Vicente Poblete, Diego Céspedes, Benjamín Bórquez, Rodrigo Altamirano, Ricardo Ziegele, Paolo Martiniello, Martín Berríos, Luciano Villarroel, Gonzalo Ovalle, Domingo Ruiz, Sebastián Flores y Matías Vera, por ser el primer grupo de amigos que tuve dentro del DIM, por estar siempre para mí. Con ustedes he tenido las conversaciones más profundas y también las más chistosas y ridículas; ambas me han marcado profundamente y le han dado un toque muy especial a mi vida.

Al equipo de coordinación 2023, a Luis Fuentes, María José Alfaro, Nicolás Cornejo, Victoria Andaur, Carlos Antil y Antonia Suazo, por verme comenzar en uno de los roles que más disfruté a lo largo de estos años, que me formó como persona en aspectos muy complejos y me enseñó cosas que no se aprenden en una sala de clases ni en un libro. Al equipo 2024, que me vio partir, Cristóbal Ramos, Magdalena Bravo, María José Ganora y Samuel Canupe, por recordarme el entusiasmo con el cual entré al equipo.

De todo este equipo, darle las gracias en particular a dos personas. A Luis, por haber sido mi primer compañero de auxiliar y por haber forjado, de ahí en adelante, una de las amistades más profundas, fructíferas y genuinas que he formado en mi vida; por darle calidez y vida al día a día, por las risas que son necesarias en los momentos más complejos. A María José Alfaro, por todas las experiencias vividas en tan poco tiempo, por enseñarme que nada es imposible, que lo único que nos separa de nuestros deseos y de la realidad es nuestra voluntad, por enseñarme demasiadas cosas sobre la vida, especialmente que son nuestras mayores dificultades en donde encontramos nuestras mayores fortalezas.

A Natacha Astromujoff, por darme toda su confianza para formar parte del equipo de coordinación, pero, sobre todo, por todas las conversaciones y consejos, por su tiempo, su comprensión y su enseñanza. A Éterin Jaña, por su pasión por llevar adelante una labor tremenda en docencia y por siempre tener una palabra y un abrazo para mí. A ambas, gracias por dejar en mí la vara alta de lo que un equipo de trabajo debe ser.

Al grupo de los Chacritas, a María José Alfaro, Matías Azócar, Nicolás Valenzuela, Vicente Salinas, Pablo Araya, Francisco Vásquez, Vicente Cabezas y demás, por todos los buenos momentos. A muchas otras personas del DIM, Amal Zgheib, Juan Pablo Sepúlveda, Javier Maass, Maite Barrera, Matías Ortíz, Tiare Letelier, Vicente Maturana, Félix Brokering, Axel Álvarez, Tomás Banduc y Yeniffer Muñoz. Y a Silvia Mariano, Karen Hérnandez y María Inés Rivera por ejercer labores fundamentales en el DIM-CMM.

%A los profesores que me dejaron ser su auxiliar, por haberme dado la confianza de hacer una de las cosas que más amo: enseñar. A Carlos Conca, Raúl Gouet, Axel Osses, Gabrielle Nornberg, Joaquín Fontbona, Claudio Muñoz, Héctor Ramírez y Francisco Vásquez. Y a todas las personas que, como estudiantes, me tuvieron que aguantar como su auxiliar.

%Al equipo de Salud Digital del CMM, Estefanía Fröhlich, Víctor Riquelme y Gloria Henríquez, y al equipo de México, Jorge X. Velasco, Tishbe Herrera, Ruth Corona, Adrián Acuña, Mario Santana e Irasema Pedroza, por haberme acompañado este último año.

A Héctor Ramírez y Axel Osses, no solo por haberme guiado en este año completo de tesis, sino también por todo el resto de experiencias en las que me han enseñado y hemos trabajado juntos; sin duda, han sido un pilar fundamental en mi formación a lo largo de estos años al darme la oportunidad de conocer muchas aristas de esta carrera y dejarme ver más allá de los teoremas y las demostraciones. Y a Joaquín Fontbona, Benjamín Herrmann y Gonzalo Ruz por haber formado parte de esta comisión y haberse unido en el último tiempo a este trabajo.