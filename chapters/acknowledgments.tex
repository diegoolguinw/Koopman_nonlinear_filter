Primero, a mi pilar fundamental, mi familia. A Rosa, por ser mi sostén en todo momento, por haber hecho posible cada cosa que he realizado durante estos años; cada palabra de esta tesis se debe a ella, y yo soy un simple escriba. A Matías, por ser mi guía desde el inicio de este camino y por ser la voz de sabiduría cuando los caminos eran oscuros; sin duda, sin él hubiese errado mucho más de lo que lo he hecho en mi vida.

A Evelyn y Emilio, quienes se durmieron en el camino de verme elegir mi carrera, por haberme formado con el espíritu firme de que lo único que toca después de caer es levantarse; espero que a través de los ojos de Matías y Rosa hayan podido ver cómo la semilla que plantaron creció en buena tierra. A Rosa María y Pamela, por ser una compañía inigualable y enseñarme que la unión nos hace más fuertes. Como familia, me han enseñado que las flores más bellas crecen en los ambientes más adversos.

A Andrés, por haber estado ahí en cada momento y siempre atento; a pesar de las distancias, hemos podido estar presentes el uno para el otro. A Natalia y Luisa, por haberme acogido como parte de su familia en todo momento.

A mis amigos del colegio, Javier, Felipe, Ignacio, Lucas, Hugo, Aquiles, Fernanda y Josefa, quienes pasaron de ser mis compañeros de puesto en kínder a ser mis compañeros para toda la vida. Gracias por enseñarme que las distancias no son nada en comparación con una amistad genuina.

A Camila, por acompañarme en todo este viaje, por vivir y comprender el frenesí de estos años, por soportarme, por ser mi cable a tierra cuando era necesario, por impulsar mis sueños y por apoyarme incondicionalmente. Agradezco a la vida que logró juntar nuestros caminos, que por más separados que se vieran, encontró un sendero por el cual hemos podido caminar juntos.

A las personas de Explora RM Sur Poniente 2019-2022, Luis, Sandra, Marianella, Lucía, Paula F., Paula T., Catalina y Luz María, por haber sido el primer equipo que formé en la facultad y por haberme confiado labores en épocas tan complejas como la pandemia y el regreso a la presencialidad. Lo que aprendí fue invaluable y resultó increíblemente útil en mis últimos años de carrera.

A mis amigos del RIP, Álvaro, Vicente, Diego, Benjamín, Rodrigo, Ricardo, Paolo, Martín, Luciano, Gonzalo, Domingo y, más recientemente, Sebastián y Matías, por ser el primer grupo de amigos que tuve dentro del DIM y de la facultad, por estar siempre para mí. Con ustedes he tenido las conversaciones más profundas y también las más chistosas y ridículas; ambas me han marcado profundamente y le han dado un toque muy especial a mi vida.

Al equipo de coordinación 2023, a Luis, María José, Nicolás, Victoria, Carlos y Antonia, por verme comenzar en uno de los roles que más disfruté a lo largo de estos años, que me formó como persona en aspectos muy complejos y me enseñó cosas que no se aprenden en una sala de clases ni en un libro. Al equipo 2024, que me vio partir, Cristóbal, Magdalena, María José G. y Samuel, por recordarme el entusiasmo con el cual entré al equipo.

De todo este equipo, darle las gracias en particular a dos personas. A Luis, por haber sido mi primer compañero de auxiliar y por haber forjado, de ahí en adelante, una de las amistades más profundas, fructíferas y genuinas que he formado en mi vida; por darle calidez y vida al día a día, por las risas que son necesarias en los momentos más complejos de cada semestre. A María José, por todas las experiencias vividas en tan poco tiempo, por enseñarme que nada es imposible, que lo único que nos separa de nuestros deseos y de la realidad es nuestra voluntad, por enseñarme demasiadas cosas sobre la vida, especialmente que son nuestras mayores dificultades en donde encontramos nuestras mayores fortalezas.

A Natacha, por darme toda su confianza para formar parte del equipo de coordinación, pero, sobre todo, por todas las conversaciones y consejos, por su tiempo, su comprensión y su enseñanza. A Eterin, por su pasión por llevar adelante una labor tremenda en docencia y por siempre tener una palabra para mí en todo momento. A ambas, gracias por dejar en mí la vara alta de lo que un equipo de trabajo debe ser.

Al grupo de los Chacritas, a Nicolás, Vicente, Pablo, Francisco, Sebastián, Javier, Bruno y demás, por todos los buenos momentos. A muchas otras personas del DIM, Amal, Juan Pablo, Javier, Matías, Tiare, Vicente, Félix, Axel, Tomás y Yeniffer. Y a Silvia, Karen y María Inés por ejercer labores fundamentales en el DIM-CMM.

A los profesores que me dejaron ser su auxiliar, por haberme dado la confianza de hacer una de las cosas que más amo: enseñar. A Carlos, Raúl, Axel, Gabrielle, Joaquín, Claudio, Héctor y Francisco. Y a todas las personas que, como estudiantes, me tuvieron que aguantar como su auxiliar.

Al equipo de Salud Digital del CMM, Estefanía, Víctor y Gloria, y al equipo de México, Jorge, Tishbe, Ruth, Adrián, Mario e Irasema, por haberme acompañado este último año.

A Héctor y Axel, no solo por haberme guiado en este año completo de tesis, sino también por todo el resto de experiencias en las que me han enseñado y hemos trabajado juntos; sin duda, han sido un pilar fundamental en mi formación a lo largo de estos años. Y a Joaquín, Benjamín y Gonzalo por haber formado parte de esta comisión y haberse unido en el último tiempo a este trabajo.