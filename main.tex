% Template:     Tesis LaTeX
% Documento:    Archivo principal
% Versión:      3.3.9 (12/04/2024)
% Codificación: UTF-8
%
% Autor: Pablo Pizarro R.
%        pablo@ppizarror.com
%
% Manual template: [https://latex.ppizarror.com/tesis]
% Licencia MIT:    [https://opensource.org/licenses/MIT]

% CREACIÓN DEL DOCUMENTO
\documentclass[
	spanish, % Idioma: spanish, english, etc.	
	letterpaper, oneside
]{book}

\usepackage{algorithm}
\usepackage{algpseudocode}
\usepackage{dsfont}
\usepackage{float}
\usepackage{wrapfig}

\newcommand{\indep}{\perp \!\!\! \perp}
\newcommand{\U}{\mathcal{U}}
\floatname{algorithm}{Algoritmo}

% INFORMACIÓN DEL DOCUMENTO
\def\documenttitle {Algoritmo de filtraje no lineal basado en operador de Koopman, aplicado a modelos en epidemiología}
\def\documentsubtitle {}
\def\degreetitle {
	Tesis para optar al grado de magíster en ciencias de la ingeniería, mención matemáticas aplicadas
	\bigbreak\vspace{0.3cm}
	Memoria para optar al título de ingeniero civil matemático
}

\def\universityname {Universidad de Chile}
\def\universityfaculty {Facultad de Ciencias Físicas y Matemáticas}
\def\universitydepartment {Departamento de Ingeniería Matemática}
\def\universitydepartmentimage {departamentos/uchile2}
\def\universitydepartmentimagecfg {height=3cm}
\def\universitylocation {Santiago de Chile}

% INTEGRANTES, PROFESORES Y FECHAS
\def\documentauthor {Diego Andrés Olguín Wende}
\def\documentdate {Marzo 2025}

\def\portrait {
	\begin{center}
	\vspace{0.8cm} ~ \\
	\MakeUppercase{\textbf{\documenttitle}} ~ \\
	\vspace{0.8cm}
	\MakeUppercase{\degreetitle} ~ \\
	\vfill
	\begin{tabular}{c}
		\MakeUppercase{\textbf{\documentauthor}} \\ \\
		\vspace{0.3cm} \\
		PROFESOR GUÍA: \\
		HÉCTOR ARIEL RAMÍREZ CABRERA \\
		PROFESOR CO-GUÍA: \\
		AXEL ESTEBAN OSSES ALVARADO \\
		\vspace{0.3cm} \\
		MIEMBROS DE LA COMISIÓN: \\
            JOAQUÍN FONTBONA TORRES \\
		GONZALO RUZ HEREDIA \\
		BENJAMÍN HERRMANN PRIESNITZ \\
		\vspace{0.3cm} \\
		Este trabajo ha sido parcialmente financiado por los proyectos: \\
		Fondecyts 1201982 y 1240200, \\ Fondo Conjunto de Cooperación Chile-México 2022: \\ `Modelamiento matemático de procesos epidémicos incorporando \\ estructura poblacional, regional y grupos de riesgo'\\ y CMM ANID Basal FB210005. \\
		\vspace{0.3cm} \\
		\MakeUppercase{\universitylocation} \\
		\MakeUppercase{\documentdate}
	\end{tabular}
	\end{center}
}
\def\abstracttable {
	\begin{tabular}{l}
		RESUMEN DE LA MEMORIA PARA OPTAR \\
		AL TÍTULO DE MAGÍSTER EN CIENCIAS \\
		DE LA INGENIERÍA CON MENCIÓN MATEMÁTICAS APLICADAS \\
		POR: \MakeUppercase{\documentauthor} \\
		FECHA: \MakeUppercase{\documentdate} \\
		PROF. GUÍA: HÉCTOR ARIEL RAMÍREZ CABRERA \\
		PROF. CO-GUÍA: AXEL ESTEBAN OSSES ALVARADO \\
	\end{tabular}
}

% IMPORTACIÓN DEL TEMPLATE
\input{template}

% INICIO DE LAS PÁGINAS
\begin{document}

% PORTADA
\templatePortrait

% CONFIGURACIÓN DE PÁGINA Y ENCABEZADOS
\templatePagecfg

% RESUMEN O ABSTRACT
\begin{abstractd}
	El problema de filtraje consiste en la reconstrucción de trayectorias a partir de observaciones parciales y ruidosas de un sistema dinámico. Este problema, recurrente en ingeniería y ciencias, ha captado la atención de diversas áreas de investigación y múltiples autores desde el siglo pasado. Su formalización inicial se atribuye a Wiener, mientras que Kalman proporcionó una solución óptima en el contexto lineal. Sin embargo, los trabajos de Stratonovich demostraron que, en el caso general, es imposible obtener soluciones óptimas y de dimensión finita, por lo que cualquier algoritmo computacional solo puede devolver soluciones subóptimas.

Desde la época de Kalman, se han desarrollado diversos algoritmos para tratar dinámicas y observaciones no lineales, como el \textit{Extended Kalman Filter} y el \textit{Unscented Kalman Filter}. No obstante, estos métodos carecen de cotas de error asintóticas con respecto al óptimo del problema de filtraje. Por otro lado, se han propuesto soluciones asintóticamente óptimas, como los filtros de partículas y los \textit{Ensemble Kalman Filters}, los cuales, en un contexto general, presentan un error del orden de \( O(N^{-1/2}) \).

Estas técnicas son de particular interés en epidemiología, donde los modelos que surgen son intrínsecamente no lineales debido a las interacciones entre una población susceptible y una infectada. Por ello, todos los experimentos numéricos presentados en este trabajo se basan en modelos derivados del clásico modelo SIR de Kermack y McKendrick.

En este trabajo de tesis, se propone un algoritmo de filtraje no lineal para un caso general, el cual posee una cota de error de \( O(N^{-1/2}) \), lo que lo posiciona como una alternativa competitiva frente a los algoritmos existentes. Para su construcción, se emplea la teoría del operador de Koopman, cuyos orígenes se remontan a la década de 1930 con los trabajos de Koopman y Von Neumann. En la última década, esta teoría ha experimentado un resurgimiento significativo en las comunidades que estudian sistemas dinámicos basados en datos. La capacidad de aproximar el operador de Koopman permite transformar un sistema no lineal de dimensión finita en uno lineal, aunque de dimensión infinita.

La propuesta combina la teoría del operador de Koopman con la de los \textit{Reproducing Kernel Hilbert Spaces} (RKHS), ampliamente utilizada en el ámbito del aprendizaje de máquinas. A partir de las propiedades del operador de Koopman en estos espacios, se deduce una cota de error de aproximación del operador, se construye un filtro en dimensión infinita, probando su cota de error asociada.

Finalmente, se realizan experimentos numéricos para comparar el algoritmo propuesto con los filtros no lineales existentes. Además, se aplica el filtro en la estimación de parámetros, compitiendo con algoritmos de tipo MCMC que utilizan \textit{samplers} como \textit{Differential Evolution Metropolis} y \textit{No-U-Turn Sampler} (NUTS). Los resultados demuestran que el algoritmo propuesto no solo es competitivo, sino que en muchos casos supera a los métodos mencionados, especialmente en términos de tiempo de ejecución.
\end{abstractd}

% DEDICATORIA
\begin{dedicatory}
    A Rosa y Emilio, \\
    por enseñarme lo bello de la realidad,\\
    a Matías y Evelyn, \\
    por enseñarme lo bello de soñar,\\
    y a mi yo de niño,\\
    que nunca dejó de creer que\\
    los sueños se hacen realidad.

\end{dedicatory}

% AGRADECIMIENTOS
\begin{acknowledgments}
	Primero, a mi pilar fundamental, mi familia, nada de lo que he logrado en mi vida sería posible sin ellos. A Rosa Cautivo, por ser mi sostén en todo momento, por su compañía incondicional, su enseñanza constante y su cariño de madre y abuela; cada palabra de esta tesis se debe a ella, y yo soy un simple escriba en este complejo proceso. A Matías Wende, por ser mi guía desde el inicio de este camino y por ser la voz de sabiduría cuando los caminos eran oscuros; sin duda, sin él hubiese errado mucho más de lo que lo he hecho en mi vida. También por su cariño de hermano, más necesario que nunca cuando todo se pone cuesta arriba.

A Evelyn Wende y Emilio Wende, quienes se durmieron en el camino de verme elegir mi carrera, por haberme formado con el espíritu firme de que lo único que toca después de caer es levantarse; espero que a través de los ojos de Matías y Rosa hayan podido ver cómo la semilla que plantaron creció en buena tierra. A Rosa María Wende y Pamela Poblete, por ser una compañía incondicional e inigualable y enseñarme que la unión nos hace más fuertes. Como familia, me han enseñado que las flores más bellas crecen en los ambientes más adversos.

A Andrés Olguín, Natalia Olguín y Luisa Pavéz por haber estado ahí durante estos años, a pesar de las distancias han podido estar presentes para mi. 

A mis amigos del colegio, Javier Rivera, Felipe Romero, Ignacio Galaz, Lucas Carnot, Fernanda Suárez, Josefa Peña y Hugo Marín, quienes pasaron de ser mis compañeros de puesto en kínder a ser mis compañeros para toda la vida. Gracias por enseñarme que las distancias y el tiempo no son nada en comparación con una amistad genuina.

A Camila Poblete, por acompañarme en todo este viaje, por vivir y comprender el frenesí de estos años, por soportarme, por ser mi cable a tierra cuando era necesario, por impulsar mis sueños y por apoyarme incondicionalmente. Agradezco a la vida que logró juntar nuestros caminos, que por más separados que se vieran, encontró un sendero por el cual hemos podido caminar juntos.

A las personas de Explora RM Sur Poniente 2019-2022, Luis Contreras, Sandra Rojas, Marianella Cofré, Lucía Nuñez, Paula Fredes, Paula Troncoso, Catalina Moya y Luz María Cortínez, por haber sido el primer equipo que formé en la facultad y por haberme confiado labores en épocas tan complejas como la pandemia y el regreso a la presencialidad. Lo que aprendí fue invaluable y resultó increíblemente útil a lo largo de mi carrera.

A mis amigos del RIP, Álvaro Márquez, Vicente Poblete, Diego Céspedes, Benjamín Bórquez, Rodrigo Altamirano, Ricardo Ziegele, Paolo Martiniello, Martín Berríos, Luciano Villarroel, Gonzalo Ovalle, Domingo Ruiz, Sebastián Flores y Matías Vera, por ser el primer grupo de amigos que tuve dentro del DIM, por estar siempre para mí. Con ustedes he tenido las conversaciones más profundas y también las más chistosas y ridículas; ambas me han marcado profundamente y le han dado un toque muy especial a mi vida.

Al equipo de coordinación 2023, a Luis Fuentes, María José Alfaro, Nicolás Cornejo, Victoria Andaur, Carlos Antil y Antonia Suazo, por verme comenzar en uno de los roles que más disfruté a lo largo de estos años, que me formó como persona en aspectos muy complejos y me enseñó cosas que no se aprenden en una sala de clases ni en un libro. Al equipo 2024, que me vio partir, Cristóbal Ramos, Magdalena Bravo, María José Ganora y Samuel Canupe, por recordarme el entusiasmo con el cual entré al equipo.

De todo este equipo, darle las gracias en particular a dos personas. A Luis, por haber sido mi primer compañero de auxiliar y por haber forjado, de ahí en adelante, una de las amistades más profundas, fructíferas y genuinas que he formado en mi vida; por darle calidez y vida al día a día, por las risas que son necesarias en los momentos más complejos. A María José Alfaro, por todas las experiencias vividas en tan poco tiempo, por enseñarme que nada es imposible, que lo único que nos separa de nuestros deseos y de la realidad es nuestra voluntad, por enseñarme demasiadas cosas sobre la vida, especialmente que son nuestras mayores dificultades en donde encontramos nuestras mayores fortalezas.

A Natacha Astromujoff, por darme toda su confianza para formar parte del equipo de coordinación, pero, sobre todo, por todas las conversaciones y consejos, por su tiempo, su comprensión y su enseñanza. A Éterin Jaña, por su pasión por llevar adelante una labor tremenda en docencia y por siempre tener una palabra y un abrazo para mí. A ambas, gracias por dejar en mí la vara alta de lo que un equipo de trabajo debe ser.

Al grupo de los Chacritas, a María José Alfaro, Matías Azócar, Nicolás Valenzuela, Vicente Salinas, Pablo Araya, Francisco Vásquez, Vicente Cabezas y demás, por todos los buenos momentos. A muchas otras personas del DIM, Amal Zgheib, Juan Pablo Sepúlveda, Javier Maass, Maite Barrera, Matías Ortíz, Tiare Letelier, Vicente Maturana, Félix Brokering, Axel Álvarez, Tomás Banduc y Yeniffer Muñoz. Y a Silvia Mariano, Karen Hérnandez y María Inés Rivera por ejercer labores fundamentales en el DIM-CMM.

%A los profesores que me dejaron ser su auxiliar, por haberme dado la confianza de hacer una de las cosas que más amo: enseñar. A Carlos Conca, Raúl Gouet, Axel Osses, Gabrielle Nornberg, Joaquín Fontbona, Claudio Muñoz, Héctor Ramírez y Francisco Vásquez. Y a todas las personas que, como estudiantes, me tuvieron que aguantar como su auxiliar.

%Al equipo de Salud Digital del CMM, Estefanía Fröhlich, Víctor Riquelme y Gloria Henríquez, y al equipo de México, Jorge X. Velasco, Tishbe Herrera, Ruth Corona, Adrián Acuña, Mario Santana e Irasema Pedroza, por haberme acompañado este último año.

A Héctor Ramírez y Axel Osses, no solo por haberme guiado en este año completo de tesis, sino también por todo el resto de experiencias en las que me han enseñado y hemos trabajado juntos; sin duda, han sido un pilar fundamental en mi formación a lo largo de estos años al darme la oportunidad de conocer muchas aristas de esta carrera y dejarme ver más allá de los teoremas y las demostraciones. Y a Joaquín Fontbona, Benjamín Herrmann y Gonzalo Ruz por haber formado parte de esta comisión y haberse unido en el último tiempo a este trabajo.
\end{acknowledgments}

% TABLA DE CONTENIDOS - ÍNDICE
\templateIndex

% CONFIGURACIONES FINALES
\templateFinalcfg

% ======================= INICIO DEL DOCUMENTO =======================

% Capítulo 1: Introducción
\chapter{Introducción}
\section{Introducción}

El problema de filtraje consiste en estimar la trayectoria de un sistema dinámico, potencialmente estocástico, que se asume, en general, no completamente observable y con observaciones contaminadas por ruido. Este problema tiene aplicaciones en diversas áreas de la ingeniería y las ciencias, tales como análisis de señales, astronomía, mecánica de fluidos, procesamiento de imágenes médicas y, en el contexto de este trabajo, epidemiología y dinámicas de enfermedades infecciosas.

Los primeros avances matemáticos relacionados con el problema de filtraje fueron realizados de manera independiente por Wiener \cite{Wiener1949ExtrapolationSeries, M.1950TheApplications.} y Kolmogorov \cite{Kolmogorov1940StationarySpace}, quienes propusieron aproximaciones estadísticas de carácter general. Dichos desarrollos sentaron las bases de una línea de investigación que aborda el filtraje en múltiples aplicaciones matemáticas con la intención de proporcionar soluciones prácticas.

El primer enfoque algorítmico para el problema de filtraje fue introducido por Kalman \cite{Kalman1960AProblems}, quien desarrolló una metodología que, para el caso lineal, permite calcular de manera exacta la solución en tiempo discreto. Posteriormente, Kalman y Bucy extendieron este enfoque al caso continuo \cite{Kalman1961NewTheory}, dando lugar a los algoritmos conocidos como filtro de Kalman y filtro de Kalman-Bucy, respectivamente.

Tras resolver el caso lineal, surgió la necesidad de abordar el problema de filtraje en su forma no lineal. Sin embargo, se ha demostrado que una solución general requiere algoritmos de dimensión infinita, lo cual limita su implementación práctica en computadoras y restringe las soluciones disponibles a métodos subóptimos.

Con el objetivo de responder a las demandas de áreas que trabajan con modelos no lineales, se han desarrollado metodologías que ofrecen aproximaciones subóptimas al problema de filtraje. Entre estas destacan el filtro de Kalman extendido \cite{Smith1962ApplicationVehicle, McElhoe1966AnVenus} y el filtro de Kalman no lineal o Unscented Kalman Filter \cite{Julier2004UnscentedEstimation}. Aunque estas metodologías son extensiones del filtro de Kalman, en ciertos casos proporcionan soluciones insatisfactorias y carecen de propiedades asintóticas que garanticen la convergencia al óptimo.

Otra aproximación ampliamente utilizada es la de los filtros de partículas \cite{Hammersley1954PoorCarlo, Liu1998SequentialSystems}, los cuales emplean métodos de Monte Carlo para aproximar la solución, representada como una esperanza condicional, manteniendo la estructura secuencial del problema. Debido a su naturaleza asintótica, estos filtros, bajo ciertas condiciones sobre el sistema y el resampleo de partículas, presentan cotas de error favorables para la solución del problema de filtraje \cite{Crisan2002APractitioners}.

En años recientes, se ha investigado la solución del problema de filtraje en espacios de Hilbert con núcleo reproducible (RKHS, por sus siglas en inglés). Estas estructuras, conocidas por sus aplicaciones en estadística, aprendizaje de máquinas y métodos numéricos \cite{Wendland2004ScatteredApproximation, Christmann2008SupportMachines}, ofrecen propiedades de interpolación y aproximación altamente valiosas. Entre los trabajos más relevantes en esta área destacan los de Fukumizu y Song \cite{Fukumizu2004DimensionalitySpaces, Song2009HilbertSystems}, quienes introdujeron la regla de Kernel Bayes, una generalización de la regla de Bayes clásica adaptada a espacios de Hilbert.

De manera complementaria, los métodos basados en datos aplicados a sistemas dinámicos han encontrado en el operador de Koopman una herramienta poderosa. Introducido por Koopman y Von Neumann \cite{Koopman1931HamiltonianSpace, Koopman1932DynamicalSpectra}, este operador permite representar sistemas dinámicos no lineales de dimensión finita como sistemas lineales en dimensión infinita. Los avances de Mezić \cite{Mezic2013AnalysisOperator}, Schmid \cite{Schmid2008DynamicData}, Surana \cite{Surana2016KoopmanSystems}, Brunton \cite{Brunton2016KoopmanControl} y otros, han ampliado el uso del operador de Koopman en áreas como mecánica de fluidos, donde los datos disponibles permiten identificar dinámicas complejas.

El presente trabajo de tesis se encuentra en la intersección de estas tres áreas, contribuyendo con el desarrollo de un algoritmo de filtraje que integra el filtro de Kalman, los RKHS y el operador de Koopman, denominado \textit{kernel Koopman Kalman Filter} (kerKKF). Esta contribución se ilustra en la figura \ref{fig:venn_diagram}.

Asimismo, este trabajo profundiza en la conexión entre el filtro de Kalman y los RKHS, destacada como conexión 1 en la figura \ref{fig:conexiones}, y emplea los resultados de aproximación del operador de Koopman en RKHS para establecer una cota de error para el algoritmo propuesto, representada como conexión 2. El objetivo principal es obtener una cota de error del orden $N^{-1/2}$, siendo $N$ la cantidad de puntos muestreados, lo cual sitúa este trabajo en un nivel comparable al de los filtros de partículas.

Por último, se incluyen resultados numéricos para modelos epidemiológicos, en particular variantes del modelo SIR clásico \cite{Hethcote1989ThreeModels, Grassly2008MathematicalTransmission}, y se explora la aplicación del filtro en la estimación de parámetros, demostrando su versatilidad y precisión.

Esta tesis está organizada en cinco capítulos que se describen a continuación:

En el \textbf{Capítulo 2}, se presentan los preliminares necesarios para el desarrollo del trabajo. Este capítulo está enfocado en la exposición formal del problema de filtraje, incluyendo su formulación matemática detallada y una revisión de los algoritmos existentes en la literatura. Además, se abordan los aspectos fundamentales de los espacios de Hilbert con núcleo reproducible (RKHS), esenciales para la comprensión del trabajo posterior.

El \textbf{Capítulo 3} contiene los resultados obtenidos en el ámbito de \textit{kernel Extended Dynamic Mode Decomposition} (kernel EDMD), un método ampliamente estudiado en la literatura. En este capítulo, se utiliza dicha técnica para construir las herramientas necesarias para el diseño del filtro y la obtención de cotas de error asociadas. El capítulo concluye con la presentación de resultados numéricos para casos lineales, así como para los modelos SIR y SIR con pérdida de inmunidad.

En el \textbf{Capítulo 4}, se introduce el algoritmo \textit{kernel Koopman Kalman Filter} (kerKKF) en dimensión infinita, describiendo tanto su diseño como el análisis del error asociado a la aproximación del filtro mediante operadores de dimensión finita. Se expone la idea de que este filtro pertenece a una clase óptima de algoritmos que se basen en métodos de Monte Carlo y una generalización de la metodología para creación de filtros que utilicen una técnica de EDMD. También se detalla una metodología para la estimación de parámetros constantes en sistemas dinámicos. Este capítulo incluye resultados numéricos que demuestran la efectividad del filtro en sistemas lineales y en modelos epidemiológicos, además de resultados relacionados con la estimación de parámetros.

El \textbf{Capítulo 5} está dedicado a las conclusiones, discusiones y direcciones futuras de esta investigación. En este capítulo se analizan las posibles contribuciones de este trabajo al campo del filtraje, así como las formas en que podría ser mejorado y extendido en investigaciones futuras.

Por último, los códigos utilizados en esta investigación están disponibles en un repositorio público de \href{https://github.com/diegoolguinw/Koopman_nonlinear_filter}{GitHub}, donde también se proporcionan instrucciones sobre las dependencias necesarias. La implementación del algoritmo kerKKF ha sido empaquetada como una librería instalable mediante \texttt{pip}, cuyos detalles pueden encontrarse en el repositorio de \href{https://github.com/diegoolguinw/kkf}{GitHub}.

\begin{figure}[h]
    \centering
    \begin{subfigure}[b]{0.45\linewidth}
        \includegraphics[width=0.9\linewidth]{img/content/chapter1/Venn_Diagram_Thesis.pdf}
        \caption{Diagrama de intersección de áreas como contribución de este trabajo.}
        \label{fig:venn_diagram}
    \end{subfigure}
    \begin{subfigure}[b]{0.45\linewidth}
        \includegraphics[width=0.9\linewidth]{img/content/chapter1/diag_contri.png}
        \caption{Diagrama de conexiones entre las áreas abordadas en este trabajo.}
        \label{fig:conexiones}
    \end{subfigure}
    \caption{Diagramas explicativos de las contribuciones de esta tesis.}
\end{figure}


% Capítulo 2: Problema resolver, relevancia, algoritmos existentes
% problemática. Estado del arte.
\chapter{Preliminares}
\section{Problema y algoritmos de filtraje}
Se considera un sistema controlado-observado estocástico a tiempo discreto
\begin{align*}
	\mathbf{x}_{k+1} &= \mathbf{f}(t_k, \mathbf{x}_k, \mathbf{u}_k, \mathbf{w}_k) \\
	\mathbf{y}_k &= \mathbf{g}(t_k, \mathbf{x}_k, \mathbf{u}_k, \mathbf{v}_k)
\end{align*}
Donde:
\begin{itemize}
	\item $\mathbf{f}: \R \times \R^{n_x} \times \R^{n_u} \times \R^{n_w} \to \R^{n_x}$ es la función de dinámica.
	\item $\mathbf{g}: \R \times \R^{n_x} \times \R^{n_u} \times \R^{n_v} \to \R^{n_y}$ es la función de observación
\end{itemize}
Siendo $n_x$ la dimensión del estado, $n_y$ la dimensión de la observación, $n_u$ la dimensión del \textit{input}, $n_w$ la dimensión del proceso estocástico en la dinámica y $n_v$ la dimensión del ruido en las observaciones. Se considera una condición inicial $\mathbf{x}_0$, por lo general desconocida sobre la que se coloca un \textit{prior} $X_0$. \\
Se supondrá que tanto $\mathbf{v}_k$ como $\mathbf{w}_k$ son vectores aleatorios centrados tal que $\mathbf{w}_k \indep \mathbf{w}_m, \, \forall k \neq m$, $\mathbf{v}_k \indep \mathbf{v}_m, \, \forall k \neq m$ y $\mathbf{w}_k \indep \mathbf{v}_k, \, \forall k$. Diremos que $\mathbf{w}$ y $\mathbf{v}$ son la perturbación en la dinámica y ruido en la medición, respectivamente. \\
El estudio de sistemas a tiempo continuo, formulados como
\begin{align*}
	\mathbf{x}'(t) &= \mathbf{f}(t, \mathbf{x}(t), \mathbf{u}(t), \mathbf{w}(t)) \\
	\mathbf{y}(t) &= \mathbf{g}(t, \mathbf{x}(t), \mathbf{u}(t), \mathbf{v}(t))
\end{align*}
muchas veces se reduce al caso discreto con esquemas de tipo Euler, por lo que durante esta tesis se mantendrá el foco en sistemas a tiempo discreto, a menos de que se necesite lo contrario. \\
El objetivo en el caso discreto será, en un instante $k$ encontrar un estimador de $\mathbf{x}_{k+\alpha}$, en base a observaciones $\{\Tilde{\mathbf{y}}_k\}_k$, que será denotado por $\hat {\mathbf{x}}_{k+\alpha}$. Dependiendo del valor de $\alpha$ el algoritmo recibirá una categoría diferente:
\begin{itemize}
	\item Predicción: $\alpha > 0$.
	\item Filtraje: $\alpha = 0$.
	\item Suavizado: $\alpha < 0$.
\end{itemize}
Así también se denotará a $\hat {\mathbf{x}}_{k_1|k_2}$ a la estimación del estado en la iteración $k_1$ dadas mediciones hasta la iteración $k_2$, entonces se dirá que $\hat{\mathbf{x}}_{k_1 | k_2}$ es un estado
\begin{itemize}
	\item Suavizado, si $k_2 > k_1$ ($t_2 > t_1$).
	\item Filtrado, si $k_2 = k_1$ ($t_2 = t_1$).
	\item Predicho, si $k_1 > k_2$ ($t_1 > t_2$).
\end{itemize}
El caso del filtraje será el centro durante este trabajo. Un algoritmo de filtraje busca estimar en el tiempo presente el estado en base a observaciones ruidosas, que puede ser entendido como el primer proceso en la estimación del estado de un sistema. \\
Posterior al filtraje de una medición ruidosa, se siguen los procesos de predicción y/o suavizado, siendo el primero la tarea poder estimar estados futuros en base al pasado, sin información del momento a estimar. Mientras que el suavizado es la capacidad de estimar momentos pasados, lo que se utiliza para mejorar las estimaciones presentes.\\
Se denota por $\mathbf{y}_{1:k}$, $\mathbf{u}_{0:k}$ a las observaciones e \textit{inputs}, respectivamente, hasta el tiempo $k \in \N$, a $k \in \N$ le llamaremos el horizonte del problema.  Notar que se considerará que no se tiene una observación de la condición inicial.\\ 
El problema de filtraje se puede formular como un problema de optimización del error cuadrático medio de la estimación, condicional a las observaciones y a los \textit{inputs}
\begin{equation*}
	(P) \quad
	\begin{cases}
		\begin{aligned}
			\text{mín} \quad & \quad \sum_{k=0}^N \E \left [ (\hat{\mathbf{x}}_{k|k} - \mathbf{x}_k)^T(\hat{\mathbf{x}}_{k|k} - \mathbf{x}_k) | \mathbf{y}_{0:k}, \, \mathbf{u}_{0:k}  \right] \\ 
			\text{s.a} \quad & \quad \mathbf{x}_{k+1} = \mathbf{f}(t_k, \mathbf{x}_k, \mathbf{u}_k, \mathbf{w}_k) \\
			\text{} \quad & \quad \mathbf{y}_k = \mathbf{g}(t_k, \mathbf{x}_k, \mathbf{u}_k, \mathbf{v}_k) \\
			\text{} \quad & \quad \mathbf{x}_0 \sim X_0
		\end{aligned}
	\end{cases}
\end{equation*}
Es entonces que el problema $(P)$ tiene como solución al estimador de mínimo error cuadrático. Es sabido de \cite{Kalman1960AProblems, Setoodeh2022NonlinearApplications} que este problema tiene un óptimo global dado por la esperanza condicional, que coincide con la noción de dicho estimador.
\begin{prop}
	El óptimo de $(P)$ viene dado por 
	\begin{equation*}
		\hat{\mathbf{x}}_{k|k} = \E \left [ \mathbf{x}_k | \mathbf{y}_{1:k}, \, \mathbf{u}_{0:k}  \right], \quad \forall k \in \{ 0, \dots, N \}
	\end{equation*}
\end{prop}
\noindent Aunque dicha cantidad no sea computable por métodos clásicos, como Monte Carlo, existen algoritmos para poder calcularla tiempo a tiempo, aunque sea de manera aproximada. 
\subsection{Caso a tiempo discreto, dinámica lineal y ruido centrado}
Para abordar el caso de sistemas dinámicos controlados-observados generales, es importante estudiar primero el caso lineal y con ruidos centrados y segundo momento finito, esto es
\begin{equation}
	\begin{cases}
		\mathbf{x}_{k+1} &= \mathbf{A}_k \mathbf{x}_k + \mathbf{B}_k \mathbf{u}_k + \mathbf{v}_k \\
		\mathbf{y}_k &= \mathbf{C}_k \mathbf{x}_k + \mathbf{D}_k \mathbf{u}_k  + \mathbf{w}_k
	\end{cases}
	\label{eq:lin_disc}
\end{equation}
con $\mathbf{A}_k \in \R^{n_x \times n_x}$, $\mathbf{B}_k \in \R^{n_x \times n_u}$, $\mathbf{C}_k \in \R^{n_y \times n_x}$, $\mathbf{D}_k \in \R^{n_y \times n_u}$,  $\mathbf{v}_k \sim \mathcal{N}(0, \mathbf{Q}_k)$ y $\mathbf{w}_k \sim \mathcal{N}(0, \mathbf{R}_k)$, $\mathbf{Q}_k \in \R^{n_x \times n_x}$, $\mathbf{R}_k \in \R^{n_y \times n_y}$. matrices de covarianzas, se tiene un algoritmo muy utilizado en ingeniería y ciencia, con suficiente fundamentos matemáticos e implementaciones eficientes, denotado el filtro de Kalman en honor a Rudolf E. Kalman, quién propuso el algoritmo por primera vez en \cite{Kalman1960AProblems}. Los detalles del algoritmo se presentan en el pseudo-código del Algoritmo \ref{alg:KF}. \\
\begin{algorithm}
	\caption{Filtro de Kalman}\label{alg:KF}
	\begin{algorithmic}[1]
		\State \textbf{Entrada:} Dinámica discreta como en (\ref{eq:lin_disc}), $\mathbf{x}_0$ \textit{prior} sobre la condición inicial, $\mathbf{y}_{1:N}$ observaciones, $\mathbf{u}_{0:N}$ \textit{inputs}.
		\State \textbf{Salida:} $(\hat{\mathbf{x}}_{k|k})_{k=0}^{N}$ estimador de la trayectoria y $(\hat{\mathbf{P}}_{k|k})_{k=0}^{N}$ matrices de covarianza.
		\State $\hat{\mathbf{x}}_{0|0}   \gets \E [\mathbf{x}_0]$
		\State $\mathbf{P}_{0|0} \gets \E [(\mathbf{x}_0 - \hat{\mathbf{x}}_{0})(\mathbf{x}_0 - \hat{\mathbf{x}}_{0})^T]$
		\For{$k = 0, \dots, N-1$}
		\State $\hat{\mathbf{x}}_{k+1|k} \gets \mathbf{A}_k \mathbf{x}_{k|k} + \mathbf{B}_k \mathbf{u}_k$
		\Comment{Estimación a priori}
		\State $\mathbf{P}_{k+1|k} \gets \mathbf{A}_k \mathbf{P}_{k|k} \mathbf{A}_k^T + \mathbf{Q}_k$
		\Comment{Error de covarianza a priori}
		\State $\hat{\mathbf{y}}_{k+1|k} \gets \mathbf{C}_{k+1} \hat{\mathbf{x}}_{k+1|k}$ 
		\Comment{Estimación de observación a priori}
		\State $\mathbf{e}_{\mathbf{y}_{k+1|k}} \gets \mathbf{y}_{k+1} - \hat{\mathbf{y}}_{k+1|k}$
		\Comment{Error a priori (innovación)}
		\State $\mathbf{K}_{k+1} \gets \mathbf{P}_{k+1|k} \mathbf{C}^T_{k+1} (\mathbf{C}_{k+1} \mathbf{P}_{k|k} \mathbf{C}^T_{k+1}+ \mathbf{R}_{k+1})^{-1}$
		\Comment{Ganancia de Kalman}
		\State $\hat{\mathbf{x}}_{k+1|k+1} \gets \hat{\mathbf{x}}_{k+1|k} + \mathbf{K}_{k+1} \mathbf{e}_{\mathbf{y}_{k+1|k}}$
		\Comment{Error a posteriori}
		\State $\mathbf{P}_{k+1|k+1} \gets (\mathbf{I} - \mathbf{K}_{k+1} \mathbf{C}_{k+1}) \mathbf{P}_{k+1|k}$
		\Comment{Error de covarianza a posteriori}
		\EndFor
	\end{algorithmic}
\end{algorithm}\\
El siguiente resultado le da toda su validez al algoritmo del Filtro de Kalman, que es probado en el artículo original \cite{Kalman1960AProblems}.
\begin{prop}
	El algoritmo del filtro de Kalman devuelve una secuencia $(\hat{\mathbf{x}}_{k|k})_k$ tal que 
	\begin{equation*}
		\hat{\mathbf{x}}_{k|k} = \E \left [ \mathbf{x}_k | \mathbf{y}_{1:k}, \, \mathbf{u}_{0:k}  \right], \quad \forall k \in \{ 0, \dots, N \}
	\end{equation*}
\end{prop}
\noindent Como consecuencia del hecho de que en el contexto Gaussiano el estimador de mínimo error cuadrático medio coincide con el estimador de máximo a posteriori, se tiene el siguiente resultado.
\begin{prop}
	El algoritmo del filtro de Kalman devuelve una secuencia $(\hat{\mathbf{x}}_{k|k})_k$ tal que 
	\begin{equation*}
		\hat{\mathbf{x}}_{k|k} \in \arg \max_{\hat{\mathbf{x}}_{k|k}} p (\hat{\mathbf{x}}_{k|k} | \mathbf{y}_{1:k}, \, \mathbf{u}_{0:k})
	\end{equation*}
\end{prop}

El algoritmo del Filtro de Kalman sigue un procedimiento de predicción y actualización:
\begin{itemize}
    \item Predicción: $\hat{\mathbf{x}}_{k|k} \to \hat{\mathbf{x}}_{k+1|k}$, donde solo es considerada la dinámica.
    \item Actualización: $\hat{\mathbf{x}}_{k+1|k} \to \hat{\mathbf{x}}_{k+1|k+1}$, donde se corrige la predicción en base a las observaciones.
\end{itemize}
Esta será la base para el resto de algoritmos. Para visualizar el comportamiento del algoritmo se dejan los siguientes diagramas, en donde se tiene un estado real o subyacente, en azul, y se le entrega al algoritmo una observación ruidosa, en amarillo.
\begin{itemize}
    \item Primera iteración: se comienza con un punto que, eventualmente está lejos de la condición inicial real del sistema, y luego se hace el proceso de predicción y actualización.
    \begin{figure}[h!]
        \centering
        \begin{subfigure}[b]{.49\linewidth}
        \centering
            \includegraphics[width=\linewidth]{img/content/chapter2/filt0.pdf}
            \caption{Condición inicial de la iteración.}
        \end{subfigure}
        \begin{subfigure}[b]{.49\linewidth}
        \centering
            \includegraphics[width=\linewidth]{img/content/chapter2/filt1.pdf}
            \caption{Predicción.}
        \end{subfigure}
        \begin{subfigure}[b]{.49\linewidth}
        \centering
            \includegraphics[width=\linewidth]{img/content/chapter2/filt2.pdf}
            \caption{Elección de la dirección de proyección.}
        \end{subfigure}
        \begin{subfigure}[b]{.49\linewidth}
        \centering
            \includegraphics[width=\linewidth]{img/content/chapter2/filt3.pdf}
            \caption{Proyección.}
        \end{subfigure}
    \end{figure}
    \item Segunda iteración: comenzando desde el mismo punto en donde terminó la iteración anterior se observa que el estado filtrado se acerca al estado real o subyacente.
        \begin{figure}[h!]
        \centering
        \begin{subfigure}[b]{.49\linewidth}
        \centering
            \includegraphics[width=\linewidth]{img/content/chapter2/filt4.pdf}
            \caption{Condición inicial de la iteración.}
        \end{subfigure}
        \begin{subfigure}[b]{.49\linewidth}
        \centering
            \includegraphics[width=\linewidth]{img/content/chapter2/filt5.pdf}
            \caption{Predicción.}
        \end{subfigure}
        \begin{subfigure}[b]{.49\linewidth}
        \centering
            \includegraphics[width=\linewidth]{img/content/chapter2/filt6.pdf}
            \caption{Elección de la dirección de proyección.}
        \end{subfigure}
        \begin{subfigure}[b]{.49\linewidth}
        \centering
            \includegraphics[width=\linewidth]{img/content/chapter2/filt7.pdf}
            \caption{Proyección.}
        \end{subfigure}
    \end{figure}
    \item Iteraciones futuras: se espera que el estado filtrado ya sea muy cercano, eventualmente idéntico, al estado subyacente, con lo que se dice que el filtro se estabiliza.
    \begin{figure}[h!]
        \centering
        \begin{subfigure}[b]{.49\linewidth}
        \centering
            \includegraphics[width=\linewidth]{img/content/chapter2/filt25.pdf}
            \caption{$k$-ésima iteración.}
        \end{subfigure}
        \begin{subfigure}[b]{.49\linewidth}
        \centering
            \includegraphics[width=\linewidth]{img/content/chapter2/filt26.pdf}
            \caption{Última iteración.}
        \end{subfigure}
    \end{figure}
\end{itemize}

\subsection{Caso a tiempo discreto no lineal general}
A diferencia del caso anterior, en el caso no lineal general hay muchos algoritmos propuestos, siendo los más clásicos las variantes del filtro de Kalman, como el \textit{Extended Kalman Filter} (EKF) y el \textit{Unscented Kalman Filter} (UKF), pero se sabe que resultan ser subóptimos \cite{Setoodeh2022NonlinearApplications}.\\ 
A pesar de que dichos algoritmos reciben dinámicas no lineales, siguen suponiendo que el ruido es aditivo, centrado y Gaussiano. Se verá en secciones posteriores que esto no es un impedimento para poder ejecutarlos, pero una familia de métodos se han construido para el caso más general, cuyos representantes más famosos son los filtros de partículas.\\
Con el motivo de no extender más este informe solo se expondrá el algoritmo de EKF. Para esta sección, se supondrá que la dinámica es de la forma
\begin{equation}
	\begin{aligned}
		\mathbf{x}_{k+1} &= \mathbf{f}(t_k, \mathbf{x}_k, \mathbf{u}_k) + \mathbf{w}_k \\
		\mathbf{y}_k &= \mathbf{g}(t_k, \mathbf{x}_k, \mathbf{u}_k) + \mathbf{v}_k 
	\end{aligned}
	\label{eq:no_lin_disc_add}
\end{equation} 
con $\mathbf{w}_k \sim \mathcal{N}(0, \mathbf{Q}_k)$ y $\mathbf{v}_k \sim \mathcal{N}(0, \mathbf{R}_k)$. 
El algoritmo de \textit{Extended Kalman Filter}, cuyo pseudo-código puede visualizarse en el algoritmo \ref{alg:EKF}, busca linearizar el sistema a primer orden vía su Jacobiano, tanto para $\mathbf{f}$ como para $\mathbf{g}$, de manera de generar las matrices $\mathbf{A}_k$ y $\mathbf{C}_k$ necesarias para el Filtro de Kalman, respectivamente. \\
A pesar de los simple que parece la adaptación de este algoritmo al caso no lineal, ilustra que extender el filtro de Kalman al caso no lineal se basa en una linealización de la dinámica, en este caso vía Jacobiano, pero podrían existir otras, que es en lo que se basará el filtro creado durante este trabajo.

\begin{algorithm}
	\caption{\textit{Extended Kalman Filter}}\label{alg:EKF}
	\begin{algorithmic}[1]
		\State \textbf{Entrada:} Dinámica discreta como en (\ref{eq:no_lin_disc_add}), $\mathbf{x}_0$ \textit{prior} sobre la condición inicial,  $\mathbf{y}_{1:N}$ observaciones, $\mathbf{u}_{0:N}$ \textit{inputs}.
		\State \textbf{Salida:} $(\hat{\mathbf{x}}_{k|k})_{k=0}^{N}$ estimador de la trayectoria y $(\hat{\mathbf{P}}_{k|k})_{k=0}^{N}$ matrices de covarianza.
		\State $\hat{\mathbf{x}}_{0|0}   \gets \E [\mathbf{x}_0]$
		\State $\mathbf{P}_0 \gets \E [(\mathbf{x}_0 - \hat{\mathbf{x}}_{0})(\mathbf{x}_0 - \hat{\mathbf{x}}_{0})^T]$
		\For{$k = 0, \dots, N-1$}
		\State $\mathbf{A}_k \gets \nabla _\mathbf{x} \mathbf{f} (t_k, \hat{\mathbf{x}}_{k|k}, \mathbf{u}_k)$
		\Comment{Linealización de la función de dinámica}
		
		\State $\hat{\mathbf{x}}_{k+1|k} \gets \mathbf{f}(t_k, \hat{\mathbf{x}}_{k|k}, \mathbf{u}_k)$
		\Comment{Estimación a priori}
		\State $\mathbf{P}_{k+1|k} \gets \mathbf{A}_k \mathbf{P}_{k|k} \mathbf{A}_k^T + \mathbf{Q}_k$
		\Comment{Error de covarianza a priori}
		\State $\mathbf{C}_{k+1} \gets \nabla_\mathbf{x} \mathbf{g} (t_k, \hat{\mathbf{x}}_{k+1|k}, \mathbf{u}_k)$
		\Comment{Linealización de la función de observación}
		\State $\hat{\mathbf{y}}_{k+1|k} \gets \mathbf{g}(t_k, \hat{\mathbf{x}}_{k+1|k}, \mathbf{x}_{k+1})$ 
		\Comment{Estimación de observación a priori}
		\State $\mathbf{e}_{\mathbf{y}_{k+1|k}} \gets \mathbf{y}_{k+1} - \hat{\mathbf{y}}_{k+1|k}$
		\Comment{Error a priori (innovación)}
		\State $\mathbf{K}_{k+1} \gets \mathbf{P}_{k+1|k} \mathbf{C}^T_{k+1} (\mathbf{C}_{k+1} \mathbf{P}_{k|k} \mathbf{C}^T_{k+1} + \mathbf{R}_{k+1})^{-1}$
		\Comment{Ganancia de Kalman}
		\State $\hat{\mathbf{x}}_{k+1|k+1} \gets \hat{\mathbf{x}}_{k+1|k} + \mathbf{K}_{k+1} \mathbf{e}_{y_{k+1|k}}$
		\Comment{Error a posteriori}
		\State $\mathbf{P}_{k+1|k+1} \gets (\mathbf{I} - \mathbf{K}_{k+1} \mathbf{C}_{k+1}) \mathbf{P}_{k+1|k}$
		\Comment{Error de covarianza a posteriori}
		\EndFor
	\end{algorithmic}
\end{algorithm}

El algoritmo del \textit{Unscented Kalman Filter} (UKF), pseudo-código del algoritmo \ref{alg:UKF}, se basa en generar puntos de muestra (llamados puntos sigma) alrededor de la estimación actual del estado del sistema. Estos puntos permiten representar la distribución estadística de la estimación sin necesidad de calcular derivadas. Al propagar estos puntos sigma a través del modelo no lineal, el UKF logra realizar estimaciones en el siguiente instante de tiempo, capturando de manera precisa las propiedades no lineales del sistema. Esta estrategia hace que el UKF sea una alternativa robusta al \textit{Extended Kalman Filter} (EKF) para el seguimiento y la estimación en sistemas no lineales. Aún así, el algoritmo puede ser costoso en la práctica y es necesaria una gran cantidad de puntos sigma para lograr una estimación fiable.

\begin{algorithm}[h!]
	\caption{\textit{Unscented Kalman Filter}}\label{alg:UKF}
	\begin{algorithmic}[1]
		\State \textbf{Entrada:} Dinámica discreta como en (\ref{eq:no_lin_disc_add}), $\mathbf{x}_0$ \textit{prior} sobre la condición inicial,  $\mathbf{y}_{1:N}$ observaciones, $\mathbf{u}_{0:N}$ \textit{inputs}.
		\State \textbf{Salida:} $(\hat{\mathbf{x}}_{k|k})_{k=0}^{N}$ estimador de la trayectoria y $(\hat{\mathbf{P}}_{k|k})_{k=0}^{N}$ matrices de covarianza.
		
		\State \textbf{Inicialización:}
		\State $\hat{\mathbf{x}}_{0|0}   \gets \E [\mathbf{x}_0]$
		\State $\mathbf{P}_0 \gets \E [(\mathbf{x}_0 - \hat{\mathbf{x}}_{0})(\mathbf{x}_0 - \hat{\mathbf{x}}_{0})^T]$
		
		\For{$k = 1, 2, \dots, N$}
		
		\State Calcular los puntos sigma $\chi$ usando la estimación del estado actual $\hat{\mathbf{x}}_{k-1}$ y la covarianza $\mathbf{P}_{k-1}$. Asignar pesos a los puntos sigma de acuerdo con pesos predefinidos $W_m$ (para la media) y $W_c$ (para la covarianza).
		
		\State $\chi_{k|k-1}^{(i)} \gets \mathbf{f}(t_k, \chi_{k-1}^{(i)}, \mathbf{u}_k)$
		\Comment Propagar cada punto sigma a través de la dinámica.
	
		\State $\hat{\mathbf{x}}_{k|k-1} \gets \sum_{i} W_m^{(i)} \chi_{k|k-1}^{(i)}$
		\Comment Calcular la media del estado predicho.
		
		\State Calcular la covarianza predicha:
		\[
		\mathbf{P}_{k|k-1} = \sum_{i} W_c^{(i)} \left( \chi_{k|k-1}^{(i)} - \hat{\mathbf{x}}_{k|k-1} \right) \left( \chi_{k|k-1}^{(i)} - \hat{\mathbf{x}}_{k|k-1} \right)^\top + \mathbf{Q}_k
		\]
		
		\State $\gamma_{k}^{(i)} = \mathbf{h}(t_k, \chi_{k|k-1}^{(i)}, \mathbf{u}_k)$
		\Comment{Pasar cada punto sigma predicho a través de la observación.}
		
		\State $\hat{\mathbf{y}}_k = \sum_{i} W_m^{(i)} \gamma_{k}^{(i)}$
		\Comment{Calcular la media de la medición predicha.}
		\State Calcular la covarianza de la medición:
		\[
		 \mathbf{S}_k = \sum_{i} W_c^{(i)} \left( \gamma_{k}^{(i)} - \hat{ \mathbf{y}}_k \right) \left( \gamma_{k}^{(i)} - \hat{\mathbf{y}}_k \right)^\top + \mathbf{R}_k
		\]
		\State Calcular la covarianza cruzada:
		\[
		\mathbf{C}_k = \sum_{i} W_c^{(i)} \left( \chi_{k|k-1}^{(i)} - \hat{\mathbf{x}}_{k|k-1} \right) \left( \gamma_{k}^{(i)} - \hat{\mathbf{x}}_k \right)^\top
		\]

		\State $\mathbf{K}_k = \mathbf{C}_k \mathbf{S}_k^{-1}$
		\Comment{Calcular la ganancia de Kalman.}
		\State $\hat{ \mathbf{x}}_{k|k} = \hat{ \mathbf{x}}_{k|k-1} +  \mathbf{K}_k ( \mathbf{y}_k - \hat{ \mathbf{y}}_k)$ 
		\Comment{Actualizar la estimación del estado.}
		\State $\mathbf{P}_{k|k} =  \mathbf{P}_{k|k-1} -  \mathbf{K}_k \mathbf{S}_k  \mathbf{K}_k^\top$
		\Comment{Actualizar la covarianza.}
		
		\EndFor
		
	\end{algorithmic}
\end{algorithm}

Los Filtros de Partículas (PF) o algoritmos de \textit{Sequential Monte Carlo} (SMC) \cite{Kemp2003AnMethods, Wills2023SequentialReview} son una familia de métodos que estiman el estado de sistemas dinámicos no lineales y/o no gaussianos mediante técnicas de muestreo tipo Monte Carlo. Estos algoritmos representan la distribución de probabilidad del estado mediante un conjunto de partículas, que son muestras aleatorias ponderadas según su probabilidad. La precisión de estos filtros depende críticamente del proceso de resampling, un paso fundamental que evita la degeneración de partículas al eliminar aquellas con pesos bajos y duplicar las más probables. Esto permite que los filtros de partículas mantengan una representación precisa de la distribución posterior en cada paso de tiempo, adaptándose dinámicamente a la complejidad del sistema.
Para el algoritmo se considerará que los modelos vienen representados por distribuciones de probabilidad, una de transición para la dinámica  \( p(\mathbf{x}_k | \mathbf{x}_{k-1}) \) y para la observación \( p(\mathbf{y}_k | \mathbf{x}_k) \).

\begin{algorithm}[h!]
	\caption{Filtro de Partículas}
	\begin{algorithmic}[1]
		
		\State \textbf{Entrada:} Modelo de dinámica \( p(\mathbf{x}_k | \mathbf{x}_{k-1}) \), modelo de medición \( p(\mathbf{y}_k | \mathbf{x}_k) \), observaciones \( \mathbf{y}_{1:k} \), número de partículas \( N_p \), estado inicial \( \{ \mathbf{x}_0^{(i)} \}_{i=1}^{N_p} \)
		\State \textbf{Salida:} Estimación del estado basada en la distribución ponderada de partículas \( \{ \mathbf{x}_k^{(i)}, w_k^{(i)} \}_{i=1}^{N_p} \)
		
		\For{$i = 1, \dots, N_p$}
		\State Generar la partícula inicial \( \mathbf{x}_0^{(i)} \) de la distribución inicial \( p(\mathbf{x}_0) \)
		\State Asignar el peso inicial \( w_0^{(i)} = \frac{1}{N_p} \)
		\EndFor
		
		\For{$k = 1, 2, \dots, N$}
		
		\For{$i = 1, \dots, N_p$}
		\State Muestrear \( \mathbf{x}_k^{(i)} \sim p(\mathbf{x}_k | \mathbf{x}_{k-1}^{(i)}) \) \Comment{Propagar cada partícula por la dinámica.}
		\EndFor
		
		\For{$i = 1, \dots, N_p$}
		\State \( w_k^{(i)} = w_{k-1}^{(i)} \cdot p(\mathbf{y}_k | \mathbf{x}_k^{(i)}) \) \Comment{Actualizar peso basado en la observación.}
		\EndFor
		
		\State \( w_k^{(i)} = \frac{w_k^{(i)}}{\sum_{j=1}^{N_p} w_k^{(j)}} \)
		\Comment{Normalizar los pesos.}
		
		\If{la degeneración de partículas es alta}
		\State Re-samplear las partículas \( \{ \mathbf{x}_k^{(i)}, w_k^{(i)} \}_{i=1}^{N_p} \) según sus pesos \( w_k^{(i)} \)
		\State Reiniciar pesos: \( w_k^{(i)} = \frac{1}{N_p} \) para todas las partículas
		\EndIf
		
		\EndFor
		
	\end{algorithmic}
\end{algorithm}

Un aspecto importante de los Filtros de Partículas es que su orden de convergencia ha sido demostrado como $O(N_p^{-1/2})$ \cite{Crisan2002APractitioners, Chopin2020AnCarlo}, es decir, la precisión del método aumenta proporcionalmente a la raíz cuadrada inversa del número de partículas utilizadas. Este número de partículas, que se elige como parámetro del filtro, controla directamente el error de estimación. Esta cota de error se tomará como referencia de comparación en el presente trabajo.

% RKHS
\section{Reproducing Kernel Hilbert Spaces}
Esta sección se basa en \cite{Wendland2004ScatteredApproximation} y \cite{Berlinet2004ReproducingStatistics}, aunque una versión más moderna de los resultados se puede encontrar en \cite{Saitoh2016TheoryApplications}. Sea \( \X \) un espacio topológico, y denote \(\C^\X\) al espacio de todas las funciones de \( \X \) en los números complejos. 

\begin{defn}[Reproducing Kernel Hilbert Space (RKHS)]  
Un espacio de Hilbert \( \mathcal{H} \subset \C^\X \) se dice un RKHS si existe una función \( k: \X \times \X \to \C \), llamada \textit{kernel reproduciente}, tal que:
\begin{enumerate}
    \item \( k_p \equiv K(\cdot, p) \in \mathcal{H}, \, \forall p \in E \).
    \item \( f(p) = \langle f, k_p \rangle_{\mathcal{H}}, \quad \forall p \in E, \, \forall f \in \mathcal{H}. \)
\end{enumerate}
\end{defn}

\noindent La segunda propiedad, conocida como \textit{propiedad reproduciente}, es fundamental en la teoría de los RKHS y da paso a muchas de las propiedades relevantes en estos espacios, como aquellas que se presentarán a continuación. Denotamos por \( \mathcal{H}_K(E) \) al espacio de Hilbert de funciones de \( E \) en \( \C \) cuyo kernel reproduciente es \( k \).

El siguiente lema proporciona una condición necesaria y suficiente para que un espacio de Hilbert dado posea un kernel reproduciente.

\begin{lema}
Un espacio de Hilbert \( \mathcal{H} \subset \C^\X \) posee un kernel reproduciente si y solo si los funcionales de evaluación \( e_p: \mathcal{H} \to \C \), definidos por \( e_p(f) = f(p) \), son continuos en \( \mathcal{H} \).
\end{lema}

\noindent Para construir un RKHS sobre un espacio topológico \( \X \), se introduce la siguiente definición.

\begin{defn}[Función semi-definida positiva]  
Una función \( k: \X \times \X \to \C \) se dice semi-definida positiva si para todo \( n \geq 1 \), \( (c_1, \dots, c_n) \in \C^n \), y \( (x_1, \dots, x_n) \in \X^n \), se cumple que:
\[
\sum_{i=1}^n \sum_{j=1}^n c_i \bar{c_j} K(x_i, x_j) \geq 0.
\]
\end{defn}

\noindent Esta propiedad es equivalente a que, para cada \( n \in \N \) y toda colección \( (x_1, \dots, x_n) \in \X^n \), la matriz  
\[
G = \left( k(x_i, x_j) \right)_{1 \leq i, j \leq n}
\]
sea Hermitiana y semi-definida positiva. Dicha matriz se denomina matriz Grammiana del kernel asociada a \( \{ x_i \}_{i=1}^n \), o simplemente matriz Grammiana cuando no haya lugar a confusión.

Finalmente, el siguiente lema describe cómo construir una función semi-definida positiva a partir de un espacio de Hilbert dado.

\begin{lema}
Sea \( \mathcal{H} \subset \C^\X \) un espacio de Hilbert con producto interno \( \langle \cdot, \cdot \rangle_\mathcal{H} \), y sea \( \varphi : \X \to \mathcal{H} \). Entonces, la función \( k: \X \times \X \to \C \), definida como
\[
k(x,y) = \langle \varphi(x), \varphi(y) \rangle_\mathcal{H},
\]
es semi-definida positiva.
\end{lema}

\noindent A continuación, dado un kernel \( k \) semi-definido positivo, se presentan un lema y un teorema que describen el comportamiento de un RKHS.

\begin{lema}
Sea \( \mathcal{H}_0 \) un subespacio de \( \C^\X \) en el cual se define un producto interno \( \langle \cdot, \cdot \rangle_{\mathcal{H}_0} \), con norma asociada \( \|\cdot\|_{\mathcal{H}_0} \). Entonces, para que exista un espacio de Hilbert \( \mathcal{H} \) tal que:
\begin{enumerate}
    \item[(1)] \( \mathcal{H}_0 \subset \mathcal{H} \subset \C^\X \), y la topología definida en \( \mathcal{H}_0 \) por el producto interno \( \langle \cdot, \cdot \rangle_{\mathcal{H}_0} \) coincida con la topología inducida en \( \mathcal{H}_0 \) por \( \mathcal{H} \),
    \item[(2)] \( \mathcal{H} \) posea un kernel reproduciente \( k \),
\end{enumerate}
es necesario y suficiente que:
\begin{enumerate}
    \item[(i)] Los funcionales de evaluación \( (e_t)_{t \in \X} \) sean continuos en \( \mathcal{H}_0 \),
    \item[(ii)] Cualquier sucesión de Cauchy \( (f_n) \) en \( \mathcal{H}_0 \) que converge puntualmente a cero, también converge a cero en el sentido de la norma.
\end{enumerate}
\end{lema}

\begin{teo}[Moore–Aronszajn \cite{Aronszajn1950TheoryKernels}]
Sea \( k \) una función semi-definida positiva en \( \X \times \X \). Entonces, existe un único espacio de Hilbert \( \mathcal{H} \subset \C^\X \) con \( k \) como kernel reproduciente. El subespacio \( \mathcal{H}_0 \) de \( \mathcal{H} \), generado por las funciones \( (k(\cdot, x))_{x \in \X} \), es denso en \( \mathcal{H} \), y \( \mathcal{H} \) coincide con el conjunto de funciones en \( \X \) que son límites puntuales de sucesiones de Cauchy en \( \mathcal{H}_0 \), con el producto interno definido por
\[
\langle f, g \rangle_{\mathcal{H}_0} = \sum_{i=1}^n \sum_{j=1}^m \alpha_i \bar{\beta}_j k(y_j, x_i),
\]
donde
\[
f(\cdot) = \sum_{i=1}^n \alpha_i k(\cdot, x_i), \quad g(\cdot) = \sum_{j=1}^m \beta_j k(\cdot, y_j),
\]
para \( (x_1, \dots, x_n) \in E^n \) y \( (y_1, \dots, y_m) \in \X^m \).
\end{teo}

\noindent El siguiente teorema permite transferir las propiedades de un RKHS a un espacio \( \ell^2 \), el cual resulta, en general, más sencillo de manipular.

\begin{teo}
Una función \( K: \X \times \X \to \C \) es un kernel reproduciente si y solo si existe una función \( \varphi: E \to \ell^2 (\Tilde{\X}) \), para algún espacio \( \ell^2(\Tilde{\X}) \), tal que para todo \( (x, y) \in \X \times \X \),
\[
k(x, y)= \langle \varphi(x), \varphi(y) \rangle_{\ell^2(\Tilde{\X})}.
\]
Además, el espacio \( \ell^2(\Tilde{\X}) \) es isométrico a \( \H_k(\X) \) a través de una isometría \( T: \X \to \ell^2(\Tilde{\X}) \), y la función \( \varphi \) viene dada por \( \varphi(x) = T(k(\cdot, x)) \). Dicha función \( \varphi \) se denomina el \textit{feature map} asociado a \( k \).
\end{teo}
En la literatura existen muchos kernels que son de utilidad en uno u otro contexto. A efectos de este trabajo se considerará el kernel de Matérn que, como se expondrá a lo largo de este escrito, posee muy buenas propiedades.
\begin{defn}
Se define el kernel de Matérn con parámetro de suavización \( \nu > 0 \) y ancho de banda \( \gamma > 0 \) como:
\[
k_{\nu}(x,y) = \frac{2^{1-\nu}}{\Gamma(\nu)} \left( \sqrt{2\nu} \frac{\|x-y\|}{\gamma} \right)^\nu B_\nu \left( \sqrt{2\nu} \frac{\|x-y\|}{\gamma} \right),
\]
donde \( \Gamma \) es la función Gamma y \( B_\nu \) es la función modificada de Bessel de segundo tipo de parámetro \( \nu \).

\end{defn}

El kernel de Matérn tiene una estrecha relación con los espacios de Sobolev fraccionarios, por lo que se introducen conceptos claves, como la definición de dicho espacio en $\R^d$ y la condición de cono interior.

\begin{defn}[Espacio de Sobolev fraccionario \cite{Adams2003SobolevSpaces}] Se define el espacio de Sobolev de regularidad $s > 0$ como
\[
H^s(\mathbb{R}^n) = \left\{ f \in L^2(\mathbb{R}^n) : \widehat{f}(\cdot)(1 + \|\cdot\|_2^2)^{s/2} \in L^2(\mathbb{R}^n) \right\},
\]
donde \( \widehat{f} \) denota la transformada de Fourier de \( f \).
\label{def:frac_sob}
\end{defn}

\begin{defn}[Condición de cono interior \cite{Wendland2004ScatteredApproximation}]
Un conjunto \( \Omega \subset \mathbb{R}^n \) satisface la condición del cono interior si existe un ángulo \( \theta \in (0, \pi/2) \) y un radio \( r > 0 \) tales que, para cada \( x \in \Omega \), existe \( \xi(x) \in \mathbb{R}^n \), con \( \|\xi(x)\|_2 = 1 \), de modo que el cono
\[
C(x, \xi(x), \theta, r) := \{x + \lambda y : y \in \mathbb{R}^n, \|y\| = 1, y^\top \xi(x) \geq \cos \theta, \lambda \in [0, r]\}
\]
está contenido en \( \Omega \).
\end{defn}
El siguiente teorema entrega una condición clave para los desarrollos de este trabajo, que es el hecho de que el kernel de Matérn es equivalente en norma a un espacio de Sobolev.
\begin{teo}[\cite{Wendland2004ScatteredApproximation, Tuo2016AProperties}]
Si \( \mathcal{X} \) es un conjunto compacto que satisface la condición del cono interior y \( k \) es el kernel de Matérn con parámetro \( \nu > 0 \), entonces para \( s = \nu + n/2 \) se cumple que \( \mathcal{H}_k (\mathcal{X}) \) es equivalente en norma a \( H^s (\mathcal{X}) \).
\end{teo}

\begin{prop}
    Si $\nu = 1/2$, se tiene que
    \[
    k_\nu (x, y) = \exp \left( -\frac{\|x-y\|}{\gamma} \right).
    \]
\end{prop}

\begin{proof}
    De \cite{Barton1965HandbookTables., Davis1944AFunctions} se sabe que
\end{proof}
    \[
B_{1/2} (z) = \sqrt{ \frac{\pi}{2} } \frac{e^{-z}}{\sqrt{z}},
\]
por lo que se deduce que:
\[
\begin{aligned}
k_\nu (x,y) &= \frac{\sqrt{2}}{\Gamma(1/2)} \left(  \frac{\|x-y\|}{\gamma} \right)^{1/2} \sqrt{ \frac{\pi}{2} } \exp \left( -\frac{\|x-y\|}{\gamma} \right) \left(  \frac{\|x-y\|}{\gamma} \right)^{-1/2} = \exp \left( -\frac{\|x-y\|}{\gamma} \right).
\end{aligned}
\]

Notar que la definición \ref{def:frac_sob} no es válida cuando se tiene un subconjunto de $\R^n$, pero la deducción de esta propiedad en \cite{Wendland2004ScatteredApproximation} pasa por extender las funciones a todo $\R^n$ y luego aplicar la definición. \\
En las secciones posteriores será de interés hacer un \textit{embedding} de una distribución en un RKHS, para lo que es necesario definir el elemento medio en un RKHS y los operadores de covarianza. Las definiciones y proposiciones que siguen han sido expuestas en \cite{Fukumizu2004DimensionalitySpaces, Song2009HilbertSystems, Muandet2017KernelBeyond}. \\
Sean $\X \subset \R^n$, $\Y \subset \R^p$,  Sea $\mu$ una medida de probabilidad sobre $\X$, $\H$ un RKHS sobre $\X$ con kernel reproduciente $k$ y feature map $\Phi$ y $X$ una variable aleatoria con soporte en $\X$ y $Y$ una variable aleatoria con soporte en $\Y$.
\begin{defn}[Elemento medio \cite{Song2009HilbertSystems}]
	Se define el elemento medio en el RKHS $\H$ como
	
	\begin{equation*}
		\hat \mu = \int_\X \Phi (x) d \mu (x) \in \H
	\end{equation*}
\end{defn}
Se define el producto Kronecker en el contexto de RKHS como un operador multiplicativo y que generaliza el producto Kronecker de vectores.
\begin{defn}[Producto de Kronecker]
    El producto Kronecker en un RKHS se define como
    el siguiente operador de rango 1 para $x, y \in \X$ fijos
	$$ \Phi_\X (x) \otimes \Phi_\X(y) : \H_\X \to \H_\X$$
	\begin{equation*}
		[\Phi_\X (x) \otimes \Phi_\X(y)] \psi = \langle \psi, \Phi_\X (y) \rangle \Phi_\X (x) = \psi (y) \Phi_\X (x)
	\end{equation*}
    \label{def:kronecker}
\end{defn}
Notar que en la última igualdad se utilizó la propiedad reproduciente. Esto motiva la definición del operador de covarianza.
\begin{defn}[Operador de covarianza]
    El operador de covarianza asociado al \textit{feature map} $\Phi$ como $C_X : \H_\X \to \H_\X$ 
    \begin{equation*}
        C_X = \int_\X \Phi_\X (x) \otimes \Phi_\X (x) d \mu_\X (x)
    \end{equation*}
\end{defn}
La definición del operador de covarianza cruzada entre dos variables aleatorias será fundamental en el desarrollo posterior como embedding de las variables en un espacio de Hilbert.
\begin{defn}[Operador de covarianza cruzada]
    El operador de covarianza cruzada asociado a las variables aleatorias $X$ e $Y$ se define como el operador $C_{XY}: \H_\X \to \H_\Y$	
    \begin{equation*}
        C_{X Y} = \E [\Phi_\X (X) \otimes \Phi_\Y (Y)] = \int_\X \int_\Y \Phi_\X (x) \otimes \Phi_\Y (y) \rho_g (x, dy) d \mu_\X (x)
    \end{equation*}
\end{defn}

	El adjunto de estos operadores es sencillo de calcular
	\begin{prop}
		Sea $x \in \X$ e $y \in \Y$, luego
		
		\begin{equation*}
			\left ( \Phi_\X (x) \otimes \Phi_\Y (y) \right )^* = \Phi_\Y (y) \otimes \Phi_\X (x)
		\end{equation*}
	\end{prop}
	\begin{proof}
		Sean $h_\X \in \H_\X$, $h_\Y \in \H_\Y$, luego
		\begin{equation*}
			\langle (\Phi_\X (x) \otimes \Phi_\Y (y)) h_\Y, h_\X \rangle = \langle h_\Y (y) \Phi_\X (x), h_\X \rangle = h_\Y (y) \langle \Phi_\X (x), h_\X \rangle
		\end{equation*}
		Por propiedad reproduciente
		\begin{equation*}
			h_\Y (y) \langle  \Phi_\X (x), h_\X \rangle = h_\Y (y) h_\X (x)
		\end{equation*}
		A la vez
		\begin{equation*}
			\langle h_\Y, (\Phi_\Y (y) \otimes \Phi_\X (x)) h_\X \rangle = \langle h_\Y, h_\X (x) \Phi_\Y (y) \rangle = h_\X (x) 
			\langle h_\Y, \Phi_\Y (y) \rangle =
		\end{equation*}
		Nuevamente, por propiedad reproduciente
		\begin{equation*}
			h_\X (x) 
			\langle h_\Y, \Phi_\Y (y) \rangle =h_\X (x) h_\Y (y) 
		\end{equation*}
		Y se concluye lo pedido.
	\end{proof}
	
	\begin{cor} $(C_{X})^* = C_X$.
	\end{cor}
	Dado que la integral es lineal
	\begin{equation*}
		(C_X)^* = \int_\X (\Phi_\X (x) \otimes \Phi_\X (x))^* d \mu_\X (x) = \int_\X \Phi_\X (x) \otimes \Phi_\X (x) d \mu_\X (x) = C_X
	\end{equation*}
	\begin{cor}
		\begin{equation*}
			(C_{XX^+})^* = C_{X^+X}, \quad (C_{XY})^* = C_{YX}
		\end{equation*}
	\end{cor}
	Esto es directo ya que el operador adjunto y la esperanza son lineales:
	\begin{equation*}
		(C_{X X^+})^* = \E [(\Phi_\X (X) \otimes \Phi_\X (X^+))^*] = \E [\Phi_\X (X^+) \otimes \Phi_\X (X)] = C_{X^+X} 
	\end{equation*}
	\begin{equation*}
		(C_{X Y})^* = \E [(\Phi_\X (X) \otimes \Phi_\Y (Y))^*] = \E [\Phi_\X (Y) \otimes \Phi_\X (X)] = C_{YX} 
	\end{equation*}
La herramienta clave para poder trabajar con el embedding de la esperanza condicional en un espacio de Hilbert será el operador de embedding condicional, cuya correcta definición se revisará en una proposición posterior.
\begin{defn}[Operador de embedding condicional]
    El operador de embedding condicional entre 2 distribuciones $X$ e $Y$, en $\X$ e $\Y$, respectivamente, como el operador $C_{X|Y} : \H_\X \to \H_\Y$ que satisface
    \begin{enumerate}
        \item $\mu_{Y|x} = \mathbb{E}_{Y|X}[\Phi_\Y (Y)|X=x] = C_{Y|X}\Phi_\X (x)$.
        \item $\mathbb{E}_{Y|X}[h(Y)|X=x] = \langle h, \mu_{Y|x} \rangle$.
    \end{enumerate}
\end{defn}
El siguiente teorema indica la existencia y forma del operador de embedding condicional \cite{Fukumizu2004DimensionalitySpaces, Song2009HilbertSystems}.
\begin{teo}
    Suponiendo que \( \mathbb{E}[h(Y)|X = \cdot] \in \mathcal{H}_X \) para cualquier \( h \in \mathcal{H}_Y \) entonces se tiene que
    \[ C_{Y|X} = C_{YX} C_{X}^{-1}.\]
\end{teo}

% Operador de Koopman
\section{Operador de Koopman para sistemas autónomos y deterministas}
El estudio de sistemas dinámicos, tanto a tiempo continuo como a tiempo discreto, ha sido un amplio campo de estudio en diferentes áreas de la matemática, ciencia e ingeniería. En el presente escrito se aboradará su estudio, tanto teórico como computacional, con el operador de Koopman. \\ Desde un punto de vista más matemático, un trabajo pionero fue el de Koopman en 1931 \cite{Koopman1931HamiltonianSpace} en donde se define el operador de Koopman para sistemas dinámicos con espectro discreto. Lo desarrollado por Koopman fue posteriormente generalizado por Koopman y von Neumann en 1932 \cite{Koopman1932DynamicalSpectra}.\\
En los tiempos recientes, el operador de Koopman ha vuelto a surgir en la literatura para el estudio de sistemas dinámicos no lineales, motivado principalmente por los trabajos de Mezić y Budišić junto a otros colaboradores \cite{Budisic2009AnObservables, Budisic2012GeometryFlows, Budisic2012AppliedKoopmanism}. \\
Aprovechando la linealidad del operador de Koopman, aunque sea a costa de trabajar ahora en dimensión infinita, es que se han propuesto técnicas para aproximar el operador vía matrices y así volver muchos problemas no lineales en tareas accesibles en un computador, en donde destacan los trabajos de Schmid \cite{Schmid2008DynamicData} en la creación del DMD y de Williams \cite{Williams2015ADecomposition} en la del EDMD.\\
Posterior a ello, trabajos más modernos se han movido a ver extensiones a otros tipos de sistemas, como sistemas con retardo o sistemas de ecuaciones en derivadas parciales, en donde no solo los trabajos de Mezić han seguido siendo relevantes \cite{Mezic2013AnalysisOperator, Mezic2020SpectrumGeometry, Mezic2022OnOperator}, sino que también los trabajos de Brunton y Kutz han marcado la pauta en ámbitos aún más aplicados \cite{Kaiser2021Data-drivenControl, Brunton2016KoopmanControl, Proctor2018GeneralizingControl, Lusch2018DeepDynamics, Kamb2020Time-delayApplications, NathanKutz2018AppliedSystems}, en donde incluso destacan ya implementaciones a nivel librería de los avances numéricos en la aproximación de Koopman \cite{Pan2024PyKoopman:Operator}.\\
Para introducir la idea y concepto del operador de Koopman a utilizar en este trabajo, se considera primero un sistema dinámico autónomo determinista.
\begin{equation}
	\mathbf{x}_{k+1} = \mathbf{f}(\mathbf{x}_k), \quad k \geq 0
	\label{eq:NL}
	\tag{NL}
\end{equation}
Con $\X \subset \R^n$ el espacio de estados. En este contexto, a las funciones $\varphi: \X \to \C$ se les denotará por \textit{observables} en $\mathcal{F}$ un espacio de Banach, con norma $||\cdot||_\mathcal{F}$. Con esto se puede definir el operador de Koopman asociado a una dinámica en tiempo discreto.
\begin{defn}[Operador de Koopman, tiempo discreto]
    El operador de Koopman asociado a $\mathbf{f}: \X \to \X$ es el operador $\U_\mathbf{f}: \mathcal{F} \to \mathcal{F}$ que se define mediante composición
    \begin{equation*}
	\U_\mathbf{f} \varphi = \varphi \circ \mathbf{f}, \quad \forall \varphi \in \mathcal{F}
    \end{equation*}
\end{defn}
\noindent El operador de Koopman da un panorama global sobre el sistema estudiado y además resulta ser un operador lineal, aunque infinito-dimensional. 
En lo que sigue se omitirá el subíndice $\mathbf{f}$ y se denotará simplemente por $\U$ y $||\U||$ será su norma de operador definida como
$$ ||\U|| = \sup \{ ||\U g||_\mathcal{F} \, : \, ||g||_\mathcal{F} = 1 \} $$
que depende del espacio $\mathcal{F}$ en donde se esté trabajando, aunque en cierto tipos de espacios se cumple lo siguiente.
\begin{prop}
	Para todo espacio de Banach $\mathcal{F}$ que contenga a las funciones constantes, se cumple que $||\U|| \geq 1$.
\end{prop}
Lo deseable para muchas aplicaciones es la continuidad del operador, incluso en muchos espacios se cumple que el operador tenga norma igual a $1$.
\begin{prop}
	Si $\mathcal{F} = L^\infty$, entonces el operador de Koopman es un operador con norma unitaria.
\end{prop}
Un caso aún más interesante, pero difícil de cumplir para un sistema dinámico cualquiera y complicado de comprobar en la práctica, es el caso de un sistema que preserve medida.
\begin{defn}[Sistema dinámico que preserva medida] Sea $\X$ un conjunto, $\mathcal{B}$ una $\sigma$-álgebra sobre $\X$, $\mu : \mathcal{B} \to [0, 1]$ una medida de probabilidad y  $F: \X \to \X$ una función medible. Se dice que $(\X, \mathcal{B}, \mu, \mathbf{f})$ preserva medida si $\mu(A) = \mu(\mathbf{f}^{-1}(A))$.
	
\end{defn}
\begin{prop}
	Sea $\mathcal{B}$ una $\sigma$-álgebra sobre $X$ y $(X, \mathcal{B}, \mu, \mathbf{f})$ un sistema dinámico que preserva medida, entonces si $\mathcal{F} = L^2 (\mu)$, se tiene que $||\U|| = 1$. Es más, $\mathcal{U}$ resulta ser un isomorfismo isométrico, esto es
	$$ \langle g, g \rangle = \langle \mathcal{U}g, \mathcal{U}g \rangle, \, \forall g \in \mathcal{F} $$
\end{prop}
Una motivación inicial para el estudio del operador de Koopman era poder analizar propiedades más explicativas de los sistemás dinámicos, como sus propiedades espectrales.
\begin{defn}[Valores y funciones propias]
	Una función propia del operador de Koopman asociado a la función $\mathbf{f}$ es una función $\phi_\lambda \in \mathcal{F} \setminus \{ 0 \}$ tal que 
	\begin{equation*}
		\U \phi_\lambda = \phi_\lambda \circ \mathbf{f} = \lambda \phi_\lambda
	\end{equation*}
	Con $\lambda \in \C$ su valor propio asociado.
\end{defn}
% \begin{prop}
% 	Para todo valor propio $\mu \in \C$ de $\U$, se tiene que $|\mu| \leq ||\U||$.
% \end{prop}
% \begin{proof}
	%     Sea $\mu \in \C$ valor propio de $\U$ y $\phi_\mu \in \mathcal{F} \setminus \{ 0 \}$ su función propia asociada, entonces
	%     $$||\mu \phi_\mu ||_\mathcal{F} = |\mu| ||\phi_\mu||_\mathcal{F} = ||\U \phi_\mu||_\mathcal{F} \leq ||\U|| \cdot ||\phi_\mu||_\mathcal{F} \leq  ||\phi_\mu||_\mathcal{F} $$
	%     Con lo que se obtiene $|\mu| \leq ||\U||$.
	% \end{proof}
Serán de interés las funciones propias y valores propios del operador de Koopman en el caso en que la dinámica del sistema sea lineal, en donde estas se corresponden con vectores y valores propios de la matriz que caracteriza el sistema.
\begin{prop}
	Si $\mathbf{f}(\mathbf{x}) = \mathbf{A}\mathbf{x}$, con $\mathbf{A} \in \R^{n \times n}$, $\mathcal{F} = C(\R^n; \C)$ y $\mathbf{A}$ posee valores propios $\{\mu_j\}_{j=1}^n$ y vectores propios por izquierda $\{ \mathbf{w}_j \}_{j=1}^n$, es decir $\mathbf{w}_j^T \mathbf{A} = \mu_j \mathbf{w}_j^T$, entonces $\{\mu_j\}_{j=1}^n$ son valores propios de $\U$ con funciones propias asociadas $\phi_j (\mathbf{x}) = \mathbf{w}_j^T \mathbf{x}$.
\end{prop}
% \begin{proof}
	%     Basta verificar que 
	%     \begin{equation*}
		%         \U \phi_j (x) = \phi_j (F(x)) = \phi_j (Ax) = w_j^T A x = \mu_j w_j^T x = \mu_j \phi_j (x)
		%     \end{equation*}
	% \end{proof}
%https://arxiv.org/pdf/1611.01209.pdf
En el contexto lineal, los valores propios $\{\mu_j\}_{j=1}^n$ y $\{ \phi_j\}_{j=1}^n$ se dicen valores y funciones propias principales, respectivamente. \\
Otra propiedad espectral importante es que el producto de funciones propias es función propia, en el caso en que esto haga sentido.
\begin{prop}
	Si $\mathcal{F}$ es cerrado para la multiplicación y $\phi_{\mu_1}, \, \phi_{\mu_2}$ son funciones propias de $\U$ con valores propios asociados $\mu_1$, $\mu_2$, respectivamente, entonces $\phi_{\mu_1} \phi_{\mu_2}$ es función propia de $\U$ con valor propio asociado $\mu_1 \mu_2$.
\end{prop}
% \begin{proof}
	%     Basta verificar que 
	%     \begin{equation*}
		%         \U(\phi_{\mu_1} \phi_{\mu_2})(x) = (\phi_{\mu_1} \phi_{\mu_2} ) (f(x)) = \phi_{\mu_1} (f(x))\phi_{\mu_2} (f(x)) = \mu_1 \phi_{\mu_1} \mu_2 \phi_{\mu_2}
		%     \end{equation*}
	%     Y notar que dado que $|\mu_1|$, $|\mu_2| \leq 1$, entonces $|\mu_1 \mu_2| \leq 1$.
	% \end{proof}
Con las propiedades espectrales del operador de Koopman, se puede entender una descomposición espectral en este contexto, siempre que el espectro sea discreto. 
\begin{defn}[Desarrollo en modos de Koopman]
Suponiendo que $\{ \phi_{\mu_j} \}_{j \geq 1}$, las funciones propias de $\U$, generan un denso en $\mathcal{F}$, entonces una expansión en modos de Koopman de una función $g$ en span$(\{ \phi_{\mu_j} \}_{j \geq 1})$ es
	\begin{equation*}
		g = \sum_{j \geq 1} \nu_j \phi_{\mu_j} 
	\end{equation*}
	A los coeficientes $\nu_j$ se les denota como los modos de Koopman de $g$.
\end{defn}
Esta definición para sistemas deterministas será la base para la definición en sistemas estocásticos que se presentará en la siguiente sección.
% El operador de Koopman permite obtener una recurrencia lineal en dimensión infinita mediante funciones observables. Se denota $g_0 \in \mathcal{F}$ y se genera la recursión
% \begin{equation}
	%      g_{k+1} = \U g_k
	%      \label{eq:koopman_linear}
	% \end{equation}
% Que es un problema lineal. Si $g_0$ es la identidad en $\mathcal{F}$ esto genera en sus evaluaciones lo siguiente
% \begin{align*}
	%     g_{1}(x_k) &= (\U g_0) (x_k) = g_0(F(x_k)) = g_0 (x_{k+1}) = x_{k+1} 
	% \end{align*}
% Es decir, al pasar la función identidad por el operador de Koopman, se génera la función \textit{shift} hacia la derecha. Así se obtiene
% \begin{prop}
	%     Si $g_0 \in \mathcal{F}$, $g_0(x) = x$, entonces se obtiene mediante la recursión
	%     $$ g_{k+1} = \U g_k $$
	%     Una secuencia de observables $\{ g_k \}_{k \geq 0} \subset \mathcal{F}$ tal que si $\{ x_k \}_{k \geq 0} \subset X$ es una secuencia generada por \ref{eq:state_aut}, entonces
	%     $$ g_n (x_k) = x_{n+k} $$
	% \end{prop}
% \begin{proof}
	%     Sea $k \in \N$ fijo, se procede por inducción. El caso base $k=1$ se demostró antes. Se supone que para $n \in \N$ $g_n (x_k) = x_{k+n}$, luego el paso inductivo es directo
	%     $$ g_{n+1}(x_k) = (\U g_n)(x_k) = g_n(F(x_k)) = g_n (x_{k+1}) = x_{n+k+1} $$
	%     Así se prueba por inducción lo pedido.
	% \end{proof}
% \begin{prop}
	%     Suponiendo que una función $g_0 \in span(\{ \phi_{\mu_j} \}_{j \geq 1})$ admite la descomposición en modos de Koopman
	%     $$g_0 = \sum_{j \geq 1} \nu_j \phi_{\mu_j}$$
	%     Y además que
	%     $$ \sum_{j \geq 1} || \nu_j ||_m ||\phi_{\mu_j}||_\mathcal{F} < \infty$$
	%     Se obtiene que la secuencia generada por \ref{eq:koopman_linear} tiene solución explícita para $k \geq 1$
	%     \begin{equation}
		%         g_k = \sum_{j \geq 1} \nu_j \mu_j^k \phi_{\mu_j}
		%         \label{eq:koopman_iter}
		%     \end{equation}
	% \end{prop}
% \begin{proof}
	%     Previo a mostrar lo pedido, hay que ver la convergencia de las series que definen a las $g_k$. Debido a la suposición de que $g_0$ está en el generado por las funciones propias de Koopman, entonces la serie de $g_0$ converge. Así
	%     \begin{align*}
		%         ||g_k|| &\leq \sum_{j \geq 1} || \nu_j ||_m \cdot |\mu_j^k| \cdot || \phi_{\mu_j} ||_\mathcal{F} \\ &\leq \max_j |\mu_j^k| \sum_{j \geq 1} || \nu_j ||_m \cdot || \phi_{\mu_j} ||_\mathcal{F} \\ & \leq \sum_{j \geq 1}  || \nu_j ||_m \cdot || \phi_{\mu_j} ||_\mathcal{F} \\
		%         &< \infty
		%     \end{align*}
	%     Es decir, la función queda bien definida y la serie converge. Ahora se prueba el caso base $k=1$:
	%     \begin{equation*}
		%         g_1 = \U g_0 = \U \left ( \sum_{j \geq 1} \nu_j \phi_{\mu_j} \right ) = \sum_{j \geq 1} \nu_j \U (\phi_{\mu_j} ) = \sum_{j \geq 1} \nu_j \mu_j \phi_{\mu_j} 
		%     \end{equation*}
	%     Así queda probado el caso base, en donde se ha utilizado la linealidad del operador de Koopman. Se supone que para $k \in \N$ se cumple que 
	%     \begin{equation*}
		%         g_k = \sum_{j \geq 1} \nu_j \mu_j^k \phi_{\mu_j}
		%     \end{equation*}
	%     Así, suponiendo que la serie converge se obtiene
	%     \begin{equation*}
		%         g_{k+1} = \U g_k = \U \left ( \sum_{j \geq 1} \nu_j \mu_j^k \phi_{\mu_j} \right ) = \sum_{j \geq 1} \nu_j \mu_j^k \U (\phi_{\mu_j}) = \sum_{j \geq 1} \nu_j \mu_j^{k+1} \phi_{\mu_j} 
		%     \end{equation*}
	%     Y por inducción queda probado lo pedido.
	% \end{proof}
% \section{Dynamic Mode Decomposition y Extended Dynamic Mode Decomposition}

% La \textbf{Dynamic Mode Decomposition (DMD)}, introducida originalmente en \cite{Schmid2008DynamicData}, constituye una de las primeras técnicas desarrolladas para aproximar el operador de Koopman mediante operadores de rango finito. Posteriormente, esta metodología fue mejorada con la introducción de la \textbf{Extended Dynamic Mode Decomposition (EDMD)}, presentada en \cite{Williams2015ADecomposition}, técnica que será empleada en este trabajo y que se detalla a continuación.

% Considérense las realizaciones de la dinámica, usualmente denominadas \textit{snapshots}, representadas como $(\{\mathbf{x}_j\}_{j=1}^N, \{\mathbf{y}_j\}_{j=1}^N)$, donde $\mathbf{y}_j = \mathbf{f}(\mathbf{x}_j)$. Sean además $\{\mathbf{\Psi}_k\}_{k=1}^{N_K}$, con $\mathbf{\Psi}_k: \mathcal{X} \to \mathbb{C}$, funciones generadoras del espacio $\mathcal{F}_{N_K} \subset \mathcal{F}$, que está finitamente generado. Se define la función de \textit{lifting forward} $\mathbf{\Psi}: \mathcal{X} \to \mathbb{C}^{N_K}$ como  
% \[
% \mathbf{\Psi}(\mathbf{x}) = (\mathbf{\Psi}_1(\mathbf{x}), \dots, \mathbf{\Psi}_{N_K}(\mathbf{x}))^T.
% \]
% Adicionalmente, se asume la existencia de una matriz de \textit{lifting back} $\mathbf{B} \in \mathbb{C}^{n \times N_K}$ tal que  
% \[
% \mathbf{B} \mathbf{\Psi}(\mathbf{x}) = \mathbf{x}, \quad \forall \mathbf{x} \in \mathcal{X}.
% \]

% El objetivo consiste en construir una aproximación lineal de la dinámica en una dimensión $N_K$, expresada como  
% \[
% \mathbf{z}_{k+1} = \mathbf{K} \mathbf{z}_k.
% \]
% Al aplicar $\mathbf{\Psi}$ a la dinámica, se obtiene  
% \[
% \mathbf{\Psi}(\mathbf{y}_j) = \mathbf{\Psi}(\mathbf{f}(\mathbf{x}_j)) = (\mathcal{U}\mathbf{\Psi})(\mathbf{x}_j) = \mathbf{K} \mathbf{\Psi}(\mathbf{x}_j) + \mathbf{r}(\mathbf{x}_j),
% \]
% donde $\mathbf{r} \in \mathcal{F}$ es una función de residuo que se busca minimizar sobre las realizaciones. Al definir las matrices  
% \[
% \mathbf{P}_X = (\mathbf{\Psi}(\mathbf{x}_1) \,|\, \dots \,|\, \mathbf{\Psi}(\mathbf{x}_N)) \in \mathbb{R}^{N_K \times N}, \quad \mathbf{P}_Y = (\mathbf{\Psi}(\mathbf{y}_1) \,|\, \dots \,|\, \mathbf{\Psi}(\mathbf{y}_N)) \in \mathbb{R}^{N_K \times N},
% \]
% el problema se formula como una regresión lineal:
% \[
% \mathbf{P}_Y = \mathbf{K} \mathbf{P}_X + \mathbf{R}(\mathbf{P}_X).
% \]
% Este problema puede escribirse como  
% \[
% \min_{\mathbf{K} \in \mathbb{R}^{N_K \times N_K}} J(\mathbf{K}) = \frac{1}{2} \|\mathbf{P}_Y - \mathbf{K} \mathbf{P}_X\|_F^2,
% \]
% donde $\|\cdot\|_F$ denota la norma de Frobenius, definida como  
% \[
% \|A\|_F = \sum_{i=1}^m \sum_{j=1}^n |a_{ij}|^2, \quad A \in \mathbb{R}^{m \times n}.
% \]

% Si $\mathbf{P}_X$ tiene rango completo, el problema admite un único mínimo global dado por  
% \[
% \mathbf{K} = \mathbf{P}_Y \mathbf{P}_X^\dagger,
% \]
% donde $\mathbf{P}_X^\dagger$ es la pseudoinversa de Moore-Penrose, definida según el rango de la matriz:
% \begin{itemize}
%     \item Si $\mathbf{A} \in \mathbb{R}^{n \times m}$ es inyectiva, entonces $\mathbf{A}^T \mathbf{A}$ es invertible y  
%     \[
%     \mathbf{A}^\dagger = (\mathbf{A}^T \mathbf{A})^{-1} \mathbf{A}^T.
%     \]
%     \item Si $\mathbf{A}$ es sobreyectiva, entonces $\mathbf{A} \mathbf{A}^T$ es invertible y  
%     \[
%     \mathbf{A}^\dagger = \mathbf{A}^T (\mathbf{A} \mathbf{A}^T)^{-1}.
%     \]
% \end{itemize}

% La matriz $\mathbf{K}$ constituye una aproximación finita del operador de Koopman. Denotando $\mathbf{z}_k = \mathbf{\Psi}(\mathbf{x}_k)$, se obtiene el sistema lineal en dimensión finita $N_K$:  
% \[
% \begin{aligned}
% \mathbf{z}_{k+1} &= \mathbf{K} \mathbf{z}_k, \\
% \mathbf{x}_k &= \mathbf{B} \mathbf{z}_k, \\ 
% \mathbf{z}_0 &= \mathbf{\Psi}(\mathbf{x}_0).
% \end{aligned}
% \tag{\(L\)}
% \]
% Un objetivo central de este trabajo es establecer resultados que permitan demostrar que las trayectorias o soluciones del sistema \((L)\) aproximan, en algún sentido, a las de la dinámica no lineal \((\text{NL})\).

% Todo lo expuesto se resume en el diagrama ilustrado en la Figura \ref{fig:KoopDiag}.  
% \begin{figure}[h]
% \centering
% \includegraphics[scale=0.75]{img/content/chapter2/KoopDiag.pdf}
% \caption{Diagrama de evolución temporal en dimensión infinita y finita, representando la dinámica mediante el operador de Koopman y sus aproximaciones. Elaboración propia, basado en \cite{Williams2015ADecomposition}.}
% \label{fig:KoopDiag}
% \end{figure}

% \section{Aproximación de rango bajo de matrices}
% La dimensión de aproximación del operador de Koopman puede ser muy alta para ciertas aplicaciones, lo que hace que ciertas operaciones se hagan muy inestables si se mantiene el rango original. Es por ello que en muchos contextos se propone bajar el rango de las matrices involucradas vía Descomposición en Valores Singulares (SVD). \\
% Primero consideremos el problema de obtener la mejor matriz de rango bajo que aproxima otra matriz dada. La solución de este problema fue dada en \cite{Eckart1936TheRank}, para ello se denota por $r(\mathbf{A})$ el rango de una matriz $\mathbf{A} \in \R^{m \times n}$.

% \begin{lema}
% 	La mejor aproximación de rango \( s \), en términos de la norma de Frobenius, para una matriz \( \mathbf{A} \) de rango \( t \) con \( t \geq s \), es decir, un minimizador global \( \hat{\mathbf{A}}^* \) de
% 	\[
% 	\min_{\hat{\mathbf{A}}} \| \mathbf{A} - \hat{\mathbf{A}} \|_F, \quad \text{s.a.} \quad r(\hat{\mathbf{A}}) \leq s
% 	\]
% 	viene dada por
% 	\[
% 	\hat{\mathbf{A}}^* = P_s (\mathbf{A}) = \mathbf{U} \Sigma_s \mathbf{V}^T,
% 	\]
% 	donde \( \Sigma_s \) es la matriz diagonal con los \( s \) valores singulares más grandes de \( \mathbf{A} \) y luego solo $0$, en donde se denota a la SVD de \( \mathbf{A} = \mathbf{U} \Sigma \mathbf{V}^T \).
% \end{lema}
% \noindent Es con esto que en \cite{Xiang2012OptimalMinimization} se provee una forma cerrada para un problema de regresión lineal en el que se busca que la matriz solución sea de rango bajo también, esto se formula como
% \begin{equation}
% 	\min_{\mathbf{M}} \| \mathbf{Y} - \mathbf{M}\mathbf{X} \|_F, \quad \text{s.a.} \quad r(\mathbf{M}) \leq s
% \end{equation}
% \begin{prop}
% 	Sea la SVD de $\mathbf{X} = \mathbf{U} \Sigma \mathbf{V}^T$, con $\textbf{U} \in \R^{m \times m}$, $\mathbf{V} \in \R^{n \times n}$ matrices ortogonales y $\Sigma$ la matriz diagonal con los valores singulares. Luego, el óptimo debe cumplir
% 	\[
% 	\mathbf{V}^T \mathbf{M}^* =
% 	\begin{bmatrix}
% 		\Sigma_{r(\mathbf{X})}^{-1} P_s(\mathbf{W}_{r(\mathbf{X})}) \\
% 		\mathbf{a}
% 	\end{bmatrix},
% 	\]
% 	donde $\Sigma_{r(\mathbf{X})}$ es la matriz diagonal con los valores singulares no nulos de $\mathbf{X}$, $\mathbf{W}_{r(\mathbf{X})}$ aquella con las primeras $r(\mathbf{X})$ filas de $\mathbf{W}$, con $\mathbf{W} = \mathbf{U}^T \mathbf{Y}$.
% \end{prop}
% \noindent Este resultado será importante para poder trabajar matrices de alta dimensionalidad, como ocurre en el caso de EDMD.


% Capítulo 3: Kernel Extended Dynamic Mode Decomposition 
\chapter{Kernel Extended Dynamic Mode Decomposition}
En lo que sigue se considerarán sistemas de la forma
\begin{align*}
	\mathbf{x}_{k+1} &= \mathbf{f}( \mathbf{x}_k, \mathbf{w}_k) \\
	\mathbf{y}_k &= \mathbf{g}(\mathbf{x}_k, \mathbf{v}_k)
\end{align*}
Es decir, en comparación al problema original, no se considerará ni el tiempo ni el \textit{input} en la dinámica. Esto no hace perder gran generalidad al problema, en primer lugar ya que el tiempo se puede agregar como una variable de estado de manera que si $\mathbf{x}_{k}^{n+1 = t_k}$, la $n+1$-ésima coordenada del estado, entonces 
\begin{equation*}
    \mathbf{x}_{k+1}^{n+1} = \mathbf{x}_{k}^{n+1} + \Delta t_k
\end{equation*}
donde $\Delta t_k$ es el paso de tiempo en el instante $k$, que en muchas aplicaciones es fijo. Para el \textit{input}, dado que no se está considerando como una variable de decisión, se considerará conocido o al menos que se tiene la capacidad de entregarse al sistema de manera indefinida, esto es, $\{ \mathbf{u}_k \}_{k \geq 0}$, es accesible y por tanto bastaría agregarlo también como una coordenada del estado: $\mathbf{x}_{k}^{n+1} = \mathbf{u}_k$.
\section{Kernel Extended Dynamic Mode Decomposition}

Denotemos por $\B_\X$ la $\sigma$-álgebra Boreliana de $\X$ y $\B_\Y$ a la de $\R^p$.
Se definen las medidas de probabilidad
\begin{equation*}
	\begin{aligned}
		\rho_f: \X \times \B_\X \to [0, 1], & \quad \rho_f (\mathbf{x}, A) = \P (\mathbf{f}(\mathbf{x}, \cdot ) \in A ) \\
		\rho_f: \X \times \B_\Y \to [0, 1], & \quad \rho_g(\mathbf{x}, A) = \P (\mathbf{g}(\mathbf{x}, \cdot ) \in A)
	\end{aligned}
\end{equation*}
Es decir, $\rho_f$ es la medida inducida por la dinámica y $\rho_g$ es la medida inducida por la observación.
\\
Supondremos que el espacio de estados $\X$ es compacto y que existe un conjunto compacto $\Y \subseteq \R^p$ tal que
\begin{equation*}
	\rho_f (\mathbf{x}, \X) = 1, \quad \rho_g (\mathbf{x}, \Y) = 1, \quad \forall \mathbf{x} \in \X
\end{equation*}
Adoptaremos la notación
\begin{equation*}
	\rho_f (\mathbf{x}, dx) = d \rho_f (\mathbf{x}, \cdot)(x), \quad \rho_g (\mathbf{x}, dy) = d \rho_f (\mathbf{x}, \cdot)(y)
\end{equation*}
Supondremos además que, para $\mu_\X$ medida sobre $\X$ y $\mu_\Y$ medida sobre $\Y$, existen \begin{equation*}
	p_f : \X \times \X \to \R_+, \quad p_g : \X \times \Y \to \R_+
\end{equation*}
tales que
\begin{equation*}
	\rho_f (\mathbf{x}, A) = \int_A p_f (\mathbf{x}, y) d \mu_\X (y), \quad \rho_g (\mathbf{x}, A) = \int_A p_g (\mathbf{x}, y) d \mu_\Y (y)
\end{equation*}
Llamaremos:
\begin{itemize}
	\item $X$ la variable aleatoria asociada a $\mu_\X$.
	\item $X^+ | \textbf{x}$ la variable aleatoria asociada a $\rho_\X$, es decir, la variable aleatoria asociada a avanzar un paso, dado $\textbf{x}$.
	\item $Y  | \textbf{x}$ la variable aleatorias asociada a $\rho_\Y$, es decir, la variable aleatoria asociada a la observación, dado $\textbf{x}$.
\end{itemize}
Por ejemplo, si la función de dinámica es de la forma
$$\mathbf{f}(\mathbf{x}_k, \mathbf{w}_k) = \Tilde{\mathbf{f}}(\mathbf{x}_k) + \mathbf{w}_k$$ 
con $\mathbf{w}_k \sim \mathcal{N}(0, \mathbf{Q}_k)$, siendo $\mathbf{Q}_k$ definida positiva y $\mu_\X$ la medida de Lebesgue sobre $\X$, se tiene que 
\begin{equation*}
	\rho_f (\mathbf{x}_k, \cdot) \sim \mathcal{N}(\tilde{\mathbf{f}}(\mathbf{x}_k), \mathbf{Q}_k)
\end{equation*}
con lo que $p_f$ es
\begin{equation*}
	p_f(x, y) = (2 \pi \text{det} \, \mathbf{Q}_k )^{-1} \text{exp} \left ( -(\tilde{\mathbf{f}}(x) - y)^\top \mathbf{Q}_k^{-1} (\tilde{\mathbf{f}}(x) - y) \right ) 
\end{equation*}
que cumple ser acotada, incluso si $\Tilde{\mathbf{f}}$ no lo es.\\
Similar a \cite{Philipp2024ErrorOperator} se asume lo siguiente
\begin{enumerate}
    \item[a)] $k_\X:\X \times \X \to \R$, $k_\Y: \Y \times \Y \to \R$ dos kernels simétricos, continuos, acotados y semi-definidos positivos. Se denotará por $\H_\X$, $\H_\Y$ a su RKHS asociados.
    \item[b)] Si $\psi_\X \in L^2(\X)$, $\psi_\Y \in L^2(\Y)$ son tales que 
    \begin{equation*}
        \int_{\X \times \X} k_\X(x,y) \psi_\X (x) \psi_\X (y) d \mu_\X (x) d \mu_\X (y) = 0 
    \end{equation*}
      \begin{equation*}
    	\int_{\Y \times \Y} k_\Y(x,y) \psi_\Y (x) \psi_\Y (y) d \mu_\Y (x) d \mu_\Y (y) = 0 
    \end{equation*}
    entonces $\psi_\X = 0$, $\psi_\Y = 0$ c.s.
    \item[c)] Si $\psi_\X \in \H_\X$, $\psi_\Y \in \H_\Y$ son tales que $\psi_\X(x) = 0$, $\psi_\Y(y) = 0$ para todo $x \in \X$ $\mu_\X$-c.s., y para todo $y \in \Y$ $\mu_\Y$-c.s. entonces $\psi_\X \equiv 0$, $\psi_\Y \equiv 0$.
    \item[d)] Se cumplen las siguientes relaciones de $\rho_\X$ y $\rho_\Y$ con respecto de $\mu_\X$ y $\mu_\Y$:
    \begin{equation*}
        \int_\X \rho_\X (x, A_\X) d\mu_\X (x) \leq L_\X \mu_\X (A_\X), \quad \forall A_\X \in \B_\X
    \end{equation*}
    \begin{equation*}
    	\int_\X \rho_\Y (x, A_\Y) d\mu_\X (x) \leq L_\Y \mu_\Y (A_\Y), \quad  \forall A_\Y \in \B_\Y
    \end{equation*}
\end{enumerate}
Un ejemplo de kernel que cumple a) y b) es Matérn, siendo el punto a) expuesto en la sección anterior y b) debido a la universalidad en $L^2$. La suposición c) se cumple si $\mu$ tiene densidad con respecto a Lebesgue, mientras que d) se cumple en el caso estocástico cuando existen las funciones $p_f$ y $p_g$ y estas son acotadas, mientras que en el caso determinista se necesita que las funciones asociadas sean difeomorfismos.

\begin{prop}
    Si $p_f \in L^{\infty} (\X \times \X)$, $p_g \in L^{\infty} (\X \times \Y)$ entonces 
    \begin{equation*}
        \int_\X \rho_\X (x, A) d\mu_\X (x) \leq L_\X \mu_\X (A), \quad \int_\X \rho_\Y (x, A) d\mu_\X (x) \leq L_\Y \mu_\Y (A)
    \end{equation*}
    con 
    \begin{equation*}
        L_\X = \mu_\X (\X) || p_f ||_\infty, \quad L_\Y = \mu_\X (\X) || p_g ||_\infty
    \end{equation*}
\end{prop}

\begin{proof}
	Para $\rho_\X$ se tiene que
    \begin{equation*}
        \begin{aligned}
            \int_\X \rho_\X (x, A) d\mu_\X (x) &= \int_\X \int_A d \rho (x, \cdot) (y) d \mu (x) \\
            &= \int_\X \int_A p_\mathbf{f} (x, y) d \mu_\X (y) d \mu_\X (x) \\
            &\leq || p_f ||_\infty \int_\X \int_A d \mu_\X (y) d \mu_\X (x) \\ 
            &= || p_f ||_\infty \mu_\X (\X) \mu_\X (A) \\
            &= L_\X \mu_\X (A)
        \end{aligned}
    \end{equation*}
    Mientras que para $\rho_\Y$ se tiene que
     \begin{equation*}
    	\begin{aligned}
    		\int_\X \rho_\Y (x, A) d\mu (x) & = \int_\X \int_A d \rho_\Y (x, \cdot) (y) d \mu_\X (x) \\
    		&= \int_\X \int_A p_g(x, y) d \mu_\Y (y) d \mu_\X (x) \\
    		&\leq || p_g ||_\infty \int_\X \int_A d \mu_\Y (y) d \mu_\X (x) \\ 
    		&= || p_g ||_\infty \mu_\X (\X) \mu_\Y (A) \\
    		&= L_\Y \mu_\Y (A)
    	\end{aligned}
    \end{equation*}
\end{proof}
En el caso en que la dinámica o la observación sean deterministas, que es equivalente a que $\mathbf{f}(\mathbf{x}, \mathbf{w}) = \Tilde{\mathbf{f}}(\mathbf{x})$ o $\mathbf{g}(\mathbf{x}, \mathbf{v}) = \Tilde{\mathbf{g}}(\mathbf{x})$, con lo que $\rho (x, \cdot) = \delta_{\Tilde{\mathbf{f}}(x)}(\cdot)$ o $\xi (x, \cdot) = \delta_{\Tilde{\mathbf{f}}(x)}(\cdot)$, entonces se necesita mayor regularidad sobre las funciones.
\begin{prop}
    Si $\rho (x, \cdot) = \delta_{\Tilde{\mathbf{f}}(x)}(\cdot)$ o $\xi (x, \cdot) = \delta_{\Tilde{\mathbf{f}}(x)}(\cdot)$ y $\tilde{\mathbf{f}}$ o $\tilde{\mathbf{g}}$, según corresponda, son difeomorfismos $C^1$ tal que
    \begin{equation*}
        \inf_{x \in \X} | \text{det} \,  D \tilde{\mathbf{f}} (x) | > 0, \quad \inf_{x \in \X} | \text{det} \,  D \tilde{\mathbf{g}} (x) | > 0
    \end{equation*}
    entonces
    \begin{equation*}
        \int_\X \rho (x, A) d\mu (x) \leq L_\rho \mu (A), \quad \int_\X \xi (x, A) d\mu (x) \leq L_\xi \mu (A)
    \end{equation*}
    con 
    \begin{equation*}
        L_\rho = || \text{det} \, D \tilde{\mathbf{f}}^{-1} ||_\infty, \quad L_\xi = || \text{det} \, D \tilde{\mathbf{g}}^{-1} ||_\infty
    \end{equation*}
\end{prop}

\begin{proof}
    La demostración se hace similar a \cite{Kohne2024L-errorDecomposition}. Notar que para $A \in \B$, se tiene que $\rho (x, A) = \delta_{\Tilde{\mathbf{f}}(x)}(A) = \mathds{1}_A (\Tilde{\mathbf{f}}(x))$, con ello
    \begin{equation*}
        \begin{aligned}
            \int_\X \rho (x, A) d\mu (x) &= \int_\X \mathds{1}_A (\mathbf{f}(x)) d \mu (x) \\
            &= \int_\X \mathds{1}_A (x) | \text{det} \, D \mathbf{f}^{-1}(x) | d \mu (x) \\
            &\leq || \text{det} \, D \mathbf{f}^{-1} ||_{\infty}  \int_\X \mathds{1}_A (x)  d\mu (x) \\
            &\leq || \text{det} \, D \mathbf{f}^{-1} ||_{\infty}  \mu (A) \\
            &= L_\rho \mu (A)
        \end{aligned}
    \end{equation*}
    Y para $\xi$ es análogo.
\end{proof}
Ahora se definen los operadores de Koopman estocásticos para la dinámica y la observación, adaptados en el caso en que se tienen las funciones $p_f$ y $p_g$ como densidades.
\begin{defn}[Operador de Koopman estocástico para la dinámica]
	Se define el operador asociado a $\mathbf{f}$ como $\U : L^2(\X) \to L^2(\X)$
	\begin{equation*}
		[\U h](x) = \E [h (\mathbf{f} (x, \cdot) )]  = \int_\X h(y) d \rho_\X (x, \cdot) (y) = \int_\X h(y) p_f (x, y) d \mu_\X (y)
	\end{equation*}
\end{defn}
\begin{defn}[Operador de Koopman estocástico para la observación]
	Se define el operador asociado a $\mathbf{g}$ como $\G : L^2(\Y) \to L^2(\X)$
	\begin{equation*}
		[\G h](x) = \E [h (\mathbf{g} (x, \cdot) )]  = \int_\Y h(y) d \rho_\Y (x, \cdot) (y) = \int_\Y h(y) p_g (x, y) d \mu_\Y (y)
	\end{equation*}
\end{defn}
Un objeto que tendrá interés pronto será el operador de Perron-Frobenius
\begin{defn}[Operador de Perron-Frobenius estocástico para la dinámica]
	Se define el operador asociado a $\mathbf{f}$ como $\mathcal{P}_\mathbf{f} : L^2(\X) \to L^2(\X)$
	\begin{equation*}
		[\mathcal{P}_\mathbf{f} h](x) = \int_\X h(y) p_\mathbf{f} (y, x) d \mu_\X (y)
	\end{equation*}
\end{defn}
\begin{defn}[Operador de Perron-Frobenius estocástico para la observación]
	Se define el operador asociado a $\mathbf{g}$ como $\mathcal{P}_\mathbf{g} : L^2(\X) \to L^2(\Y)$
	\begin{equation*}
		[\mathcal{P}_\mathbf{g} h](x) =  \int_\X h(y) p_\mathbf{g} (y, x) d \mu_\X (y)
	\end{equation*}
\end{defn}
Entonces notar que, gracias a la representación de los operadores a través de $p_\mathbf{f}$ y $p_\mathbf{g}$ se tiene que
\begin{equation*}
	\U^* = \mathcal{P}_\mathbf{f}, \quad \G^* = \mathcal{P}_\mathbf{g}
\end{equation*}

\begin{defn}[Feature map]
	Se definen $\Phi_\X : \X \to \H_X$, $\Phi_\Y : \Y \to \H_\Y$ los \textit{feature maps} de ambos \textit{kernels} como
	\begin{equation*}
		\Phi_\X (x) = k_\X (x, \cdot), \quad \Phi_\Y (y) = k_\Y (y, \cdot)
	\end{equation*}
\end{defn}

\begin{defn}
    Se definen, para $x_1, x_2 \in \X$, $y_1, y_2 \in \Y$ los operadores de rango 1 $C_{x_1,x_2}: \H_\X \to \H_\X$, $C_{y_1,y_2}: \H_\Y \to \H_\Y$ y $C_{y_1,x_1}: \H_\X \to \H_\Y$ respectivamente, como
     \begin{equation*}
    	C_{x_1, x_2} \psi = [\Phi_\X (x_1) \otimes \Phi_\X (x_2)] \psi = \langle \psi, \Phi_\X (x_2) \rangle \Phi_\X (x_1) = \psi (x_2) \Phi_\X (x_1)
    \end{equation*}
    \begin{equation*}
    	C_{y_1, y_2} \psi = [\Phi_\Y (y_1) \otimes \Phi_\Y (y_2)] \psi = \langle \psi, \Phi_\Y (y_2) \rangle \Phi_\Y (y_1) = \psi (y_2) \Phi_\Y (y_1)
    \end{equation*}
    \begin{equation*}
    	C_{y_1, x_1} \psi = [\Phi_\Y (y_1) \otimes \Phi_\X (x_1)] \psi = \langle \psi, \Phi_\X (x_1) \rangle \Phi_\Y (y_1) = \psi (x_1) \Phi_\Y (y_1)
    \end{equation*}
\end{defn}

\begin{defn}
    Se definen los operadores de covarianza $C_\X : \H_\X \to \H_\X$, $C_\Y : \H_\Y \to \H_\Y$ como
        \begin{equation*}
        C_\X \psi = \int_\X C_{x, x} \psi d \mu_\X (x) = \int_\X [\Phi_\X (x) \otimes \Phi_\X (x)] \psi d \mu_\X (x)
    \end{equation*}
     \begin{equation*}
    	C_\Y \psi = \int_\Y C_{y,y} \psi d \mu_\X (y) =  \int_\Y [\Phi_\Y (y) \otimes \Phi_\Y (y)] \psi d \mu_\Y (y)
    \end{equation*}
\end{defn}
\begin{defn}
    Se define el operador de covarianza cruzada asociada a la dinámica como el operador $C_{\X \X^+} : \H_\X \to \H_\X$ dado por 
    \begin{equation*}
        C_{\X \X^+} \psi = \int_\X \int_\X C_{x, y} \psi d\rho_\X (x, \cdot)(y) d \mu_\X (x)= \int_\X \int_\X [\Phi_\X (x) \otimes \Phi_\X (y)] d\rho_\X (x, \cdot)(y) d \mu_\X (x) 
    \end{equation*}
    Y el operador de covarianza cruzada asociada a la observación como $C_{\Y \X} : \H_\X \to \H_\Y$ dado por
    \begin{equation*}
        C_{\Y \X} \psi = \int_\X \int_\Y C_{y,x} \psi d\rho_\Y (x, \cdot) (y) d \mu_\X (x) = \int_\X \int_\Y [\Phi_\Y (y) \otimes \Phi_\X (x)] \psi d\rho_\Y (x, \cdot) (y) d \mu_\X (x)
    \end{equation*}
\end{defn}
\begin{defn}[Operadores de embedding condicional \cite{Song2009HilbertSystems}]   
	Se definen los operadores de embedding condicional como $C_{\X^+ | \X} : \H_\X \to \H_\X$, $C_{\Y | \X} : \H_\X \to \H_\Y$
	\begin{equation*}
		C_{\X^+ | \X} = C_{\X^+ \X} C_{\X}^{-1}
	\end{equation*}
	\begin{equation*}
		C_{\Y | \X} = C_{\Y \X} C_{\X}^{-1}
	\end{equation*}
\end{defn}
Asumiremos ahora que $\U \H_\X \subset \H_\X$ y $\G \H_\Y \subset \H_\X$, para ello primero asumiremos que $\H_\X$ y $H_\Y$ son espacios de Sobolev (pensar en Matérn) y ahora la siguiente proposición.
\begin{prop}[Invarianza Sobolev para Koopman]
	Si $p_\mathbf{f} \in C^{k,k} (\X \times \X)$ y $p_\mathbf{g} \in C^{k,k} (\X \times \Y)$, entonces 
	\begin{equation*}
		\U \H^k (\X) \subset \H^k (\X), \quad \G \H^k (\Y) \subset \H^k (\X)
	\end{equation*}
\end{prop}
\begin{proof}
	Basta tomar primero $m \in \N$, $m \leq k$ y $| \alpha | = m$ un multiíndice, luego, por teorema de la convergencia dominada, para $h \in \H^k (\X)$
	\begin{equation*}
		\partial^\alpha_x (\U h)(x) = \int_\X h(y) \partial^\alpha_x p_{\mathbf{f}} (x, y) d \mu_\X (y)
	\end{equation*}
	Así
	\begin{equation*}
		\begin{aligned}
			\| \partial^\alpha_x (\U h) \|_{L^2}^2 & \leq \int_\X \int_\X h(y)^2 \partial^\alpha_x p_{\mathbf{f}} (x, y)^2 d \mu_\X (y) d \mu_\X (x) \\
			& \leq \| \partial^\alpha_x p_{\mathbf{f}} \|_{C^{k,k}} \mu (\X) \int_\X h(y)^2 d \mu_\X (y) \\
			& \leq \| \partial^\alpha_x p_{\mathbf{f}} \|_{C^{k,k}} \mu_\X (\X) \| h \|_{H^k (\X)} 
		\end{aligned}
	\end{equation*}
	Con lo que 
	\begin{equation*}
		\| \U h \|_{H^k} \leq \left ( \mu_\X (\X) \sum_{|\alpha| \leq k} \|  \partial^\alpha_x p_{\mathbf{f}} \|_{C^{k,k}} \right ) \| h \|_{H^k} 
	\end{equation*}
	Y se concluye que $\U h \in H^k$, y de hecho
	\begin{equation*}
		\| \U \|_{H^k \to H^k} \leq  \mu_\X (\X) \sum_{|\alpha| \leq k} \|  \partial^\alpha_x p_{\mathbf{f}} \|_{C^{k,k}} 
	\end{equation*}
	Análogamente, para $h \in H^k(\Y)$
	\begin{equation*}
		\| \G h \|_{H^k (\X)} \leq \left ( \mu_\X (\X) \sum_{|\alpha| \leq k} \|  \partial^\alpha_x p_{\mathbf{g}} \|_{C^{k,k}} \right ) \| h \|_{H^k (\Y)} 
	\end{equation*}
\end{proof}
\noindent Suponiendo la invarianza de los operadores de Koopman a través del RKHS, se tiene lo siguiente
\begin{equation*}
	\begin{aligned}
		C_{\X \X} \psi & = \int_\X \int_\X [\Phi_\X (x) \otimes \Phi_\X (y)] \psi d\rho_\X (x, \cdot) (y) d \mu_\X (x)  \\
		& =  \int_\X \int_\Y \psi(y) \Phi_\X (x) d\rho_\X (x, \cdot) (y) d \mu_\X (x) \\
		& = \int_\X (\U \psi) (x) \Phi_\X (x) d \mu_\X (x) \\
		& = \int_\X [\Phi_\X (x) \otimes \Phi_\X (x) ] (\U \psi) d \mu_\X (x) \\
		& = C_\X  \U  \psi \\
	\end{aligned}
\end{equation*}
Por tanto se tiene que $C_{\X \X} = C_{\X} \U  $. Notando que
\begin{equation*}
    	C_{\X} = \E[\Phi_\X (\X) \otimes \Phi_\X (\X)], \quad	C_{\X \X^+} = \E[\Phi_\X (\X) \otimes \Phi_\X (\X^+)]
\end{equation*}
y que 
\begin{equation*}
	(\Phi_\X (\X) \otimes \Phi_\X (\Y))^* = \Phi_\X (\Y) \otimes \Phi_\X (\X)
\end{equation*}
se tiene $C_{\X}* = C_{\X}$ y $C_{\X \X^+}^* = C_{\X^+ \X}$
\begin{equation*}
	C_{\X^+ \X} C_\X^{-1} =  \U^* 
\end{equation*}
con lo que $C_{\X^+ \X} C_\X^{-1} = C_{\X^+ | \X} = \U^* = \mathcal{P}_\mathbf{f}$ y análogamente $C_{\Y | \X} = \G^* = \mathcal{P}_\mathbf{g}$. Esto debe entenderse de manera cuidadosa y está bien definido si es que $C_{\X \X}$ es inyectivo \cite{Fukumizu2013KernelKernels}.

	Ahora veremos cómo aproximar los tres operadores (Koopman de dinámica, observación y reconstrucción). Para ello sean $N$ puntos (este será nuestro parámetro de precisión de la aproximación), digamos $\{ x_i \}_{i=1}^N \sim \mu_\X^N$ y puntos $\{ x^+_i \}_{i=1}^N$, $\{ y_i \}_{i=1}^N$ sampleados tal que
	\begin{equation*}
		x^+_i \sim \rho_f (x_i, \cdot), \quad y_i \sim \rho_g (x_i, \cdot), \quad i = 1, \dots, N
	\end{equation*}
	Llamaremos al espacio
	\begin{equation*}
		\H_{\X, N} = \text{span} \{\Phi_\X (x_i) : i = 1, \dots, N \}
	\end{equation*}
	cuya base canónica es $\{\Phi_\X (x_i) : i = 1, \dots, N \}$.
	Llamamos a las matrices
	\begin{equation*}
		X = (x_{1} | \dots | x_N), \quad Y = (y_1 | \dots | y_N)
	\end{equation*}
	\begin{equation*}
		\Phi_N (X) = (k(x_i, x_j))_{i,j = 1}^N
	\end{equation*}
	\begin{equation*}
		\Phi_N (X^+) = (k(x_i, x^+_j))_{i,j = 1}^N
	\end{equation*}

	Se definen los operadores
	\begin{equation*}
		C_{X}^N : \H_{\X, N} \to \H_{\X, N}, \quad C_{XX^+}^N : \H_{\X, N} \to \H_{\X, N}
	\end{equation*}
	\begin{equation*}
		C_{XY}^N : \R^p \to \H_{\X, N}, \quad C_{XZ}^N : \R^n \to \H_{\X, N}
	\end{equation*}
	\begin{equation*}
		C_{X}^N \Phi_\X (x_i) = \frac{1}{N} \sum_{j = 1}^N [\Phi_\X (x_j) \otimes \Phi_\X (x_j)] \Phi_\X (x_i) = \frac{1}{N} \sum_{j = 1}^N k_\X(x_i, x_j) \Phi_\X (x_j)
	\end{equation*}
	\begin{equation*}
		C_{XX^+}^N \Phi_\X (x_i) = \frac{1}{N} \sum_{j = 1}^N [\Phi_\X (x_j) \otimes \Phi_\X (x_j)] \Phi_\X (x_i^+) = \frac{1}{N} \sum_{j = 1}^N k_\X(x_i^+, x_j) \Phi_\X (x_j)
	\end{equation*}
	\begin{equation*}
		C_{XY}^N \Phi_\Y (y_i) = \frac{1}{N} \sum_{j = 1}^N [\Phi_\X (x_j) \otimes \Phi_\Y (y_j)] \Phi_\Y (y_i) = \frac{1}{N} \sum_{j = 1}^N k_\Y (y_i, y_j) \Phi_\X (x_j)
	\end{equation*}
	\begin{equation*}
		C_{XZ}^N \phi (x_i) = \frac{1}{N} \sum_{j = 1}^N [\Phi_\X (x_j) \otimes \phi (x_j)] \phi (x_i) = \frac{1}{N} \sum_{j = 1}^N \langle x_i, x_j \rangle \Phi_\X (x_j)
	\end{equation*}
	Que vienen representados por $\Phi_N (X)$, $\Phi_N (X^+)$, $Y$, $X$, respectivamente, con respecto a su base canónica.

	Entonces se definen los operadores 
	\begin{equation*}
		\U_N : \H_{\X, N} \to \H_{\X, N}, \quad \G_N : \R^p \to \H_{\X, N}, \quad \B_N : \R^n \to \H_{\X, N}
	\end{equation*}
	que vienen representación por las matrices
	\begin{equation*}
		U_N = (\Phi_N (X))^{-1} \Phi_N (X^+)^\top
	\end{equation*}
	\begin{equation*}
		G_N = (\Phi_N (X))^{-1} Y^\top
	\end{equation*}
	\begin{equation*}
		B_N = (\Phi_N (X))^{-1} X^\top
	\end{equation*}
	Definiremos además
	\begin{equation*}
		\| k_\X \|_1 = \int_\X k_\X (x,x) d \mu_\X (x), \quad \| k_\X \|_\infty = \sup_{x \in \X} k_\X (x,x)
	\end{equation*}
	que para el caso de Matérn son ambas finitas.

	Una vez definido todo lo anterior, utilizaremos el resultado de Philipp et al. \cite{Philipp2024ErrorOperator}.
	\begin{teo}
		Sea un \( r \in \mathbb{N} \) arbitrario, se supone que los primeros \( r + 1 \) valores propios \( \lambda_j \) de \( C_X \) son simples, es decir, \( \lambda_{j+1} < \lambda_j \) para todo \( j = 1, \ldots, r \). Se define
		
		\[
		\delta_r = \min_{j=1, \ldots, r} \frac{\lambda_j - \lambda_{j+1}}{2}, \quad c_r = \frac{1}{\sqrt{\lambda_r}} + \frac{r + 1}{\delta_r \lambda_r} (1 + \|k_\X\|_{1}) \|k_\X \|^{1/2}_{1}
		\]
		Además, sea \( \varepsilon \in (0, \delta_r) \) y \( \delta \in (0, 1) \) arbitrarios, y \( N \geq \max\{r, \frac{8\|k\|^2_\infty \ln(4/\delta)}{\varepsilon^2}\} \). Si $\U \H_\X \subset \H_\X$, entonces, con probabilidad al menos \( 1 - \delta \), se cumple que
		
		\[
		\|\U - \U_N \|_{\H_{\X, N} \to L^2(\X; \mu_\X)} \leq \sqrt{\lambda_{r+1}} \|\U \|_{\H_{X} \to \mathcal{H_\X}} + c_r \varepsilon
		\]
		\label{teo:error_koop}
	\end{teo}
	La misma cota es válida para $\G_N$ y $\B_N$. El siguiente teorema culmina esta sección y será la síntesis de la cota necesaria para la sección siguiente.
\begin{teo}
    Si $\H_\X$ es equivalente en norma a $H^{\nu + n/2}$, entonces 
    \begin{equation}
        \| \U - \U_N \|_{\H_{\X, N} \to L^2(\X; \mu_\X)} \leq C N^{-1/2}  
        \label{eq:kEDMD_bound}
    \end{equation}
	\label{teo:error_koop_sqrt_N}
    \end{teo}
Antes de su demostración, un lema para el orden de los valores propios de la matriz de covarianza.
	\begin{teo}[Santin et al. \cite{Santin2016ApproximationSpaces}]
		Si $\H_\X$ es equivalente en norma a un espacio de Sobolev $H^{\nu + n/2}$ y $\lambda_j$ son los valores propios de $C_X$ en orden descendente, luego
	\begin{equation*}
		\sqrt{\lambda_{N+1}} \leq c_1 N^{-(\nu + n)/(2n)}
	\end{equation*}
	con $c_1$ alguna constante que no depende de $N$.
	\label{lema:eig_val_decay}
	\end{teo}
	\begin{proof}
		Gracias a la arbitrariedad de $\varepsilon$ en el teorema \ref{teo:error_koop} se tiene que
		\[
		\|\U - \U_N \|_{\H_{\X, N} \to L^2(\X; \mu_\X)} \leq c\sqrt{\lambda_{r+1}} 
		\]
		y por el lema \ref{lema:eig_val_decay} se tiene que existe una constante $C$ tal que
		\begin{equation*}
			\|\U - \U_N \|_{\H_{\X, N} \to L^2(\X; \mu_\X)} \leq c_1 N^{-(\nu + n)/(2n)}
		\end{equation*}
		como $\nu > 0$ se concluye que
		\begin{equation*}
			\|\U - \U_N \|_{\H_{\X, N} \to L^2(\X; \mu_\X)} \leq c_1 N^{-1/2}\\
		\end{equation*}
	\end{proof}
	En principio $N^{-1/2}$ es una cota menos ajustada que la mostrada en \cite{Philipp2024ErrorOperator}, pero esta corresponde al orden de error de otras aproximaciones que aparecerán en la sección siguiente.


\subsection{Resultados numéricos}

A pesar de que las cotas de error del kernel Extended Dynamic Mode Decomposition serán útiles para la construcción de la cota de error para el filtro, también son interesantes por sí solas y se puede visualizar al aproximar distintos sistemas dinámicos y operadores de manera empírica.\\
Será de interés en esta sección no solo explorar cómo un sistema lineal puede aproximar otro sistema de interés mediante el \textit{sampleo} de datos.

\subsubsection{kEDMD para el caso lineal}

Un primer caso de interés es estudiar qué ocurre cuando se tiene un sistema lineal, esto es considerar un sistema de la forma
\begin{equation*}
    \mathbf{x}_{k+1} = \mathbf{A} \mathbf{x}_{k}.
\end{equation*}
Para este ejemplo se considera
\begin{equation*}
    \mathbf{A} = 
    \begin{pmatrix}
        1.01 & 0.04 & 0 \\
        0.01 & 1.02 & \alpha \\
        0 & 0.04 & 1.02
    \end{pmatrix}
\end{equation*}
en donde el parámetro $\alpha$ determinará el comportamiento del sistema dinámico en el tiempo. Se considera como distribución de probabilidad en $\X = \R^n$ una normal $N(0_{3}, 3 \cdot I_{3\times3})$, para la dinámica se considera un ruido normal $N(0_{3}, 10^{-7} \cdot I_{3\times3})$ y $\mathbf{x}_0 = (0.1, 0.1, 0.1)^T$ como condición inicial.\\
Notar que en este ejemplo no se satisface la hipótesis de compacidad, pero el sistema con alta probabilidad se mantiene en un compacto en horizonte de tiempo finito.\\
Se compara la dinámica original con aquella linealizada vía Operador de Koopman \textit{sampleando} 2500 puntos del espacio de estados, utilizando \textit{kernel} de Matérn de parámetro $\nu = 1/2$ y un ancho de banda $\gamma = 10^{-3}$. Esto es, se compara con el sistema
\begin{align*}
    \mathbf{z}_{k+1} = & \mathbf{U}_N \mathbf{z}_k \\
    \hat{\mathbf{x}}_k = & \mathbf{B}_N \mathbf{z}_k \\
    \mathbf{z}_0 = & \Phi_N(\mathbf{x}_0)
\end{align*}

Se compara lo obtenido para $\alpha \in \{ -0.3, -0.1, 0.05\}$.
\begin{figure}[htbp]
    \centering
    \begin{subfigure}[b]{0.32\textwidth}
        \centering
        \includegraphics[width=\textwidth]{img/content/chapter3/Linear1.pdf}
        \caption{$\alpha=-0.3$}
        \label{fig:image1}
    \end{subfigure}
    \hfill
    \begin{subfigure}[b]{0.32\textwidth}
        \centering
        \includegraphics[width=\textwidth]{img/content/chapter3/Linear2.pdf}
        \caption{$\alpha=-0.1$}
        \label{fig:image2}
    \end{subfigure}
    \hfill
    \begin{subfigure}[b]{0.32\textwidth}
        \centering
        \includegraphics[width=\textwidth]{img/content/chapter3/Linear3.pdf}
        \caption{$\alpha=0.05$}
    \end{subfigure}
    \caption{Ilustración de los tres casos de $\alpha$ elegidos para la comparación entre el sistema lineal original y el sistema linealizado por Koopman a 2500 puntos \textit{sampleados} de una variable aleatoria normal. En forma de puntos se dejan los valores reales que toma el sistema y en línea continua los valores entregados por el sistema linealizado, que se consideran como predicción.}
    \label{fig:Comp_traj_lin}
\end{figure}
Aunque la cota en \eqref{eq:kEDMD_bound} no hace referencia directa a la diferencia en norma de las trayectorias, es de interés analizar el orden en que decae el error en función de $N$, para ello se calcula la diferencia en norma de las trayectorias generadas para $N \in \{ 100k : k \in \{1, \dots, 30\} \}$. \\
En la figura \ref{fig:Comp_traj_lin} se observa que las trayectorias obtenidas con el sistema linealizado vía Koopman son muy cercanas a las obtenidas con el sistema lineal original, solo difiriendo en zonas donde el sistema original toma valores muy altos, lo que se puede asociar a la poca probabilidad de \textit{sampleo} que le da la distribución normal elegida a puntos muy alejados del origen. \\
Mientras que en la figura \ref{fig:ErrorLin} se observa que en realidad el orden de decaimiento del error en función de $N$ es mayor que $N^{-1/2}$, mejorando la cota mostrada en \eqref{eq:kEDMD_bound}.
\begin{figure}[h]
    \centering
    \begin{subfigure}[b]{0.32\textwidth}
        \centering
        \includegraphics[width=\textwidth]{img/content/chapter3/Linear1Errors.pdf}
        \caption{$\alpha=-0.3$}
        \label{fig:image1}
    \end{subfigure}
    \hfill
    \begin{subfigure}[b]{0.32\textwidth}
        \centering
        \includegraphics[width=\textwidth]{img/content/chapter3/Linear2Errors.pdf}
        \caption{$\alpha=-0.1$}
        \label{fig:image2}
    \end{subfigure}
    \hfill
    \begin{subfigure}[b]{0.32\textwidth}
        \centering
        \includegraphics[width=\textwidth]{img/content/chapter3/Linear3Errors.pdf}
        \caption{$\alpha=0.05$}
        \label{fig:image3}
    \end{subfigure}
    \caption{Ilustración de los tres casos de $\alpha$ elegidos para la evolución en función de $N$ de la diferencia en norma entre el sistema lineal original y el sistema linealizado por Koopman a $N$ puntos \textit{sampleados} de una variable aleatoria normal. En forma de puntos se deja la evolución observada del error y en línea continua la mejor curva de la forma $C \cdot N^{a}$, donde $a$ es el exponente que se deja en la leyenda.}
    \label{fig:ErrorLin}
\end{figure}
\subsubsection{kEDMD para modelos utilizados en epidemiología}
Para esta sección se ilustrarán los resultados de kEDMD para modelos epidemiológicos que no consideran nacimientos ni muertes, es decir, su población total se mantiene constante en el tiempo. Estos modelos reciben el nombre de compartimentales, ya que cada parte de la población se encuentra en un único compartimento y toda la población está en alguno de ellos. Para estos casos se supondrá que la población está normalizada, esto es, que cada compartimento está en $[0, 1]$.\\
Esto último permite decir que el espacio de estados de este modelo es, sin considerar los eventuales factores estocásticos, un $(n-1)$-simplex, siendo $n$ la cantidad de compartimentos considerados, que se define como el conjunto
\begin{equation*}
    \Delta_{n-1} = \left \{ x \in \R^n : \sum_{i=1}^n x_i = 1, \, x_i \geq 0, \, \forall i \in \{1, \dots, n\} \right \}.
\end{equation*}
De este conjunto se puede \textit{samplear} de manera eficiente desde una distribución Dirichlet \cite{Frigyik2010IntroductionProcesses}, y se encuentra implementado en las principales librerías con funcionalidades estadísticas como SciPy \cite{Virtanen2020SciPyPython}, que será la utilizada en este trabajo.\\
La densidad de una variable aleatoria Dirichlet con parámetros $\alpha_1, \alpha_2, \ldots, \alpha_K$ se define como:
\[
f(x_1, x_2, \ldots, x_K; \alpha_1, \alpha_2, \ldots, \alpha_K) = \frac{1}{B(\alpha)} \prod_{i=1}^{K} x_i^{\alpha_i - 1}
\]
donde $x_i \geq 0$ para todo $i$, $\sum_{i=1}^{K} x_i = 1$, y
\[
B(\alpha) = \frac{\prod_{i=1}^{K} \Gamma(\alpha_i)}{\Gamma\left(\sum_{i=1}^{K} \alpha_i\right)}
\]
es la función beta multivariable, y $\Gamma(\cdot)$ es la función gamma. En la figura \ref{fig:Dirichlet_samples} se puede apreciar la diferencia de \textit{samplear} para diferentes valores de $\alpha$. Un valor de $\alpha$ con entradas iguales genera la misma dispersión en todas las direcciones, siendo $(1, 1, 1)$ la variable aleatoria uniforme en $\Delta_{n-1}$, mientras que valores altos de $\alpha$ generan una alta concentración de muestras en el centro del conjunto. Por otro lado valores desiguales generar mayor cantidad de muestras en alguna de las caras del conjunto.
\begin{figure}[htbp]
    \centering
    \includegraphics[width=0.95\linewidth]{img/content/chapter3/Dirichlet.pdf}
    \caption{1000 muestras de una variable aleatoria Dirichlet para diferentes valores de $\alpha$, observando dos casos de $\alpha$ con valores homogéneos y uno con valores heterogéneos, lo que provoca un desbalance de muestras.}
    \label{fig:Dirichlet_samples}
\end{figure}
El primer ejemplo clásico es el modelo SIR, en donde se supone que se tiene una dinámica de contagios tal que los susceptibles y los infectados interactúan y a cierta tasa se generan más contagios; luego, a cierta tasa, estos infectados se recuperan, ganan inmunidad y por tanto son removidos de la dinámica de contagio, por lo que no vuelven a ser susceptibles.\\
En tiempo discreto, la dinámica de este modelo viene dada por las ecuaciones
\begin{equation}
    \begin{aligned}
    S_{k+1} &= S_k -\beta S_k I_k, \\
    I_{k+1} &= I_k + \beta S_k I_k - \gamma I_k, \\
    S_{k+1} &= R_k + \gamma I_k,
    \end{aligned}
    \label{eq:SIR}
\end{equation}
donde:
\begin{itemize}
    \item \(S_k\) es el número de individuos susceptibles en el tiempo k,
    \item \(I_k\) es el número de individuos infecciosos en el tiempo \(k\),
    \item \(R_k\) es el número de individuos recuperados en el tiempo \(k\),
    \item \(\beta\) es la tasa de transmisión de la enfermedad,
    \item \(\gamma\) es la tasa de recuperación.
\end{itemize}
Se visualiza lo obtenido por kEDMD, utilizando \textit{kernel} de Matérn de parámetro $\nu=1/2$ y ancho de banda $\gamma=10^{-3}$. Se utiliza una distribución Dirichlet de parámetro $(1,1,1)$ como variable aleatoria asociada al espacio de estados y ruido aditivo centrado y Gaussiano de matriz de covarianza $10^{-7} I_{3 \times 3}$. Los resultados obtenidos se pueden ver en la imagen \ref{fig:SIR_kEDMD}. \\
\begin{figure}[h]
    \centering
    \includegraphics[width=\linewidth]{img/content/chapter3/SIR.pdf}
    \caption{Comparación de kEDMD para modelo SIR, \textit{sampleando} 1000 puntos desde una distribución de Dirichlet. En forma de puntos se puede observar la trayectoria real y en línea continua la trayectoria generada por kEDMD.}
    \label{fig:SIR_kEDMD}
\end{figure}
Similar a lo hecho en el caso lineal, se compara la diferencia en norma entre la trayectoria original y la trayectoria linealizada, con la mejor curva de la forma $C \cdot N^{a}$, para observar si los errores andan en el orden de decaimiento en \eqref{eq:kEDMD_bound}.
\begin{figure}[h]
    \centering
    \begin{subfigure}[b]{0.32\textwidth}
        \centering
        \includegraphics[width=\textwidth]{img/content/chapter3/SIR1Errors.pdf}
        \caption{$\alpha=-0.3$}
    \end{subfigure}
    \hfill
    \begin{subfigure}[b]{0.32\textwidth}
        \centering
        \includegraphics[width=\textwidth]{img/content/chapter3/SIR2Errors.pdf}
        \caption{$\alpha=-0.1$}
    \end{subfigure}
    \hfill
    \begin{subfigure}[b]{0.32\textwidth}
        \centering
        \includegraphics[width=\textwidth]{img/content/chapter3/Linear3Errors.pdf}
        \caption{$\alpha=0.05$}
    \end{subfigure}
    \caption{Ilustración de los tres casos del modelo SIR elegido para la evolución en función de $N$ de la diferencia en norma entre el sistema lineal original y el sistema linealizado por Koopman a $N$ puntos,  \textit{sampleados} de una variable aleatoria normal. En forma de puntos se deja la evolución observada del error y en línea continua la mejor curva de la forma $C \cdot N^{a}$, donde $a$ es el exponente que se deja en la leyenda.}
    \label{fig:ErrorSIR}
\end{figure}
Se observa nuevamente que la discrepancia entre las trayectorias en efecto anda del orden de la cota expuesta en este capítulo, incluso con decaimientos aún más rápidos.\\
Con ello, de manera sencilla puede incluirse un factor de pérdida de inmunidad, lo que viene dado por las ecuaciones
\begin{equation}
    \begin{aligned}
    S_{k+1} &= S_k -\beta S_k I_k + \alpha R_k, \\
    I_{k+1} &= I_k + \beta S_k I_k - \gamma I_k, \\
    S_{k+1} &= R_k + \gamma I_k - \alpha R_k,
    \end{aligned}
    \label{eq:SIR_rec}
\end{equation}
En donde el nuevo parámetro $\alpha$ es la tasa de pérdida de inmunidad después de haberse recuperado de la enfermedad.

% Capítulo 4: Algoritmo de filtraje no lineal en tiempo discreto y estimación de parámetros
\chapter{Algoritmo de filtraje no lineal en tiempo discreto y estimación de parámetros}
El hecho de que sea posible construir un sistema lineal que se asemeje significativamente a otro sistema, eventualmente no lineal, sugiere la viabilidad de emplear dicho sistema linealizado para abordar problemas de filtraje no lineal mediante el uso del Filtro de Kalman. 

En la literatura, ya existen diversas conexiones entre el Filtro de Kalman y el operador de Koopman. Por ejemplo, se ha explorado su utilización para la corrección de errores \cite{Jiang2022CorrectingFilters}, la estimación de modos de Koopman \cite{Liu2024EstimateFilter}, y la mejora de algoritmos como el Filtro de Kalman Extendido \cite{Ramadan2024ExtendedControl}. Además, algoritmos similares al que se propone en este capítulo han sido presentados y validados en diferentes aplicaciones, como se observa en \cite{Wang2022KoopmanSystem, Wang2023Innovation-saturatedOutliers, Netto2018RobustEstimation, Syed2021Koopman-basedXFEL, HuangData-DrivenFlight, Yang2025Data-DrivenPredictor}. 

Por otra parte, trabajos pioneros en la línea de observabilidad e identificación de sistemas, liderados por Surana y colaboradores, han proporcionado una base sólida en este campo, como se documenta en \cite{Surana2016KoopmanSystems, Surana2016LinearFramework}.

Por otro lado, hay enfoques que aplican la \textit{kernel Bayes Rule} a sistemas dinámicos y filtraje, donde se incluyen los trabajos de Song, Fukumizu y colaboradores \cite{Fukumizu2004DimensionalitySpaces, Fukumizu2013KernelKernels, Fukumizu2015NonparametricEmbedding, Song2009HilbertSystems, Song2013KernelModels}, lo que aparece de manera natural al demostrar la utilidad de los operadores para representar esperanzas condicionales.

Aunque el algoritmo propuesto en este capítulo no constituye una contribución original en términos de su estructura, sí lo es la justificación de su funcionamiento y convergencia bajo el supuesto de contar con un muestreo suficiente de puntos. En este contexto, se demuestra, en base a las cotas probadas anteriormente para la aproximación vía \textit{kernel} Dynamic Mode Decomposition, que un filtro subóptimo representado en dimensión finita converge al óptimo en dimensión infinita. 

El objetivo principal de este capítulo es demostrar que, bajo ciertas hipótesis, la tasa de convergencia del algoritmo es del orden de $N^{-1/2}$, lo que lo posiciona como un competidor viable frente a otros algoritmos de filtraje presentes en la literatura, tales como los filtros de partículas, discutidos previamente en la sección de preliminares.

Para ello, se propone primero descomponer el error del Filtro de Kalman aplicado a dos sistemas lineales, formulando este error en función de los datos provenientes de los sistemas. Este análisis, hasta la fecha de redacción de este trabajo, no se encuentra documentado en la literatura y, por lo tanto, constituye una contribución original de esta investigación.

Posteriormente, se construye el filtro propuesto, denominado \textit{Koopman Kalman Filter} (KKF), siguiendo una metodología análoga a la empleada en el Filtro de Kalman para sistemas lineales con mínimos cuadrados recursivos \cite{Kalman1960AProblems, Triantafyllopoulos2021BayesianBeyond}, y particularmente a como se formula en \cite{Gebhard2019} para la construcción de la denominada \textit{kernel Kalman Rule}.

Finalmente, utilizando los resultados obtenidos en el capítulo anterior junto con la descomposición del error, se demuestra la cota de error propuesta. El desempeño del filtro es evaluado en diferentes sistemas mediante una implementación en Python desarrollada específicamente para este trabajo.

Se muestran resultados numéricos del filtro en modelos epidemiológicos, tanto en problemas de filtraje con observaciones parciales como problemas de estimación de parámetros de estos modelos. El trabajo de Catalán Muñoz \cite{CatalanMunoz2022DesarrolloChile} fue una inspiración para este trabajo al aplicar Filtro de Kalman para estimación de parámetros de un modelo epidemiológico relevante para la realidad de Santiago de Chile. Cabe notar que Mezić et al. ya han aplicado técnicas relacionadas con operador de Koopman para modelos epidemiológicos \cite{Mezic2024ACases}, aunque nunca se habla de filtraje, dinámicas específicas ni estimación de parámetros. 

\section{Descomposición de error de Kalman}

El objetivo de esta sección es analizar el error que se genera entre dos reglas de Kalman, lo cual permitirá cuantificar la discrepancia entre una regla de Kalman aproximante y otra exacta, ambas definidas formalmente más adelante en esta misma sección. Para ello se asumirá que el horizonte de tiempo $T$ es finito, lo que evita tener que preocuparse por el comportamiento de las constantes para tiempos largos.

Para este propósito, se consideran dos sistemas dinámicos observados en un espacio de Hilbert con espacios de estados $E_x$ y de observaciones $E_y$, descritos por las siguientes ecuaciones:  
\begin{equation*}
	\begin{aligned}
		\mu_{i,k}  &= A_{i,k} \mu_{i,k-1} + \nu_{i,k}, \\
		y_{i,k} &= C_{i,k} \mu_{i,k} + \xi_{i,k},
	\end{aligned}
\end{equation*}
donde $A_{i,k} : E_x \to E_x$ y $C_{i,k}: E_x \to E_y$ son operadores lineales; $\nu_{i,k} \in E_x$ y $\xi_{i,k} \in E_y$ representan variables aleatorias con segundo momento finito y operadores de covarianza $\mathcal{Q}_{i,k}$ y $\mathcal{R}_{i,k}$, respectivamente. Todo esto se considera para $i \in \{1,2\}$ y $k \geq 1$.

Cada uno de estos sistemas tiene asociada una regla de Kalman, la cual está definida por las siguientes expresiones:  
\begin{equation*}
	\begin{aligned}
		\mathcal{P}_{i,k}^- &= A_{i,k}^* \mathcal{P}_{i,k-1}^+ A_{i,k} + \mathcal{Q}_{i,k}, \\
		\S_{i,k} &= C_{i,k} \mathcal{P}_{i,k}^- C_{i,k}^* + \mathcal{R}_{i,k}, \\
		\K_{i,k} &= \mathcal{P}_{i,k}^- C_{i,k} \S_{i,k}^{-1}, \\
		\mathcal{P}_{i,k}^+ &= (I - \K_{i,k} C_{i,k}) \mathcal{P}_{i,k}^-, \\
		\hat{\mu}_{i,k} &= A_{i,k} \hat{\mu}_{i,k-1} + \K_{i,k} (y_{i,k} - C_{i,k} \hat{\mu}_{i,k-1}),
	\end{aligned}
\end{equation*}
con $i \in \{1,2\}$ y $k \geq 1$. Aquí, $\mathcal{P}_{i,k}^-$ y $\mathcal{P}_{i,k}^+$ representan los operadores de covarianza del error a priori y a posteriori, respectivamente, mientras que $\K_{i,k}$ es el operador de ganancia de Kalman, todos ellos definidos en los espacios indicados. Estas reglas se inicializan como sigue:
\begin{equation*}
	\hat{\mu}_{i,0} = \mathbb{E}[\mu_{i,0}], \quad \mathcal{P}_{i,0} = \text{Cov}(\mu_{i,0}).
\end{equation*}

Con estas definiciones, se presenta un resultado clave que ilustra cómo la discrepancia en norma entre las reglas de Kalman puede descomponerse en función de las discrepancias en norma de los elementos asociados, junto con la influencia de las iteraciones previas.


\begin{teo}[Descomposición de error de Kalman]
	Sea $k \geq 1$. Si los operadores $\S_{i,k}$ son invertibles, entonces existen constantes $c_{k,j}^i$ con $j \in \{1, \dots, 7\}$, $i \in \{1, 2\}$, tales que se cumplen las siguientes desigualdades:
	\begin{equation*}
		\begin{aligned}
			\| \hat{\mu}_{1,k} - \hat{\mu}_{2,k} \| \leq & \, c_{1,k}^1 \| A_{1,k} - A_{2,k} \| + c_{2,k}^1 \| C_{1,k} - C_{2,k} \| \\ 
			&+ c_{3,k}^1 \| \mathcal{Q}_{1,k} - \mathcal{Q}_{2,k} \| + c_{4,k}^1 \| \mathcal{R}_{1,k} - \mathcal{R}_{2,k} \| \\
			&+ c_{5,k}^1 \| y_{1,k} - y_{2,k} \| + c_{6,k}^1 \| \hat{\mu}_{1,k-1} - \hat{\mu}_{2,k-1} \| \\
			&+ c_{7,k}^1 \| \mathcal{P}_{1,k-1}^+ - \mathcal{P}_{2,k-1}^+ \|,
		\end{aligned}
	\end{equation*}
	y
	\begin{equation*}
		\begin{aligned}
			\| \mathcal{P}_{1,k}^+ - \mathcal{P}_{2,k}^+ \| \leq & \, c_{1,k}^2 \| A_{1,k} - A_{2,k} \| + c_{2,k}^2 \| C_{1,k} - C_{2,k} \| \\ 
			&+ c_{3,k}^2 \| \mathcal{Q}_{1,k} - \mathcal{Q}_{2,k} \| + c_{4,k}^2 \| \mathcal{R}_{1,k} - \mathcal{R}_{2,k} \| \\
			&+ c_{5,k}^2 \| y_{1,k} - y_{2,k} \| + c_{6,k}^2 \| \hat{\mu}_{1,k-1} - \hat{\mu}_{2,k-1} \| \\
			&+ c_{7,k}^2 \| \mathcal{P}_{1,k-1}^+ - \mathcal{P}_{2,k-1}^+ \|.
		\end{aligned}
	\end{equation*}

	Aquí, las constantes $c_{k,j}^i$ son positivas y dependen de $k$ únicamente a través de las normas $\| A_{i,k} \|$, $\| C_{i,k} \|$, $\| \mathcal{Q}_{i,k} \|$, $\| \mathcal{R}_{i,k} \|$, $\| \S_{i,k}^{-1} \|$, $\| y_{i,k} \|$, $\| \hat{\mu}_{i,k-1} \|$ y $\| \mathcal{P}_{i,k-1}^+ \|$.
	\label{teo:error_kalman}
\end{teo}

\begin{proof}
Se observa que  
\begin{equation*}
	\begin{aligned}
		&	\| \hat \mu_{1,k} - \hat \mu_{2,k} \|_{E_x}  \\
		\leq & \, \| A_{1,k} \mu_{1,k-1}  - A_{2,k} \mu_{2,k-1} \|  \\
		& + \|  \K_{1,k} (y_{1,k} - C_{1,k} \hat\mu_{1,k-1}) -  \K_{2,k} (y_{2,k} - C_{2,k} \hat\mu_{2,k-1})  \|.
	\end{aligned}
\end{equation*}
El primer término, denominado \textit{error de predicción}, satisface la siguiente desigualdad:
\begin{equation*}
	\begin{aligned}
		& \| A_{1,k} \mu_{1,k-1}  - A_{2,k} \mu_{2,k-1} \|  \\
		& \leq \| A_{1,k} \mu_{1,k-1}  - A_{1,k} \mu_{2,k-1} \| + \| A_{1,k} \mu_{2,k-1}  - A_{2,k} \mu_{2,k-1} \| \\
		& \leq \| A_{1,k} \| \| \mu_{1,k-1}  - \mu_{2,k-1} \| +  \| \mu_{2,k-1} \| \| A_{1,k} - A_{2,k} \|.
	\end{aligned}
\end{equation*}
Por otro lado, el segundo término, denominado \textit{error de actualización}, cumple:
\begingroup
\allowdisplaybreaks
\begin{align*}
    & \|  \K_{1,k} (y_{1,k} - C_{1,k} \hat\mu_{1,k-1}) -  \K_{2,k} (y_{2,k} - C_{2,k} \hat\mu_{2,k-1})  \| \\
		& \leq  \| \K_{1,k} y_{1,k} -  \K_{2,k} y_{2,k}  \| + \| \K_{1,k} C_{1,k} \hat\mu_{1,k-1} - \K_{2,k} C_{2,k} \hat\mu_{2,k-1}  \| \\
		& \leq \| \K_{1,k} y_{1,k} -  \K_{1,k} y_{2,k}  \| + \| \K_{1,k} y_{2,k} -  \K_{2,k} y_{2,k}  \| \\
		& \quad + \| \K_{1,k} C_{1,k} \hat\mu_{1,k-1} - \K_{1,k} C_{2,k} \hat\mu_{2,k-1}  \| + \| \K_{1,k} C_{2,k} \hat\mu_{2,k-1} - \K_{2,k} C_{2,k} \hat\mu_{2,k-1}  \| \\
		& \leq \| \K_{1,k} \| \|  y_{1,k} - y_{2,k}  \| + \| y_{2,k} \| \| \K_{1,k}  -  \K_{2,k}  \| \\
		& \quad + \| \K_{1,k} \| \|  C_{1,k} \hat\mu_{1,k-1} - C_{2,k} \hat\mu_{2,k-1}  \| + \| C_{2,k} \hat\mu_{2,k-1} \| \| \K_{1,k}  - \K_{2,k} \| \\
		& \leq \| \K_{1,k} \| \|  y_{1,k} - y_{2,k}  \| + \| y_{2,k} \| \| \K_{1,k}  -  \K_{2,k}  \| \\
		& \quad + \| \K_{1,k} \| \left ( \|  C_{1,k} \hat\mu_{1,k-1} - C_{1,k} \hat\mu_{2,k-1}  \| + \|  C_{1,k} \hat\mu_{2,k-1} - C_{2,k} \hat\mu_{2,k-1}  \| \right ) \\
		& \quad + \| C_{2,k} \hat\mu_{2,k-1} \| \| \K_{1,k}  - \K_{2,k} \| \\
		& \leq \| \K_{1,k} \| \|  y_{1,k} - y_{2,k}  \| + \| y_{2,k} \| \| \K_{1,k}  -  \K_{2,k}  \| \\
		& \quad + \| \K_{1,k} \| \left ( \| C_{1,k}  \| \|  \hat\mu_{1,k-1} - \hat\mu_{2,k-1}  \| + \| \hat\mu_{2,k-1}  \| \| C_{1,k} - C_{2,k}  \| \right ) \\
		& \quad + \| C_{2,k} \hat\mu_{2,k-1} \| \| \K_{1,k}  - \K_{2,k} \|.
\end{align*}
\endgroup

En virtud de lo anterior, se debe analizar la diferencia en norma de los operadores de ganancia:
\begin{equation*}
	\begin{aligned}
		& \| \K_{1,k}  - \K_{2,k} \| \\
		& \leq \| \mathcal{P}_{1,k}^- C_{1,k}\S_{1,k}^{-1} -  \mathcal{P}_{2,k}^- C_{2,k} \S_{2,k}^{-1} \| \\
		& \leq \| \mathcal{P}_{1,k}^- C_{1,k}\S_{1,k}^{-1} - \mathcal{P}_{2,k}^- C_{1,k}\S_{1,k}^{-1} \| + \| \mathcal{P}_{2,k}^- C_{1,k}\S_{1,k}^{-1} - \mathcal{P}_{2,k}^- C_{2,k}\S_{2,k}^{-1} \| \\
		& \leq \| C_{1,k}\S_{1,k}^{-1} \| \| \mathcal{P}_{1,k}^- - \mathcal{P}_{2,k}^-\| + \| \mathcal{P}_{2,k}^- \| \|  C_{1,k}\S_{1,k}^{-1} -  C_{2,k}\S_{2,k}^{-1}\| \\
		& \leq \| C_{1,k}\S_{1,k}^{-1} \| \| \mathcal{P}_{1,k}^- - \mathcal{P}_{2,k}^-\| \\
		& \quad + \| \mathcal{P}_{2,k}^- \| ( \|  C_{1,k}\S_{1,k}^{-1} -  C_{1,k}\S_{2,k}^{-1}\| + \|  C_{1,k}\S_{2,k}^{-1} -  C_{2,k}\S_{2,k}^{-1}\|) \\
		& \quad + \| \mathcal{P}_{2,k}^- \| ( \|  C_{1,k} \| \| \S_{1,k}^{-1} -  \S_{2,k}^{-1}\| + \| \S_{2,k}^{-1} \| \| C_{1,k} -  C_{2,k}\|).
	\end{aligned}
\end{equation*}
En donde
\begin{equation*}
	\| \mathcal{P}_{i,k}^- \|  \leq \| A_{i,k} \|^2 \| \mathcal{P}_{i,k-1}^+\| + \| \mathcal{Q}_{i,k}\|, \quad i \in \{ 1, 2 \}.
\end{equation*}

Primero, para las diferencias en norma de los operadores de covarianza de error a priori se tiene:
\begin{equation*}
	\begin{aligned}
		& \| \mathcal{P}_{1,k}^- - \mathcal{P}_{2,k}^-\| \\
		& = \| A_{1,k}^* \mathcal{P}_{1,k-1}^+ A_{1,k} + \mathcal{Q}_{1,k} - A_{2,k}^* \mathcal{P}_{2,k-1}^+ A_{2,k} + \mathcal{Q}_{2,k}  \| \\
		& \leq \|A_{1,k}^* \mathcal{P}_{1,k-1}^+ A_{1,k} - A_{2,k}^* \mathcal{P}_{2,k}^+ A_{2,k} \| + \| \mathcal{Q}_{1,k} - \mathcal{Q}_{2,k}  \| \\
		& \leq \|A_{1,k}^* \mathcal{P}_{1,k-1}^+ A_{1,k} - A_{1,k}^* \mathcal{P}_{2,k-1}^+ A_{2,k} \| +  \|A_{1,k}^* \mathcal{P}_{2,k-1}^+ A_{2,k} - A_{2,k}^* \mathcal{P}_{2,k-1}^+ A_{2,k} \| \\ 
		& \quad + \| \mathcal{Q}_{1,k} - \mathcal{Q}_{2,k}  \| \\
		& \leq \|A_{1,k}^* \| \| \mathcal{P}_{1,k-1}^+ A_{1,k} - \mathcal{P}_{2,k-1}^+ A_{2,k} \| + \| \mathcal{P}_{2,k-1}^+ A_{2,k}  \| \|A_{1,k}^* - A_{2,k}^* \| \\ 
		& \quad + \| \mathcal{Q}_{1,k} - \mathcal{Q}_{2,k}  \| \\
		& \leq \|A_{1,k} \| \| \mathcal{P}_{1,k-1}^+ A_{1,k} - \mathcal{P}_{2,k-1}^+ A_{2,k} \| + \| \mathcal{P}_{2,k-1}^+ A_{2,k}  \| \|A_{1,k} - A_{2,k} \| \\ 
		& \quad + \| \mathcal{Q}_{1,k} - \mathcal{Q}_{2,k}  \| \\
		& \leq \|A_{1,k} \| (\| \mathcal{P}_{1,k}^+ A_{1,k} - \mathcal{P}_{1,k-1}^+ A_{2,k} \| + \| \mathcal{P}_{1,k-1}^+ A_{2,k} - \mathcal{P}_{2,k-1}^+ A_{2,k} \| ) \\
		& \quad + \| \mathcal{P}_{2,k-1}^+ \| \| A_{2,k}  \| \|A_{1,k} - A_{2,k} \| \\ 
		& \quad + \| \mathcal{Q}_{1,k} - \mathcal{Q}_{2,k}  \| \\
		& \leq \|A_{1,k} \| ( \| \mathcal{P}_{1,k-1}^+ \| \|  A_{1,k} - A_{2,k} \| + \| A_{2,k} \| \| \mathcal{P}_{1,k-1}^+  - \mathcal{P}_{2,k-1}^+  \| ) \\
		& \quad + \| \mathcal{P}_{2,k-1}^+ \| \| A_{2,k}  \| \|A_{1,k} - A_{2,k} \| \\ 
		& \quad + \| \mathcal{Q}_{1,k} - \mathcal{Q}_{2,k}  \|.
	\end{aligned}
\end{equation*}

Se analizará finalmente el término $\| \S_{1,k}^{-1} -  \S_{2,k}^{-1}\|$. Para ello, se observa que
\begin{equation*}
	\S_{1,k}^{-1} -  \S_{2,k}^{-1} = \S_{2,k}^{-1} (\S_{2,k} - \S_{1,k}) \S_{1,k}^{-1}.
\end{equation*}
A partir de esta expresión, se tiene que
\begin{equation*}
	\begin{aligned}
		\| \S_{1,k}^{-1} -  \S_{2,k}^{-1} \| & \leq  \| \S_{2,k}^{-1} (\S_{2,k} - \S_{1,k}) \S_{1,k}^{-1} \| \\
		& \leq \| \S_{1,k}^{-1} \| \|  \S_{2,k}^{-1} \| \| \S_{2,k} - \S_{1,k}\| \\
		& \leq  \| \S_{1,k}^{-1} \| \|  \S_{2,k}^{-1} \|  \| C_{1,k} \mathcal{P}_{1,k}^- C_{1,k}^* + \mathcal{R}_{1,k} - C_{2,k} \mathcal{P}_{2,k}^- C_{2,k}^* + \mathcal{R}_{2,k} \|.
	\end{aligned}
\end{equation*}
En este contexto, y de manera análoga a lo previamente demostrado, se obtiene la siguiente estimación para la expresión anterior:
\begin{equation*}
	\begin{aligned}
		\| C_{1,k} \mathcal{P}_{1,k}^- C_{1,k}^* + \mathcal{R}_{1,k} - C_{2,k} \mathcal{P}_{2,k}^- C_{2,k}^* + \mathcal{R}_{2,k} \| \\
		& \leq \|C_{1,k-1} \|  \| \mathcal{P}_{1,k-1}^+ \| \|  C_{1,k} - C_{2,k} \|  \\
            & \quad + \|C_{1,k-1} \|  \| C_{2,k} \| \| \mathcal{P}_{1,k-1}^+  - \mathcal{P}_{2,k-1}^+  \| \\
		& \quad + \| \mathcal{P}_{2,k-1}^+ \| \| C_{2,k}  \| \|C_{1,k} - C_{2,k} \| \\
		& \quad+ \| \mathcal{R}_{1,k} - \mathcal{R}_{2,k}  \|.
	\end{aligned}
\end{equation*}

\end{proof}

Con lo anterior, se concluye que, para una iteración $k$, el error depende tanto del error en la condición para la estimación del estado como del operador de covarianza del error a posteriori.

\section{Koopman Kalman Filter}

En esta sección se presenta la deducción del algoritmo de filtraje solo utilizando la teoría de RKHS junto con el operador de Koopman de manera meticulosa. Primero se deduce una dinámica del \textit{embedding}, y sus observaciones asociadas.

\begin{prop}
    El \textit{embedding} de $\{\mathbf{x}_k\}_{k=0}^T$ en $\H_\X$ satisface
    \begin{align*}
        \Phi_\X (\mathbf{x}_{k+1}) &= C_{X^+|X} \Phi_\X (\mathbf{x}_k) + \zeta_k \\
        \mathbf{y}_k &= C_{Y|X} \Phi_\X (\mathbf{x}_k) + \nu_k,
    \end{align*}
    donde $\zeta_k$ y $\nu_k$ son realizaciones de variables aleatorias a valores en $\H_\X$ y $\R^p$, respectivamente, ambas con media nula y segundo momento finito, con operador de covarianza $\mathcal{Q}_k : \H_\X \to \H_\X$ y matriz de covarianza $\mathcal{R}_k \in \R^{p \times p}$ semidefinida positiva, respectivamente. Si además, $\mathbf{g}$ cumple que
    \begin{equation}
    \label{eq:condi_g}
        \E[\| (\mathbf{g}(\mathbf{x}, \cdot) - \E[\mathbf{g}(\mathbf{x}, \cdot)])^\top v \|] = 0 \iff v = 0
    \end{equation}
    entonces $\mathcal{R}_k \in \R^{p \times p}$ es definida positiva.
\end{prop}

\begin{proof}
    Primero para la dinámica
    \begin{equation*}
	\begin{aligned}
		\Phi_\X (\mathbf{x}_{k+1}) &= \Phi_\X (\mathbf{f}(\mathbf{x}_k, \mathbf{w}_k)) \\
        &= \E[\Phi_\X (X^+) | X = \mathbf{x}_k] + \Phi_\X (\mathbf{f}(\mathbf{x}_k, \mathbf{w}_k)) - \E[\Phi_\X (X^+) | X = \mathbf{x}_k]\\
        &= C_{X^+|X} \Phi_\X (\mathbf{x}_k) + \zeta_k
	\end{aligned}
\end{equation*}
donde 
\begin{equation*}
    \zeta_k = \Phi_\X (\mathbf{f}(\mathbf{x}_k, \mathbf{w}_k)) - \E[\Phi_\X (X^+) | X = \mathbf{x}_k]
\end{equation*}
es una variable aleatoria infinita dimensional cuyo operador de covarianza está acotado y se denota por 
\[
\mathcal{Q}_k := \int_\X \Phi_\X (x) \otimes \Phi_\X(x) d \rho_f (\mathbf{x}_k,x). 
\]
Además, esta variable aleatoria tiene media nula, esto ya que
\begin{equation*}
    \mathbf{f}(\mathbf{x}_k, \mathbf{w}_k) \sim X^+ | X = \mathbf{x}_k,
\end{equation*}
con lo que
\begin{equation*}
    \begin{aligned}
         \E[\zeta_k] &= \E[\Phi_\X (\mathbf{f}(\mathbf{x}_k, \mathbf{w}_k))] - \E[\E[\Phi_\X (X^+) | X = \mathbf{x}_k]] \\
         &= \E[\E[\Phi_\X (X^+) | X = \mathbf{x}_k]] - \E[\E[\Phi_\X (X^+) | X = \mathbf{x}_k]] \\
         &= 0.
    \end{aligned}
\end{equation*}
De manera análoga, para el \textit{embedding} de la observación se tiene
\begin{equation*}
	\begin{aligned}
		\Phi_\Y (\mathbf{y}_{k}) &= \Phi_\Y (\mathbf{g}(\mathbf{x}_k, \mathbf{v}_k)) \\
        &= \E[\Phi_\Y (Y) | X = \mathbf{x}_k] + \Phi_\Y (\mathbf{g}(\mathbf{x}_k, \mathbf{v}_k)) - \E[\Phi_\Y (Y) | X = \mathbf{x}_k]\\
        &= C_{Y|X} \Phi_\X (\mathbf{x}_k) + \nu_k
	\end{aligned}
\end{equation*}
donde $\nu_k$ es una variable aleatoria infinita dimensional cuyo operador de covarianza está acotado, denotado por
\[
\mathcal{R}_k := \E[ (\mathbf{g}(\mathbf{x}, \cdot) - \E[\mathbf{g}(\mathbf{x}, \cdot)])^\top (\mathbf{g}(\mathbf{x}, \cdot) - \E[\mathbf{g}(\mathbf{x}, \cdot)]) ],
\]
que es semidefinida positiva ya que
\[
v^\top \E[ (\mathbf{g}(\mathbf{x}, \cdot) - \E[\mathbf{g}(\mathbf{x}, \cdot)])^\top (\mathbf{g}(\mathbf{x}, \cdot) - \E[\mathbf{g}(\mathbf{x}, \cdot)]) ] v = \E[\| (\mathbf{g}(\mathbf{x}, \cdot) - \E[\mathbf{g}(\mathbf{x}, \cdot)])^\top v \|^2] \geq 0,
\]
y si cumple \ref{eq:condi_g}, se cumple $\mathcal{R}_k $ es definida positiva. Y está centrada ya que
\begin{equation*}
    \mathbf{g}(\mathbf{x}_k, \mathbf{v}_k) \sim Y | X = \mathbf{x}_k
\end{equation*}
con lo que
\begin{equation*}
    \begin{aligned}
         \E[\nu_k] &= \E[\Phi_\Y (\mathbf{g}(\mathbf{x}_k, \mathbf{v}_k))] - \E[\E[\Phi_\Y (Y) | X = \mathbf{x}_k]] \\
         &= \E[\E[\Phi_\Y (Y) | X = \mathbf{x}_k]] - \E[\E[\Phi_\Y (Y) | X = \mathbf{x}_k]] \\
         &= 0.
    \end{aligned}
\end{equation*}
\end{proof}

\begin{obs}
    En lo que sigue se supondrá que se cumple \ref{eq:condi_g}, lo que hará que se satisfaga la hipótesis del teorema \ref{teo:error_kalman} sobre la invertibilidad de los $\mathcal{S}_k$ al ser suma de una matriz simétrica semidefinida positiva y una matriz simétrica definida positiva, argumento que también se verá en la deducción del operador de ganancia de Kalman. 
    
    Un ejemplo en donde esto se cumple es cuando $\mathbf{g}$ es aditiva y el ruido tiene matriz de covarianza definida positiva, esto es, si
    \[
    \mathbf{g}(\mathbf{x}, \mathbf{v}) = \Tilde{\mathbf{g}}(\mathbf{x}) + \mathbf{v}
    \]
    entonces
    \[
    \E[\mathbf{g}(\mathbf{x, \cdot)}] = \Tilde{\mathbf{g}}(\mathbf{x}),
    \]
    así
    $\mathcal{R}_k = \E[\mathbf{v_k} \mathbf{v_k}^\top]$, que es definida positiva.
\end{obs}

Siguiendo un procedimiento similar al utilizado en \cite{Gebhard2019}, se define
\begin{equation*}
	\hat{\mu}_k = \mathbb{E} [\Phi_\X (\mathbf{x}_k) | \mathbf{y}_{1:k}], \quad \mathcal{P}_{k} = \text{Cov}(\Phi_\X(\mathbf{x}_k) - \hat{\mu}_k)
\end{equation*}
con inicialización dada por
\begin{equation*}
	\hat{\mu}_0 = \hat{\mu}_0^- = \mathbb{E} [\Phi_\X (\mathbf{x}_0)], \quad \mathcal{P}_{0} = \text{Cov}(\Phi_\X (\mathbf{x}_0) - \hat{\mu}_0).
\end{equation*}
Se define también una forma para la predicción
\begin{equation*}
	\hat{\mu}_{k+1}^- = \mathbb{E} [\Phi_\X (\mathbf{x}_{k+1}) | \mathbf{y}_{1:k}], \quad \mathcal{P}_{k+1}^- = \text{Cov}(\Phi_\X (\mathbf{x}_{k+1}) - \hat{\mu}_{k+1}^-),
\end{equation*}
con lo que se entrega una forma más cerrada e iterativa para estos elementos.

\begin{prop}[Sobre la predicción]
    $\hat{\mu}_{k+1}^- $ y $\mathcal{P}_{k+1}^-$ satisfacen
    \[
    \hat{\mu}_{k+1}^- = C_{X^+|X} \hat{\mu}_k, \quad \mathcal{P}_{k+1}^- = C_{X^+|X} \mathcal{P}_k (C_{X^+|X})^* + \mathcal{Q}_{k+1}
    \]
\end{prop}

\begin{proof}
    Lo primero es directo de la \textit{kernel} Bayes Rule \cite{Fukumizu2013KernelKernels}, es decir
\begin{equation*}
	\hat{\mu}_{k+1}^- = \mathbb{E} [\Phi_\X (\mathbf{x}_{k+1}) | \mathbf{y}_{1:k}] = C_{X^+|X} \mathbb{E} [\Phi_\X (\mathbf{x}_{k}) | \mathbf{y}_{1:k}] = C_{X^+|X} \hat{\mu}_k.
\end{equation*}
A partir de esto, y utilizando la independencia de $\zeta_k$ con respecto a $\Phi_\X (\mathbf{x}_k)$, se obtiene
\begin{equation*}
	\begin{aligned}
		\mathcal{P}_{k+1}^- &= \text{Cov}(\Phi_\X (\mathbf{x}_{k+1}) - \hat{\mu}_{k+1}^-)  \\
		&= \text{Cov}(C_{X^+|X}\Phi_\X (\mathbf{x}_{k}) + \zeta_{k+1} - C_{X^+|X}\hat{\mu}_{k}) \\
		&= C_{X^+|X} \text{Cov} (\Phi_\X (\mathbf{x}_{k}) - \hat{\mu}_{k})C_{X^+|X}^* + \text{Cov}(\zeta_{k+1}) \\
		&= C_{X^+|X} \mathcal{P}_k (C_{X^+|X})^* + \mathcal{Q}_{k+1}
	\end{aligned}
\end{equation*}
\end{proof}

Posteriormente, se debe proyectar sobre las observaciones para obtener la estimación a posteriori, es decir,
\begin{equation*}
	\hat{\mu}_{k+1}^- = \mathbb{E} [\Phi_\X (\mathbf{x}_{k+1}) | \mathbf{y}_{1:k}] \quad \text{se actualiza a} \quad \hat{\mu}_{k+1} = \mathbb{E} [\Phi_\X (\mathbf{x}_{k+1}) | \mathbf{y}_{1:k+1}].
\end{equation*}

Por lo que se debe agregar un factor de proyección, que indique la dirección a proyectar, por lo que propone un operador $\K_k : \mathbb{R}^p \to \mathcal{H}_\X$, denominado el operador de ganancia de Kalman, tal que
\begin{equation*}
	\hat{\mu}_{k+1} = \hat{\mu}_{k+1}^- + \K_{k+1} (\mathbf{y}_{k+1} - C_{Y|X} \hat{\mu}_{k+1}^-).
\end{equation*}
En Gebhardt et al. \cite{Gebhard2019} se deduce que $\hat{\mu}_k$ es un estimador insesgado de $\mu_k$, para todo $k$ y además una expresión para el operador de ganancia, que se deja a continuación por completitud de la deducción del filtro, en donde se ha ajustado la notación a lo presentado anteriormente.

\begin{prop}[Sobre la actualización, Adaptación de Gebhardt et al. \cite{Gebhard2019}]
\label{prop:unbias_kalman_operator}
    El estimador $\hat{\mu}_k$ es insesgado para $\mu_k$, para todo $k$ y el operador de ganancia de Kalman $\K_k: \R^p \to \H_\X$ viene dado por
    \begin{equation*}
        \mathcal{K}_k = \mathcal{P}^-_{k}C_{Y|X}^*(C_{Y|X}\mathcal{P}^-_{k}C_{Y|X}^* + \mathcal{R}_k)^{-1}.
    \end{equation*}
    Además, el operador de covarianza de error a posteriori viene dado por
    \begin{equation*}
    \mathcal{P}_k = (I - \K_k C_{Y|X})\mathcal{P}^-_{k}.
\end{equation*}
\end{prop}
\begin{proof}
    Denotando $\varepsilon_k^- = \hat{\mu}_k^- - \mu_k \in \H_\X$ el error a priori y $\varepsilon_k^+ = \hat{\mu}_k - \mu_k \in \H_\X$ el error a posteriori, se tiene lo siguiente
    \begin{align*}
    \varepsilon_k^+ &= \hat{\mu}_k - \mu_k \\
                &=\hat{\mu}_{k}^- + \K_{k} (\mathbf{y}_{k} - C_{Y|X} \hat{\mu}_{k}^-) - \mu_k \\
                &= \varepsilon_k^- +  \K_{k} \mathbf{y}_{k} - \K_k C_{Y|X} \hat{\mu}_{k}^- \\
                &= \varepsilon_k^- +  \K_{k} \mathbf{y}_{k} - \K_k C_{Y|X} \hat{\mu}_{k}^- + \K_k C_{Y|X} \mu_k - \K_k C_{Y|X} \mu_k\\
                &= \varepsilon_k^- + \K_k C_{Y|X} (-\hat{\mu}_{k}^- + \mu_k) + \K_{k} \mathbf{y}_{k} - \K_k C_{Y|X} \mu_k\\
                &= \varepsilon_k^- - \K_k C_{Y|X}  \varepsilon_k^- + \K_{k} \mathbf{y}_{k} - \K_k C_{Y|X} \mu_k\\
                &= ( I - \K_k C_{Y|X} ) \varepsilon_k^- + \K_{k} (\mathbf{y}_{k} - C_{Y|X} \mu_k)\\
                &= ( I - \K_k C_{Y|X} ) \varepsilon_k^- + \K_{k} \nu_k
\end{align*}

Primero se probará que $\E[\varepsilon_k^-] = 0$, esto por inducción. En primer lugar, para $k=0$ se tiene por construcción, con lo que el caso el caso base queda probado. Suponiendo que se cumple para $k \in \N$, se propone probarlo para $k+1$. Notando que 
\begin{equation*}
    \varepsilon_{k+1}^- = \hat{\mu}_{k+1}^- - \mu_{k+1} = C_{X^+|X} \hat{\mu}_k - C_{X^+|X} \mu_k - \zeta_k,
\end{equation*}
se tiene que
\begin{equation*}
    \E[\varepsilon_{k+1}^-] = C_{X^+|X} \E [\hat{\mu}_k] - \E[\zeta_k] = C_{X^+|X} \E[\varepsilon^+_k],
\end{equation*}
ya que las variables aleatorias $\zeta_k$ son centradas. Con ello
\begin{equation*}
    \E[\varepsilon^+_k] = ( I - \K_k C_{Y|X} ) \E[\varepsilon_k^-] + \K_{k} \E[\nu_k] = 0,
\end{equation*}
ya los $\nu_k$ son centrados y por la hipótesis de inducción $\E[\varepsilon_k^-] = 0$. Luego por principio de inducción queda probado que $\E[\varepsilon_k^-] = 0$, para todo $k$.  Haciendo la misma manipulación de antes, se prueba que $\E[\varepsilon^+_k] = 0$, para todo $k$, con lo que el estimado $\hat{\mu}_k$ es insesgado para $\mu_k$.

Ahora para el operador de ganancia se debe recordar que en el problema de filtraje se busca minimizar la pérdida cuadrática esperada asociada a la estimación de la trayectoria, la que viene dada por
\begin{equation*}
    \E[(\hat{\mu}_{k} - \mu_{k})^* (\hat{\mu}_{k} - \mu_{k}) ] = \E [ (\varepsilon_k^+)^* \varepsilon_k^+ ].
\end{equation*}
Dado que el estimador es insesgado, esto se puede reformular como la minimización de la traza de la covarianza de error a posteriori $\mathcal{P}_k$, es decir
\begin{align*}
\min_{\K_k} \mathbb{E}[(\varepsilon_k^+)^* \varepsilon_k^+] &= \min_{\K_k} \text{Tr} \mathbb{E}[\varepsilon_k^+(\varepsilon_k^+)^*] \\
&= \min_{\K_k} \text{Tr} \mathcal{P}_{k}.
\end{align*}
Sustituyendo la expresión para el error a posteriori se tiene que
\begin{align*}
\mathcal{P}_k &= \mathbb{E}[\varepsilon^+_k(\varepsilon^+_k)^*] \\
&= \mathbb{E}[((I - \K_k C_{Y|X})\varepsilon^-_k - \K_k \nu_k)((I - \K_k C_{Y|X})\varepsilon^-_k - \K_k \nu_k)^*] \\
&= (I - \K_k C_{Y|X})\mathbb{E}[\varepsilon^-_k(\varepsilon^-_k)^*](I - \K_k C_{Y|X})^* 
- \K_k\mathbb{E}[\nu_k(\varepsilon^-_k)^*](I - \K_k C_{Y|X})^* \\
&\quad - (I - \K_k C_{Y|X})\mathbb{E}[\varepsilon^-_k\nu_k^*]\K_k^* + \K_k\mathbb{E}[\nu_k\nu_k^*]\K_k^*.
\end{align*}
Dado que se asume que el ruido del operador de observación es independiente del error de estimación y dado que se asumió que la estimación a priori tiene media cero, se obtiene
\begin{equation*}
\mathbb{E}[\nu_k(\varepsilon^-_k)^*] = \mathbb{E}[\nu_k]\mathbb{E}[(\varepsilon^-_k)^*] = \mathbb{E}[(\varepsilon^-_k)^*]\mathbb{E}[\nu_k] = \mathbb{E}[\varepsilon^-_k\nu_k^*] = 0.
\end{equation*}

Con esta perspectiva, el operador de covarianza posterior puede reformularse como
\begin{equation*}
\mathcal{P}_k = (I - \K_k C_{Y|X})\mathcal{P}^-_{k}(I - \K_k C_{Y|X})^* + \K_k \mathcal{R}_k \K_k^*,
\end{equation*}
donde $\mathcal{R}_k = \mathbb{E}[\nu_k\nu_k^*]$ es la covarianza del ruido asociado a la observación. Tomando la derivada de la traza del operador de covarianza con respecto del operador de ganancia e igualándola a cero se obtiene
\begin{align*}
2(I - \K_k C_{Y|X})\mathcal{P}^-_{k}(-C_{Y|X}^*) + 2\K_k \mathcal{R}_k &= 0,
\end{align*}
que es equivalente a 
\begin{equation}
    -(I - \K_k C_{Y|X})\mathcal{P}^-_{k}C_{Y|X}^* + \K_k \mathcal{R}_k = 0.
    \label{eq:kalman_gain_rel}
\end{equation}

Con esto
\begin{align*}
\K_k C_{Y|X}\mathcal{P}^-_{k}C_{Y|X}^* + \K_k \mathcal{R}_k &= \K^-_{k}C_{Y|X}^*, \\
\K_k(C_{Y|X}\mathcal{P}^-_{k}C_{Y|X}^* + \mathcal{R}_k) &= \mathcal{P}^-_{k}C_{Y|X}^* .
\end{align*}
Aquí la decisión $\H_\Y = (\R^p)^*$ cobra sentido, ya que entonces, dado que $\mathcal{R}_k$ es una matriz simétrica definida positiva, es invertible, como $C_{Y|X}\mathcal{P}^-_{k}C_{Y|X}^*$ es una matriz simétrica semidefinida positiva, entonces el término que acompaña a $\K_k$ es invertible y por tanto
\begin{equation*} 
\K_k = \mathcal{P}^-_{k}C_{Y|X}^*(C_{Y|X}\mathcal{P}^-_{k}C_{Y|X}^* + \mathcal{R}_k)^{-1}.
\end{equation*}
Con lo que se obtiene la expresión requerida para el operador de ganancia de Kalman. 

Recobrando la expresión para el operador de covarianza de error a posteriori
\begin{equation*}
\mathcal{P}_k = (I - \K_k C_{Y|X})\mathcal{P}^-_{k}(I - \K_k C_{Y|X})^* + \K_k \mathcal{R}_k \K_k^*,
\end{equation*}
se obtiene que
\begin{equation*}
\mathcal{P}_k = (I - \K_k C_{Y|X})\mathcal{P}^-_{k} - (I - \K_k C_{Y|X})\mathcal{P}^-_{k} C_{Y|X}^* \K_k^* + \K_k \mathcal{R}_k \K_k^*,
\end{equation*}
y con ello
\begin{equation*}
\mathcal{P}_k = (I - \K_k C_{Y|X})\mathcal{P}^-_{k} + (-(I - \K_k C_{Y|X})\mathcal{P}^-_{k} C_{Y|X}^*  + \K_k \mathcal{R}_k) \K_k^*,
\end{equation*}
que, recordando de \eqref{eq:kalman_gain_rel} que se tiene
\begin{equation*}
    -(I - \K_k C_{Y|X})\mathcal{P}^-_{k} C_{Y|X}^*  + \K_k \mathcal{R}_k
\end{equation*}
se llega a que
\begin{equation*}
    \mathcal{P}_k = (I - \K_k C_{Y|X})\mathcal{P}^-_{k}.
\end{equation*}
\end{proof}

Entonces, con expresiones cerradas para el operador de ganancia de Kalman y los operadores de covarianza de error, las ecuaciones para cada iteración se expresan como
\begin{equation*}
	\begin{aligned}
		\hat{\mu}_{k+1}^- & = C_{X^+|X} \hat{\mu}_{k}, \\
		\mathcal{P}_{k+1}^- & = C_{X^+|X} \mathcal{P}_k (C_{X^+|X})^* + \mathcal{Q}_{k+1}, \\
		\S_{k+1} & = C_{Y|X} \mathcal{P}_{k+1}^- (C_{Y|X})^* + \mathcal{R}_{k+1}, \\
		\K_{k+1} & = \mathcal{P}_{k+1}^- (C_{Y|X})^* \S_{k+1}^{-1}, \\
		\mathcal{P}_{k+1} & = (I - \K_{k+1} C_{Y|X}) \mathcal{P}_{k+1}^-, \\
		\hat{\mu}_{k+1} &= C_{X^+|X} \hat{\mu}_k + \K_{k+1} (\mathbf{y}_{k+1} - C_{Y|X} \hat{\mu}_{k+1}^-),
	\end{aligned}
\end{equation*}
con las condiciones iniciales
\begin{equation*}
	\hat{\mu}_0 = \E [\Phi_\X (\mathbf{x}_0)], \quad \mathcal{P}_{0} = \text{Cov}(\Phi_\X (\mathbf{x}_0) - \hat{\mu}_0).
\end{equation*}
Ahora, expresando todo en términos del operador de Koopman, gracias a que
\begin{equation*}
	C_{X^+|X} = \U^*, \quad C_{Y|X} = \G^*,
\end{equation*}
se obtiene que
\begin{equation*}
	\begin{aligned}
		\hat{\mu}_{k+1}^- & = \U^* \hat{\mu}_{k}, \\
		\mathcal{P}_{k+1}^- & = \U^* \mathcal{P}_k \U + \mathcal{Q}_{k+1}, \\
		\S_{k+1} & = \G^* \mathcal{P}_{k+1}^- \G + \mathcal{R}_{k+1}, \\
		\K_{k+1} & = \mathcal{P}_{k+1}^- \G \S_{k+1}^{-1}, \\
		\mathcal{P}_{k+1} & = (I - \K_{k+1} \G^*) \mathcal{P}_{k+1}^-, \\
		\hat{\mu}_{k+1} &= \U^* \hat{\mu}_k + \K_{k+1} (\mathbf{y}_{k+1} - \G^* \hat{\mu}_{k+1}^-).
	\end{aligned}
\end{equation*}
Ahora se proponen las siguiente aproximaciones que poseen representación en dimensión finita en base a lo visto en el capítulo anterior
\begin{equation*}
	\begin{aligned}
		\hat{\mu}_{N, k+1}^- & = \U^*_N \hat{\mu}_{N, k}, \\
		\mathcal{P}_{N, k+1}^- & = \U^*_N \mathcal{P}_{N,k} \U_N + \mathcal{Q}_{N, k+1}, \\
		\K_{N,k+1} & = \mathcal{P}_{N, k+1}^- \G_N (\G^*_N \mathcal{P}_{N, k+1}^- \G_N + \mathcal{R}_{N, k+1})^{-1}, \\
		\mathcal{P}_{N, k+1} & = (I - \K_{N,k+1} \G_N^*) \mathcal{P}_{N,k+1}^-, \\
		\hat{\mu}_{N,k+1} &= \U_N^* \hat{\mu}_{N,k} + \K_{N,k+1} (\mathbf{y}_{k+1} - \G^*_N \hat{\mu}_{N,k+1}^-).
	\end{aligned}
\end{equation*}
En donde $\mathcal{Q}_{N,k+1}$, $\mathcal{R}_{N,k+1}$ son los estimadores insesgados de $\mathcal{Q}_{k+1}$ y $\mathcal{R}_{k+1}$, respectivamente, es decir:
\begin{equation}
	\mathcal{Q}_{N,k+1} = \frac{1}{N-1}\sum_{j=1}^N (z_{1,j} - \bar{z}_1)(z_{1,j} - \bar{z}_1)^\top, \quad \mathcal{R}_{N,k+1} = \frac{1}{N-1}\sum_{j=1}^N (z_{2,j} - \bar{z}_2)(z_{2,j} - \bar{z}_2)^\top,
	\label{eq: emp_covars}
\end{equation}
donde $\{ z_{1,j} \}_{j=1}^N \sim \zeta_k^N$, $\{ z_{2,j} \}_{j=1}^N \sim \nu_k^N$ y 
\begin{equation*}
	\bar{z}_i = \frac{1}{N} \sum_{j=1}^N z_{i,j}.
\end{equation*}
Si $X_0$ es la distribución dada para la condición inicial y $\{ x_j \}_{j=1}^N \sim X_0$, entonces la inicialización viene dada por:
\begin{equation}
	\hat{\mu}_{N,0} = \frac{1}{N} \sum_{j=1}^N \Phi_\X(x_{k}), \quad \mathcal{P}_{N,0} = \frac{1}{N-1} \sum_{j=1}^N (\Phi_\X(x_{k}) - \hat{\mu}_{N,0})^2.
	\label{eq: mean_element_covar}
\end{equation}

Es con todo esto que se concluye la deducción del filtro, tanto en dimensión infinita como su representante en dimensión infinita, para este último se puede encontrar su pseudo-código en el algoritmo \ref{alg:KKF}.

\begin{algorithm}[h]
\caption{Cov($W, N, h$)}\label{alg:AppCov}
\begin{algorithmic}[1]
\State \textbf{Entrada:} $W$ ley de una variable \textit{sampleable} con soporte en un conjunto $\Omega$, $N \geq 2$ cantidad de muestras a tomar, $h:\Omega \to \R^d$ función.
\Ensure $\hat{\Sigma} \in \R^{d \times d}$ estimación de matriz de covarianzas de $h(W)$.
\State $\{ \mathbf{w}_i \}_{i=1}^N \sim W^N$ \Comment{$N$ muestras independientes de $W$} 
\State $\hat{h} = \frac{1}{N} \sum_{i=1}^N h (\mathbf{w}_i)$ \Comment{Promedio empírico}
\State $\hat{\Sigma} = \frac{1}{N-1} \sum_{i=1}^N (h (\mathbf{w}_i) - \hat{h})^T(h (\mathbf{w}_i) - \hat{h})$  \Comment{Covarianza muestral insesgada}
\end{algorithmic}
\end{algorithm}

\begin{algorithm}[h]
\caption{Kalman Koopman Filter (KKF)}\label{alg:KKF}
\begin{algorithmic}[1]
\State \textbf{Entrada:} Dinámica discreta como en (\ref{eq:no_lin_dis_chap3}), $\mathbf{x}_0$ \textit{prior} sobre la condición inicial, $\mathbf{y}_{1:N}$ observaciones, $\mathbf{k}:\X \times \X \to \R$ \textit{kernel} semidefinido positivo, $N$ dimensión de aproximación del operador de Koopman y $n_{\text{samples}}$ cantidad de muestras para aproximar las matrices de covarianza.
\State \textbf{Salida:} $(\hat{\mathbf{x}}_{k|k})_{k=0}^{N}$ estimador de la trayectoria y $(\hat{\mathbf{P}}^{\mathbf{x}}_{k|k})_{k=0}^{N}$ matrices de covarianza de error.
\State $\mathbf{U}_N, \, \Phi_N (\cdot), \, \mathbf{G}_N, \mathbf{B}_N \gets $ kEDMD($\mu_\X$, $\rho_f$, $\rho_g$, $k$, $N$)
\State $\hat{\mathbf{x}}_{0|0}   \gets \E [\mathbf{x}_0]$ \Comment{Estimación de la condición inicial para $\mathbf{x}$}
\State $\hat{\mathbf{z}}_{0|0}   \gets \mathbf{\Phi}_N(\hat{\mathbf{x}}_{0|0})$ \Comment{Estimación de la condición inicial para $\mathbf{z}$}
\State $\hat{\mathbf{P}}^\mathbf{x}_{0|0} \gets \E [(\mathbf{x}_0 - \hat{\mathbf{x}}_{0})(\mathbf{x}_0 - \hat{\mathbf{x}}_{0})^T]$ \Comment{Covarianza de error inicial para $\mathbf{x}$}
\State $\hat{\mathbf{P}}^\mathbf{z}_{0|0} \gets$ Cov($\mathbf{x}_0$, $n_{\text{samples}}, \mathbf{\Phi}_N$) \Comment{Covarianza de error inicial para $\mathbf{z}$}
\For{$k = 0, \dots, N-1$}
    \State $\hat{\mathbf{x}}_{k+1|k} \gets \Tilde{\mathbf{f}}(\mathbf{x}_{k|k})$
    \Comment{Estimación a priori para $\mathbf{x}$}
    \State $\hat{\mathbf{z}}_{k+1|k} \gets \mathbf{\Phi}_N(\hat{\mathbf{x}}_{k+1|k})$
    \Comment{Estimación a priori para $\mathbf{z}$}
    \State $\mathbf{Q}_k \gets $ Cov($\mathbf{w}_k$, $n_{\text{samples}}, \mathbf{\Phi}_N(\mathbf{f}(\mathbf{x}_{k|k}, \cdot))$) 
    \Comment{Covarianza de la dinámica para $\mathbf{z}$}
    \State $\mathbf{P}^{\mathbf{z}}_{k+1|k} \gets \mathbf{U} \mathbf{P}^{\mathbf{z}}_{k|k} \mathbf{U}^T + \mathbf{Q}_k$
    \Comment{Error de covarianza a priori}
    \State $\hat{\mathbf{y}}_{k+1|k} \gets \Tilde{\mathbf{g}}(\hat{\mathbf{x}}_{k+1|k})$ 
    \Comment{Estimación de observación a priori}
    \State $\mathbf{e}_{\mathbf{y}_{k+1|k}} \gets \mathbf{y}_{k+1} - \hat{\mathbf{y}}_{k+1|k}$
    \Comment{Error a priori (innovación)}
    \State $\mathbf{R}_{k+1} \gets $ Cov($\mathbf{v}_k$, $n_{\text{samples}}, \mathbf{g}(\mathbf{x}_{k+1|k}, \cdot)$) 
    \Comment{Covarianza de la observación para $\mathbf{z}$}
    \State $ \mathbf{S}_{k+1} \gets \mathbf{C} \mathbf{P}^{\mathbf{z}}_{k|k} \mathbf{C}^T + \mathbf{R}_{k+1}$
    \Comment{Covarianza residual para $\mathbf{z}$}
    \State $\mathbf{K}_{k+1} \gets \mathbf{P}^{\mathbf{z}}_{k+1|k} \mathbf{C}^T$Cholesky$(\mathbf{S}_{k+1})^{-1}$
    \Comment{Ganancia de Kalman}
    \State $\hat{\mathbf{z}}_{k+1|k+1} \gets \hat{\mathbf{z}}_{k+1|k} + \mathbf{K}_{k+1} \mathbf{e}_{\mathbf{y}_{k+1|k}}$
    \Comment{Estimación a posteriori para $\mathbf{z}$}
    \State $\hat{\mathbf{x}}_{k+1|k+1} \gets \mathbf{B}\hat{\mathbf{z}}_{k+1|k+1}$
    \Comment{\textit{Lift back} para el estado}
    \State $\mathbf{P}^\mathbf{z}_{k+1|k+1} \gets (\mathbf{I} - \mathbf{K}_{k+1} 
    \mathbf{C}) \mathbf{P}^{\mathbf{z}}_{k+1|k}$
    \Comment{Error de covarianza a posteriori para $\mathbf{z}$}
    \State $\mathbf{P}^\mathbf{x}_{k+1|k+1} \gets \mathbf{B}\mathbf{P}^\mathbf{z}_{k+1|k+1} \mathbf{B}^T$
    \Comment{\textit{Lift back} para la covarianza}
\EndFor
\end{algorithmic}
\end{algorithm}

\section{Cota de error del filtro}

El primer paso para deducir la cota de error es dejar la discrepancia en norma de los elementos a comparar, en un cierto instante $k$, en función del instante anterior $k-1$ y en función de los operadores involucrados.

\begin{prop}
	Para $k \geq 1$, existen constantes $c_{k,j}^i$ con $j \in \{ 1, \dots, 6\}$, $i \in \{ 1, 2\}$ tales que
	\begin{equation*}
		\begin{aligned}
			\| \hat \mu_{k} - \hat \mu_{N,k}  \| \leq & \, c_{1,k}^1 \| \U - \U_N \| +  c_{2,k}^1 \| \G - \G_N \| \\ 
			&+ c_{3,k}^1 \| \mathcal{Q}_{k} - \mathcal{Q}_{N, k} \| +c_{4,k}^1 \| \mathcal{R}_{k} - \mathcal{R}_{N, k} \| \\
			& + c_{5,k}^1 \| \hat \mu_{k-1} - \hat \mu_{N, k-1} \| + c_{6,k}^1 \| \mathcal{P}_{k-1} - \mathcal{P}_{N, k-1} \|,
		\end{aligned}
	\end{equation*}
	\begin{equation*}
		\begin{aligned}
			\| \mathcal{P}_{k} - \mathcal{P}_{N,k} \| \leq & \, c_{1,k}^2 \| \U - \U_N \| +  c_{2,k}^2 \| \G - \G_N \| \\ 
			&+ c_{3,k}^2 \| \mathcal{Q}_{k} - \mathcal{Q}_{N, k} \| +c_{4,k}^2 \| \mathcal{R}_{k} - \mathcal{R}_{N, k} \| \\
			& + c_{5,k}^2 \| \hat \mu_{k-1} - \hat \mu_{N, k-1} \| + c_{6,k}^2 \| \mathcal{P}_{k-1} - \mathcal{P}_{N, k-1} \|.
		\end{aligned}
	\end{equation*}
	En donde las constantes $c_{k,j}^i$ son positivas y dependen de $k$ solo a través de $\| \U \| $, $\| \G \| $, $\| \mathcal{Q}_{k} \| $, $\| \mathcal{R}_{k} \| $, $\| \S_{k}^{-1} \| $, $\| \hat{\mu}_{k-1} \| $ y $\| \mathcal{P}_{k-1} \| $.
	\label{prop:err_KKF_1}
\end{prop}

\begin{proof}
    Esto es directo del teorema \ref{teo:error_kalman}, notando que los $\mathcal{S}_k$ son invertibles al ser $\mathcal{R}_k$ una matriz simétrica definida positiva. Ocupando
    \begin{equation*}
        A_{1,k} = \U, \quad A_{2,k} = \U_N 
    \end{equation*}
    \begin{equation*}
        C_{1,k} = \G, \quad C_{2,k} = \G_N 
    \end{equation*}
    \begin{equation*}
        \mathcal{Q}_{1,k} = \mathcal{Q}_k, \quad \mathcal{Q}_{2,k} = \mathcal{Q}_{N, k}
    \end{equation*}
    \begin{equation*}
        \mathcal{R}_{1,k} = \mathcal{R}_k, \quad \mathcal{R}_{2,k} = \mathcal{R}_{N, k}
    \end{equation*}
    \begin{equation*}
        y_{1, k} = y_{2, k} = \mathbf{y}_k.
    \end{equation*}
    Es decir, no se tiene el error por diferencia en las observaciones, ya que se considera que ambos sistemas tienen las mismas observaciones.
\end{proof}

Esto da paso directo a poder hacer inducción para poder dejar todo en función de un único orden de decaimiento del error, para lo que se utilizará el resultado probado en el capítulo anterior para la convergencia de kEDMD. 

\begin{teo}
    Sean $\delta \in (0, 1)$ y $N \in \N$ tales que
\[
\delta > 2 \text{exp} \left ( -\frac{Nc_1^2}{8\|k\|_\infty}\right )
\]
con $c_1$ la constante de inyectividad de $C_X$. Si $\U \H_\X \subset \H_\X$, $\G (\R^p)^* \subset \H_\X$ y $\B (\R^n)^* \subseteq \H_\X$, entonces existen constantes $C_{k,\delta}^1$ y $C_{k,\delta}^2$, tales que con probabilidad al menos $(1 - \delta)^4$ ocurre que
	\begin{equation*}
		\begin{aligned}
			\| \hat \mu_{k} - \hat \mu_{N,k}  \| \leq & \, C_{k, \delta}^1 N^{-1/2}
		\end{aligned}
	\end{equation*}
	\begin{equation*}
		\begin{aligned}
			\| \mathcal{P}_{k} - \mathcal{P}_{N,k}  \| \leq & \, C_{k, \delta}^2 N^{-1/2} 
		\end{aligned}
	\end{equation*}
	En donde las constantes $C_{k,\delta}^j = \Tilde{C}_k^j C_\delta$, con $C_\delta$ la constante definida en el teorema \ref{teo:error_koop_sqrt_N_def} y las $\Tilde{C}_k^j$ son positivas y dependen de $k$ solo a través de $\| \U \| $, $\| \G \| $, $\| \mathcal{Q}_{j} \| $, $\| \mathcal{R}_{j} \| $, $\| \S_{k}^{-1} \| $, $\| \hat{\mu}_{j} \| $ y $\| \mathcal{P}_{j} \| $, con $j \in \{ 0, \dots, k-1\}$.
    \label{teo:teo_KKF_2}
\end{teo}

Antes de probar esta proposición, se enuncia un lema que permite dar cotas para los elementos y operadores cuya norma se puede acotar por algo de orden $N^{-1/2}$, que son resultados conocidos en la literatura.
\begin{lema}[Zhou et al. \cite{Zhou2019ASpaces}] Sea $N \in \N$ y $\mathcal{Q}_{N,0}$, $\mathcal{R}_{N,0}$, $\mu_{N,0}$, $\mathcal{P}_{N,0}$, definidos en \ref{eq: emp_covars} y \ref{eq: mean_element_covar}, respectivamente, luego existe una constante $C$ tal que
	\begin{equation*}
		\| \hat{\mu}_0 - \hat{\mu}_{N,0} \|_{\H_\X}, \, \|  \mathcal{P}_{0} -  \mathcal{P}_{N,0}\|_{HS}, \, \, \| \mathcal{Q}_{k} - \mathcal{Q}_{N, k} \|_{HS}, \| \mathcal{R}_{k} - \mathcal{R}_{N, k} \|_{HS} \leq C\cdot N^{-1/2}
	\end{equation*}
	que, sin pérdida de generalidad, se puede tomar común para todas las cotas.
	\label{lema:oper_sqrt_N}
\end{lema}
Ahora se procede con la demostración de la Proposición \ref{teo:teo_KKF_2}.
\begin{proof}
	Gracias a la proposición \ref{prop:err_KKF_1} y el lema \ref{lema:oper_sqrt_N} se obtiene que existen constantes $c_{k,j}^i$ con $j \in \{ 1, \dots, 6\}$, $i \in \{ 1, 2\}$ tales que
	\begin{equation*}
		\begin{aligned}
			\| \hat \mu_{k} - \hat \mu_{N,k}  \| \leq & \, c_{1,k}^1 \| \U - \U_N \| +  c_{2,k}^1 \| \G - \G_N \| \\ 
			&+ c_{3,k}^1 C N^{-1/2}+c_{4,k}^1 C N^{-1/2} \\
			& + c_{5,k}^1 \| \hat \mu_{k-1} - \hat \mu_{N, k-1} \| + c_{6,k}^1 \| \mathcal{P}_{k-1} - \mathcal{P}_{N, k-1} \|,
		\end{aligned}
	\end{equation*}
	\begin{equation*}
		\begin{aligned}
			\| \mathcal{P}_{k} - \mathcal{P}_{N,k} \| \leq & \, c_{1,k}^2 \| \U - \U_N \| +  c_{2,k}^2 \| \G - \G_N \| \\ 
			&+ c_{3,k}^2 C N^{-1/2}+c_{4,k}^2 C N^{-1/2} \\
			& + c_{5,k}^2 \| \hat \mu_{k-1} - \hat \mu_{N, k-1} \| + c_{6,k}^2 \| \mathcal{P}_{k-1} - \mathcal{P}_{N, k-1} \|.
		\end{aligned}
	\end{equation*}
	Por teorema \ref{teo:error_koop_sqrt_N_def} entonces se concluye que con probabilidad $(1-\delta)^4$ existen constantes $C^1_{1, k, \delta}$ y $C^2_{1, k, \delta}$, dependientes de $\delta$, tales que
	\begin{equation*}
		\begin{aligned}
			\| \hat \mu_{k} - \hat \mu_{N,k}  \| \leq & \, C_{1,k, \delta}^1 N^{-1/2} + C_{2,k, \delta}^1 \| \hat \mu_{k-1} - \hat \mu_{N, k-1} \| + C_{3,k, \delta}^1 \| \mathcal{P}_{k-1}  - \mathcal{P}_{N, k-1}  \|,
		\end{aligned}
	\end{equation*}
	\begin{equation*}
		\begin{aligned}
			\| \mathcal{P}_{k} - \mathcal{P}_{N,k}  \| \leq & \, C_{1,k, \delta}^2 N^{-1/2} + C_{2,k, \delta}^2 \| \hat \mu_{k-1} - \hat \mu_{N, k-1} \| + C_{3,k, \delta}^2 \| \mathcal{P}_{k-1} - \mathcal{P}_{N,k-1} \|.
		\end{aligned}
	\end{equation*}
	Para propagar el error hasta la condición inicial se procede por inducción. Para ello primero el caso base $k=1$ que se tiene directo por el teorema \ref{teo:error_kalman} aplicado a $k=1$.\\
	Se supone entonces que para $k \in \N$ se cumple 
	\begin{equation*}
		\begin{aligned}
			\| \hat \mu_{k} - \hat \mu_{N,k}  \| \leq & \, C_{1,k,\delta}^1 N^{-1/2} + C_{2,k,\delta}^1 \| \hat \mu_{0} - \hat \mu_{N, 0} \| + C_{3,k,\delta}^1 \| \mathcal{P}_{0}  - \mathcal{P}_{N, 0}  \|
		\end{aligned}
	\end{equation*}
	\begin{equation*}
		\begin{aligned}
			\| \mathcal{P}_{k} - \mathcal{P}_{N,k}  \| \leq & \, C_{1,k,\delta}^2 N^{-1/2} + C_{2,k,\delta}^2 \| \hat \mu_{0} - \hat \mu_{N, 0} \| + C_{3,k,\delta}^2 \| \mathcal{P}_{0} - \mathcal{P}_{N, 0} \|
		\end{aligned}
	\end{equation*}
	Ahora se prueba para $k+1$, que basta hacerlo para $\| \hat \mu_{k+1} - \hat \mu_{N,k+1}  \|$, para la otra cota análoga. Por el desarrollo anterior, se tiene que
	\begin{equation*}
	\begin{aligned}
		\| \hat \mu_{k+1} - \hat \mu_{N,k+1}  \| \leq & \, C_{1,k+1,\delta}^1 N^{-1/2} + c_{5,k+1}^1 \| \hat \mu_{1, k} - \hat \mu_{2, k} \| + c_{6,k+1}^1 \| \mathcal{P}_{k}  - \mathcal{P}_{N, k}  \| 
	\end{aligned}
	\end{equation*}
	Ocupando la hipótesis inductiva
	\begin{equation*}
	\begin{aligned}
		\leq & \, C_{1,k,\delta}^1 N^{-1/2} \\
		& + c_{5,k+1}^1 (C_{1,k}^1 N^{-1/2} + C_{2,k,\delta}^1 \| \hat \mu_{0} - \hat \mu_{N, 0} \| + C_{3,k,\delta}^1 \| \mathcal{P}_{0}  - \mathcal{P}_{N, 0}  \|) \\
		& + c_{6,k+1}^1 (C_{1,k,\delta}^2 N^{-1/2} + C_{2,k,\delta}^2 \| \hat \mu_{0} - \hat \mu_{N, 0} \| + C_{3,k,\delta}^2 \| \mathcal{P}_{0} - \mathcal{P}_{N, 0} \|) \\
		= & \, C_{1,k+1,\delta}^1 N^{-1/2} + C_{2,k+1,\delta}^1 \| \hat \mu_{0} - \hat \mu_{N, 0} \| + C_{3,k+1,\delta}^1 \| \mathcal{P}_{0} - \mathcal{P}_{N, 0} \|
	\end{aligned}
	\end{equation*}
	Usando el lema \ref{lema:oper_sqrt_N} se obtiene que existe una constante $C_{k+1,\delta}^1$ tal que
	\begin{equation*}
		\begin{aligned}
			\| \hat \mu_{k+1} - \hat \mu_{N,k+1}  \| \leq & \, C_{k+1,\delta}^1 N^{-1/2}
		\end{aligned}
	\end{equation*}
\end{proof}

Ahora se debe estudiar el error inducido por volver al espacio de dimensión original mediante el operador de \textit{lifting back} $\B : (\R^n)^* \to \X$ definido en la sección anterior, que cumple
\begin{equation*}
    \B^* \Phi_\X (\mathbf{x}) = \mathbf{x}.
\end{equation*}
Se propone entonces el estimador exacto del problema como
\begin{equation*}
    \hat{\mathbf{x}}_{k} = \B^* \hat{\mu}_{k}
\end{equation*}
que en la práctica es inaccesible, pero que se aproxima por el estimador
\begin{equation*}
    \hat{\mathbf{x}}_{N, k} = \B^*_{N} \hat{\mu}_{N,k}.
\end{equation*}
Este será el estimador que entregará el algoritmo de filtraje KKF. Además, recordando que la matriz de covarianza de error a posteriori del problema está definida por
\begin{equation*}
    \mathbf{P}_{k} = \E[ (\hat{\mathbf{x}}_k - \mathbf{x}_k) (\hat{\mathbf{x}}_k - \mathbf{x}_k)^\top ],
\end{equation*}
con lo que se obtiene que
\begin{equation*}
    \begin{aligned}
        \mathbf{P}_{k} & = \E[ (\hat{\mathbf{x}}_k - \mathbf{x}_k) (\hat{\mathbf{x}}_k - \mathbf{x}_k)^\top ] \\
        & = \E[ (\B^* \Phi_\X (\hat{\mathbf{x}}_k) - \B^* \Phi_\X (\mathbf{x}_k)) (\B^* \Phi_\X (\hat{\mathbf{x}}_k) - \B^* \Phi_\X (\mathbf{x}_k))^\top ] \\
        & = \B^* \E[ (\Phi_\X (\hat{\mathbf{x}}_k) - \Phi_\X (\mathbf{x}_k)) (\Phi_\X (\hat{\mathbf{x}}_k) -  \Phi_\X (\mathbf{x}_k))^* ] \B \\
        & = \B^* \E[ (\mu_k - \hat{\mu}_{k}) (\mu_k - \hat{\mu}_{k})^*] \B \\
        & = \B^* \mathcal{P}_k \B.
    \end{aligned}
\end{equation*}

En virtud de ello se define la matriz de covarianza de error a posteriori aproximante como
\begin{equation*}
    \mathbf{P}_{N, k} = \B^*_{N} \mathcal{P}_{N, k} \B_{N}
\end{equation*}
Con esto se puede deducir de manera sencilla la cota de error deseada para la aproximación del filtro.

\begin{teo}[Error de KKF]
   Sean $\delta \in (0, 1)$ y $N \in \N$ tales que
\[
\delta > 2 \text{exp} \left ( -\frac{Nc_1^2}{8\|k\|_\infty}\right )
\]
con $c_1$ la constante de inyectividad de $C_X$. Si $\U \H_\X \subset \H_\X$, $\G (\R^p)^* \subset \H_\X$ y $\B (\R^n)^* \subseteq \H_\X$, entonces existen constantes $\Tilde{C}^1_{k,\delta}$, $\Tilde{C}^2_{k,\delta}$ tales que con probabilidad al menos $(1 - \delta)^4$ ocurre que
    \begin{equation*}
        \| \hat{\mathbf{x}}_k - \hat{\mathbf{x}}_{N, k} \| \leq \Tilde{C}^1_{k,\delta} N^{-1/2},
    \end{equation*}
    \begin{equation*}
        \| \mathbf{P}_k - \mathbf{P}_{N, k} \| \leq \Tilde{C}^2_{k,\delta} N^{-1/2}.
    \end{equation*}
    \label{teo:error_KKF_fin}
\end{teo}
\begin{proof}
    Primero para el término asociado a la estimación del estado
    \begin{equation*}
        \begin{aligned}
            \| \hat{\mathbf{x}}_k - \hat{\mathbf{x}}_{N, k} \| & = \| \B^* \hat{\mu}_{k} - \B^*_{N} \hat{\mu}_{N,k} \| \\
            & = \| \B^* \hat{\mu}_{k} - \B^* \hat{\mu}_{N, k} + \B^* \hat{\mu}_{N, k} - \B^*_{N} \hat{\mu}_{N,k} \| \\
            & \leq \| \B^* \hat{\mu}_{k} - \B^* \hat{\mu}_{N, k}\| + \|\B^* \hat{\mu}_{N, k} - \B^*_{N} \hat{\mu}_{N,k} \| \\
            & \leq \| \B^* \| \| \hat{\mu}_{k} -  \hat{\mu}_{N, k}\| + \|\B^* - \B^*_{N}  \| \|\hat{\mu}_{N,k} \|.
        \end{aligned}
    \end{equation*}
    Entonces, por teoremas \ref{teo:error_koop_sqrt_N_def} y \ref{teo:teo_KKF_2} se tiene que con probabilidad $(1-\delta)^4$ existen constantes $C_\delta$, $C_{k,\delta}^1$ tales que 
    \begin{equation*}
         \| \hat{\mathbf{x}}_k - \hat{\mathbf{x}}_{N, k} \| \leq \| \B \| C_{k,\delta}^1 N^{-1/2} + \| \hat{\mu}_{N, k} \| C_\delta N^{-1/2} = \Tilde{C}_{k,\delta}^1 N^{-1/2}
    \end{equation*}
    donde se ha utilizado $\| \B^* \| \leq \| \B \|$, $\| \B^* - \B^*_N \| \leq \| \B - \B_N \|$ y denotado $\Tilde{C}_{k}^1 = \| \B \| C_k^1 + \| \hat{\mu}_{N, k} \| C_\delta N^{-1/2}$. Ahora para la matriz de covarianza de error a posteriori
    \begin{equation*}
        \begin{aligned}
            \| \mathbf{P}_k - \mathbf{P}_{N, k} \| & = \| \B^*\mathcal{P}_k \B - \B^*_N \mathcal{P}_{N, k} \B_N \| \\
            & = \| \B^*\mathcal{P}_k \B - \B^*_N \mathcal{P}_k \B + \B^*_N \mathcal{P}_k \B - \B^*_N \mathcal{P}_{N, k} \B_N \| \\
            & \leq \| \B^*\mathcal{P}_k \B - \B^*_N \mathcal{P}_k \B \| + \| \B^*_N \mathcal{P}_k \B - \B^*_N \mathcal{P}_{N, k} \B_N \| \\
            & \leq \| \B^* - \B^*_N  \| \| \mathcal{P}_k \B \| + \| \B^*_N \| \| \mathcal{P}_k \B - \mathcal{P}_{N, k} \B_N \| \\
            & = \| \B - \B_N  \| \| \mathcal{P}_k \B \| + \| \B_N \| \| \mathcal{P}_k \B -\mathcal{P}_k \B_N + \mathcal{P}_k \B_N - \mathcal{P}_{N, k} \B_N \| \\
            & \leq \| \B - \B_N  \| \| \mathcal{P}_k \B \| + \| \B_N \| \left (\| \mathcal{P}_k \B -\mathcal{P}_k \B_N \| + \| \mathcal{P}_k \B_N - \mathcal{P}_{N, k} \B_N \| \right ) \\
            & \leq \| \B - \B_N  \| \| \mathcal{P}_k \B \| + \| \B_N \| \left ( \| \mathcal{P}_k \B -\mathcal{P}_k \B_N \| + \| \mathcal{P}_k \B_N - \mathcal{P}_{N, k} \B_N \| \right ) \\
            & \leq \| \B - \B_N  \| \| \mathcal{P}_k \B \| + \| \B_N \| \left ( \| \B - \B_N \| \| \mathcal{P}_k \| + \| \mathcal{P}_k - \mathcal{P}_{N, k} \| \| \B_N \| \right ).
        \end{aligned}
    \end{equation*}
    Usando nuevamente los teoremas \ref{teo:error_koop_sqrt_N_def} y \ref{teo:teo_KKF_2} se tiene que probabilidad $(1-\delta)^4$ existen constantes $C_\delta$, $C_{k,\delta}^2$ tales que 
    \begin{equation*}
        \| \mathbf{P}_k - \mathbf{P}_{N, k} \| \leq C_{k,\delta}^2 \| \mathcal{P}_k \B \| N^{-1/2} + \| \B_N \| \left ( C_\delta \| \mathcal{P}_k \| + C_{k,\delta}^2 \| \B_N \| \right ) N^{-1/2}
    \end{equation*}
    y además
    \begin{equation*}
        \| \B_N \| = \| \B_N - \B + \B \| \leq \| \B_N - \B \| + \| \B \| \leq C_\delta N^{-1/2} + \| \B \| 
    \end{equation*}
    con lo que se obtiene
    \begin{equation*}
        \begin{aligned}
            \| \mathbf{P}_k - \mathbf{P}_{N, k} \| & \leq C_{k,\delta}^2 \| \mathcal{P}_k \B \| N^{-1/2} + (C_\delta N^{-1/2} + \| \B \|) \left ( C_\delta \| \mathcal{P}_k \| + C_{k,\delta}^2 (C_\delta N^{-1/2} + \| \B \|) \right ) N^{-1/2} \\
            & \leq \left ( C_{k,\delta}^2 \| \mathcal{P}_k \B \| + (C_\delta N^{-1/2} + \| \B \|) \left ( C_\delta \| \mathcal{P}_k \| + C_{k,\delta}^2 (C_\delta N^{-1/2} + \| \B \|) \right ) \right ) N^{-1/2} \\
            & \leq \left ( C_{k,\delta}^2 \| \mathcal{P}_k \B \| + (C_\delta + \| \B \|) \left ( C_\delta \| \mathcal{P}_k \| + C_{k,\delta}^2 (C_\delta  + \| \B \|) \right ) \right ) N^{-1/2}
        \end{aligned}
    \end{equation*}
    con lo que denotando $\Tilde{C}_{k,\delta}^2 = C_{k,\delta}^2 \| \mathcal{P}_k \B \| + (C_\delta + \| \B \|) \left ( C_\delta \| \mathcal{P}_k \| + C_{k,\delta}^2 (C_\delta + \| \B \|) \right )$ se tiene que 
    \begin{equation*}
        \| \mathbf{P}_k - \mathbf{P}_{N, k} \| \leq \Tilde{C}_{k,\delta}^2 N^{-1/2}
    \end{equation*}
    completando el resultado.
\end{proof}
Con esta cota, en conjunto con otros desarrollos realizados en secciones anteriores, se puede probar el siguiente resultado sobre el sesgo y el error del estimador.

\begin{teo}
    Sean $\delta \in (0, 1)$ y $N \in \N$ tales que
\[
\delta > 2 \text{exp} \left ( -\frac{Nc_1^2}{8\|k\|_\infty}\right )
\]
con $c_1$ la constante de inyectividad de $C_X$. Si $\U \H_\X \subset \H_\X$, $\G (\R^p)^* \subset \H_\X$ y $\B (\R^n)^* \subseteq \H_\X$, entonces existen constantes $\hat{C}^1_{k,\delta}$ y $\hat{C}^2_{k,\delta}$ tales que con probabilidad al menos $(1 - \delta)^4$ ocurre que 
    \begin{enumerate}
        \item El sesgo del estimador $\hat{\mathbf{x}}_{N, k}$, para la trayectoria $\mathbf{x}_k$, está acotado por $\hat{C}^1_{k,\delta} N^{-1/2}$, esto es
    \begin{equation*}
        \left \| \E \left [ \hat{\mathbf{x}}_{N, k} - \mathbf{x}_k \right] \right \| \leq \hat{C}^1_{k,\delta} N^{-1/2}
    \end{equation*}
        \item El estimador $\hat{\mathbf{x}}_{N, k}$ es un $\hat{C}^2_{k,\delta} N^{-1/2}$-mínimo del problema de filtraje, esto es
    \begin{equation*}
        \E \left [ (\hat{\mathbf{x}}_{N, k} - \mathbf{x}_k)^\top (\hat{\mathbf{x}}_{N, k} - \mathbf{x}_k) \right] \leq  \E \left [ (\hat{\mathbf{x}}_{k} - \mathbf{x}_k)^\top (\hat{\mathbf{x}}_{k} - \mathbf{x}_k) \right] + \hat{C}^2_{k,\delta} N^{-1/2}
    \end{equation*}
    \end{enumerate}
\end{teo}
\begin{proof}
    Primero para el punto 1. se tiene que
    \begin{equation*}
        \begin{aligned}
            \| \E [\hat{\mathbf{x}}_{N, k} - \mathbf{x}_k] \| & = \| \E [ \B^*_N \hat{\mu}_{N,k} - \B^*_N \hat{\mu}_{k} + \B^*_N \hat{\mu}_{k} - \B^* \hat{\mu}_k + \B^* \hat{\mu}_k - \B^* \mu_k ] \| \\
            &  = \| \E [ \B^*_N \hat{\mu}_{N,k} - \B^*_N \hat{\mu}_{k} ] + \E [\B^*_N \hat{\mu}_{k} - \B^* \hat{\mu}_k ] + \E [\B^* \hat{\mu}_k - \B^* \mu_k ] \| \\
            & = \| \E [ \B^*_N \hat{\mu}_{N,k} - \B^*_N \hat{\mu}_{k} ] + \E [\B^*_N \hat{\mu}_{k} - \B^* \hat{\mu}_k ] + \B^* \E [ \hat{\mu}_k - \mu_k ] \|.
        \end{aligned}
    \end{equation*}
    Gracias a la proposición \ref{prop:unbias_kalman_operator} se tiene que $\hat{\mu}_k$ es insesgado para $\mu_k$ con lo que
    \begin{equation*}
        \begin{aligned}
            \| \E [\hat{\mathbf{x}}_{N, k} - \mathbf{x}_k] \| 
            & = \| \E [ \B^*_N \hat{\mu}_{N,k} - \B^*_N \hat{\mu}_{k} ] + \E [\B^*_N \hat{\mu}_{k} - \B^* \hat{\mu}_k ] + \B^* \E [ \hat{\mu}_k - \mu_k ] \| \\
            & = \| \E [ \B^*_N \hat{\mu}_{N,k} - \B^*_N \hat{\mu}_{k} ] + \E [\B^*_N \hat{\mu}_{k} - \B^* \hat{\mu}_k ] \| \\
            & \leq  \E [ \| \B^*_N \hat{\mu}_{N,k} - \B^*_N \hat{\mu}_{k} \| ] + \E [\| \B^*_N \hat{\mu}_{k} - \B^* \hat{\mu}_k \|] \\
            & \leq \E [\| \B^*_N \| \| \hat{\mu}_{N,k} - \hat{\mu}_{k}\|] + \E[\| \B^*_N - \B^* \| \| \hat{\mu}_k \|] \\
            & \leq \hat{C}^1_{k,\delta} N^{-1/2}
        \end{aligned}
    \end{equation*}
    donde se han utilizado nuevamente los teoremas \ref{teo:error_koop_sqrt_N_def} y \ref{teo:teo_KKF_2} para decir que dicha constante $\hat{C}^1_{k,\delta}$ existe con probabilidad a lo menos $(1-\delta)^4$. Para probar el punto 2. notar que
    \begin{equation*}
        \E [ (\hat{\mathbf{x}}_{N, k} - \mathbf{x}_k)^\top (\hat{\mathbf{x}}_{N, k} - \mathbf{x}_k) ] = \E [\hat{\mathbf{x}}_{N, k}^\top \hat{\mathbf{x}}_{N, k}] - 2 \E [\hat{\mathbf{x}}_{N, k}^\top \mathbf{x}_k] + \E[\mathbf{x}_k^\top \mathbf{x}_k]
    \end{equation*}
    \begin{equation*}
        \E [ (\hat{\mathbf{x}}_{k} - \mathbf{x}_k)^\top (\hat{\mathbf{x}}_{k} - \mathbf{x}_k) ] = \E [\hat{\mathbf{x}}_{k}^\top \hat{\mathbf{x}}_{k}] - 2 \E [\hat{\mathbf{x}}_{k}^\top \mathbf{x}_k] + \E[\mathbf{x}_k^\top \mathbf{x}_k].
    \end{equation*}
    Restando ambas se tiene 
    \begin{equation*}
        \begin{aligned}
            \E & [ (\hat{\mathbf{x}}_{N, k} - \mathbf{x}_k)^\top (\hat{\mathbf{x}}_{N, k} - \mathbf{x}_k) ] - \E [ (\hat{\mathbf{x}}_{k} - \mathbf{x}_k)^\top (\hat{\mathbf{x}}_{k} - \mathbf{x}_k) ] \\
            & =  \E [\hat{\mathbf{x}}_{N, k}^\top \hat{\mathbf{x}}_{N, k}] - \E [\hat{\mathbf{x}}_{k}^\top \hat{\mathbf{x}}_{k}] + 2 \E [(\hat{\mathbf{x}}_{k} - \hat{\mathbf{x}}_{N, k})^T \mathbf{x}_k] \\
            & = \E [(\hat{\mathbf{x}}_{N, k} - \hat{\mathbf{x}}_{k})^\top (\hat{\mathbf{x}}_{N, k} + \hat{\mathbf{x}}_{k})] + 2 \E [(\hat{\mathbf{x}}_{k} - \hat{\mathbf{x}}_{N, k})^T \mathbf{x}_k]
        \end{aligned}   
    \end{equation*}
    Notar que dado que $\hat{\mathbf{x}}_k$ es óptimo del problema de filtraje, se tiene que 
    \begin{equation*}
        0 \leq \E [ (\hat{\mathbf{x}}_{N, k} - \mathbf{x}_k)^\top (\hat{\mathbf{x}}_{N, k} - \mathbf{x}_k) ] - \E [ (\hat{\mathbf{x}}_{k} - \mathbf{x}_k)^\top (\hat{\mathbf{x}}_{k} - \mathbf{x}_k) ].
    \end{equation*}
    Por lo que, y utilizando Cauchy-Schwarz se tiene
    \begin{equation*}
        \begin{aligned}
            0 & \leq \E [(\hat{\mathbf{x}}_{N, k} - \hat{\mathbf{x}}_{k})^\top (\hat{\mathbf{x}}_{N, k} + \hat{\mathbf{x}}_{k})] + 2 \E [(\hat{\mathbf{x}}_{k} - \hat{\mathbf{x}}_{N, k})^T \mathbf{x}_k] \\
            & \leq \E [\| \hat{\mathbf{x}}_{N, k} - \hat{\mathbf{x}}_k \| \| \hat{\mathbf{x}}_{N, k} + \hat{\mathbf{x}}_k \|] + 2 \E [\| \hat{\mathbf{x}}_{N, k} - \hat{\mathbf{x}}_k \| \| \mathbf{x}_k \| ] \\
            & \leq \E [\| \hat{\mathbf{x}}_{N, k} - \hat{\mathbf{x}}_k \| ( \| \hat{\mathbf{x}}_{N, k} \| + \| \hat{\mathbf{x}}_k \|)] + 2 \E [\| \hat{\mathbf{x}}_{N, k} - \hat{\mathbf{x}}_k \| \| \mathbf{x}_k \| ] \\
            & \leq \hat{C}^2_{k,\delta} N^{-1/2}
        \end{aligned}
    \end{equation*}
    donde se han utilizado nuevamente los teoremas \ref{teo:error_koop_sqrt_N_def} y \ref{teo:teo_KKF_2} para decir que dicha constante $\hat{C}^2_{k,\delta}$ existe con probabilidad a lo menos $(1-\delta)^4$. Con ello
    \begin{equation*}
        \E [ (\hat{\mathbf{x}}_{N, k} - \mathbf{x}_k)^\top (\hat{\mathbf{x}}_{N, k} - \mathbf{x}_k) ] - \E [ (\hat{\mathbf{x}}_{k} - \mathbf{x}_k)^\top (\hat{\mathbf{x}}_{k} - \mathbf{x}_k) ] \leq \hat{C}^2_{k,\delta} N^{-1/2}, \quad \forall k \in \{ 0, \dots, T \},
    \end{equation*}
    denotando
    \[
    \hat{C}^2_\delta := \sup_{k \in \{ 0, \dots,T\} } \hat{C}^2_{k,\delta},
    \]
    se obtiene que
    \begin{equation*}
        \E [ (\hat{\mathbf{x}}_{N, k} - \mathbf{x}_k)^\top (\hat{\mathbf{x}}_{N, k} - \mathbf{x}_k) ] \leq \E [ (\hat{\mathbf{x}}_{k} - \mathbf{x}_k)^\top (\hat{\mathbf{x}}_{k} - \mathbf{x}_k) ] + \hat{C}^2_{k,\delta} N^{-1/2}
    \end{equation*}
    por lo que $\hat{\mathbf{x}}_{N, k}$ es $\hat{C}^2_{\delta} N^{-1/2}$-mínimo del problema de filtraje.
\end{proof}

\section{Consecuencias}

\subsection{Extended Dynamic Mode Decomposition Kalman Filters}

Cabe notar que, en virtud del teorema \ref{teo:error_kalman}, el error del filtro se desglosa en 

\begin{equation}
    \text{Error filtro } \leq \text{ Error EDMD } + \text{ Error aproximando promedios}.
    \label{eq:error_explicacion}
\end{equation}

El primer término depende de la manera de realizar EDMD, siendo kEDMD la que tiene como cota de error $O(N^{-1/2})$. Mientras que el segundo está asociado a aproximar elementos como $\mathcal{P}_0$, $\mathcal{Q}_k$ y $\mathcal{R}_k$, lo que comete un error del orden $O(N^{-1/2})$, característico del Teorema Central del Límite (TCL) y, por tanto, el error que cometen los algoritmos basados en métodos de Monte Carlo. 

Esto, de alguna manera, impone una cota de error base para el filtro si es que se basa en este tipo de métodos. No importan cuán buena sea la cota de error de EDMD si es que no se puede superar esta barrera propia del TCL.

Es por ello que parece ser que kEDMD comete la menor cota de error posible, al nivel del TCL y por tanto el filtro creado, KKF, parece estar dentro de una clase óptima de filtros que alcanzan a esta cota, impuesta de manera natural en un sistema donde los promedios se aproximan a tasa $N^{-1/2}$.

Es más, la metodología de demostración crea algo de manera más directa, dada una forma de realizar Dynamic Mode Decomposition, existe un filtro asociado de manera instantanea, que tendrá un error de acuerdo a \ref{eq:error_explicacion}. Esta clase de filtros que nacen de una manera de hacer EDMD se pueden denominar los \textbf{Extended Dynamic Mode Decomposition Kalman Filters (EDMD-KF)}. Idea que, vista con esta generalidad, no se ha encontrado a día de hoy en literatura. 

\subsection{Metodología de Estimación de Parámetros}

Partiendo de la existencia de un algoritmo de filtraje, es posible desarrollar una metodología para la estimación de parámetros constantes y desconocidos en un sistema dinámico en tiempo discreto. 

La idea detrás de este enfoque radica en que, si el sistema o el algoritmo logra integrar de manera efectiva la premisa de que debe existir un parámetro que se ajuste a las observaciones, entonces con la estabilización proporcionada por el operador de ganancia de Kalman, debería permitir que el parámetro constante se aproxime a su valor real conforme las iteraciones temporales del algoritmo de filtraje tiendan a infinito.

De manera formal, si el sistema tiene $n_p$ parámetros sujetos a estimación, es decir, desconocidos en la práctica, denotados por $\mathbf{p} \in \R^{n_p}$, entonces se agregan como estados que tienen derivada discreta nula, con un ruido aditivo centrado en $0$ y segundo momento finito, esto es
\begin{equation*}
    \mathbf{x}_{k+1}^{n+i} = \mathbf{x}_{k}^{n+1} + \Tilde{\mathbf{w}}_k^i, \quad i = 1, \dots, n_p,
\end{equation*}
este ruido $\mathbf{w}_k^i$ ayudará a la convergencia del parámetro a través de las iteraciones. Lo esperado es que
\begin{equation*}
    \mathbf{x}_{k}^{n+1} \to \mathbf{p}_{i}, \quad k \to \infty
\end{equation*}
y esto se observará en experimentos numéricos.

Este enfoque no es novedoso y ya ha sido explorado en la literatura, por ejemplo en \cite{Deng2013AdaptiveObjects, Jiang2007AEstimation, Kandepu2008ApplyingEstimation}. Sin embargo, en esta sección se propone una reinterpretación de dicha metodología para la estimación de parámetros. Además, se busca establecer una equivalencia con los métodos existentes en otras áreas de la literatura, como aquellos empleados en investigaciones aplicadas en las que se enfrenta incertidumbre sobre un parámetro a partir de observaciones. Un ejemplo destacado de estas técnicas son los algoritmos tipo Markov Chain Monte Carlo (MCMC) \cite{Sammut2010EncyclopediaLearning}.

Como ocurre en muchas aplicaciones, en particular aquellas que serán de interés en este trabajo, la cantidad de observaciones disponibles es finita, y en muchos casos extremadamente limitada. Por esta razón, considerar la posibilidad de que el algoritmo de filtraje pueda converger al parámetro real en el límite cuando las iteraciones temporales tienden a infinito no resulta una opción válida en este contexto.

Por ello, se propone realizar el filtraje de manera iterativa. En este enfoque, se establece un prior como condición inicial, permitiendo que el algoritmo de filtraje complete todas las iteraciones temporales disponibles. Posteriormente, el último elemento producido por el filtro se utiliza como una nueva condición inicial, aprovechando la matriz de covarianza de error en la última iteración como matriz de covarianza para la condición inicial y la dinámica, lo que haría que el mismo método regule su variabilidad.

Eso se puede repetir todas las veces que sean necesarias y debería, en teoría, converger eventualmente al parámetro original. Este comportamiento será evaluado empíricamente en las gráficas y elementos que se presentarán a continuación.

Es más, y haciendo un análogo con el algoritmo de Markov Chain Monte Carlo (MCMC), se puede aprovechar la paralelización de procesos, utilizando todos los núcleos disponibles para generar múltiples filtros que se ejecuten en paralelo. De este modo, se pueden obtener estimaciones distintas para cada uno de los parámetros, lo que potencialmente mejora la precisión y la robustez del proceso de estimación.

Notar que, para cada hilo o subproceso, se genera una serie de tiempos, en los cuales la unidad temporal corresponde al número de iteraciones que se asignen al algoritmo iterativo. En cada uno de estos tiempos se generará una estimación distinta para el parámetro. Esto da lugar, de manera natural, a una distribución de probabilidad que debería concentrarse en torno al parámetro real. A partir de esta distribución, se eliminarán las primeras iteraciones del algoritmo, las cuales se considerarán como iteraciones de \textit{warm-up} o calentamiento, un procedimiento comúnmente utilizado también en el contexto de Markov Chain Monte Carlo y como lo hacen ciertos \textit{frameworks} que implementan este método \cite{Patil2010PyMC:Python, Carpenter2017Stan:Language, Burkner2017Brms:Stan, Abril-Pla2023PyMC:Python}.

\begin{algorithm}[h]
\caption{ParamEstim($\mathbf{x}_0$, $\mathbf{p}_0$, $\mathbf{y}$, $\text{iters}$)}
\label{alg:ParamEstim}
\begin{algorithmic}[1]
\State \textbf{Entrada:} $\mathbf{\mathbf{x}}_0$ prior sobre los estados, $\mathbf{Q}_0$ covarianza a priori sobre los estados, $\mathbf{\mathbf{p}}_0$ prior sobre el parámetro, $\mathbf{\Tilde{P}}_0$ covarianza a priori sobre los estados, $\text{iters}$ la cantidad de iteraciones.
\State \textbf{Salida:} $\mathbf{\hat{p}} \in \R^{(\text{iters}+1)\times n_p}$ estimación de los parámetros en cada iteración.
\State $\hat{\mathbf{x}}_{0,:} \gets \hat{\mathbf{x}}_0$.
\State $\hat{\mathbf{p}}_{0,:} \gets \hat{\mathbf{p}}_0$.
\State Tomar como prior inicial para KKF $$\Tilde{\mathbf{x}} \gets (\hat{\mathbf{x}}_{0}, \hat{\mathbf{p}}_{0,:}).$$
\State Tomar como covarianza inicial para KKF 
\begin{equation*}
    \mathbf{P} = \begin{pmatrix}
        \mathbf{Q}_0 & 0 \\
        0            & \Tilde{\mathbf{P}}_0
    \end{pmatrix}
\end{equation*}
\For{k=1, \dots, \text{iters}} \\
    \quad $\hat{\mathbf{x}}, \, \hat{\mathbf{P}} \gets $ KKF($\Tilde{\mathbf{x}}, \mathbf{P}, \mathbf{y})$. \Comment{$\mathbf{y}$ corresponden a las observaciones.} \\
    \quad $\hat{\mathbf{p}}_{k,:} \gets \hat{\mathbf{x}}_{-1,n_p:}$ \Comment{Última iteración del filtro en las últimas $n_p$ entradas.} \\
    \quad $\Tilde{\mathbf{x}} \gets (\hat{\mathbf{x}}_{0}, \hat{\mathbf{p}}_{k,:})$ \\
    \quad Actualizar la matriz de covarianzas como
    \begin{equation*}
    \mathbf{P} = \begin{pmatrix}
        \mathbf{Q}_0 & 0 \\
        0            & \hat{\mathbf{P}}_{n_p:, n_p:}
    \end{pmatrix}
\end{equation*}
    \quad donde $\hat{\mathbf{P}}_{n_p:, n_p:}$ es la submatriz cuadradada en el último bloque de tamaño $n_p \times n_p$.
\EndFor
\end{algorithmic}
\end{algorithm}

En el algoritmo \ref{alg:ParamEstim} se puede observar el pseudo-código para la estimación de parámetros constantes propuestos y que se verá aplicado en la siguiente sección.

\section{Resultados numéricos}

A continuación se presentan los resultados numéricos para los aspectos presentados del filtro creado. Se implementan además los filtros de Kalman, Extended Kalman, Unscented Kalman y Filtro de Partículas con Importance Resampling, todas son implementaciones propias y están basadas en \cite{Setoodeh2022NonlinearApplications}.

\subsection{Comparación el filtro de Kalman en el caso lineal}

Se compara el KKF con el filtro de Kalman, el cual entrega la solución analítica para el problema de filtraje en el caso en que la dinámica y la observación sean de la forma
\begin{equation*}
    \begin{aligned}
        \mathbf{x}_{k+1} &= \mathbf{A} \mathbf{x}_k + \mathbf{w}_k \\
        \mathbf{y}_k &= \mathbf{C} \mathbf{x}_k + \mathbf{v}_k.
    \end{aligned}
\end{equation*}
Para este experimento se utilizaron las matrices de la forma
\begin{equation*}
    \mathbf{A} =
    \begin{pmatrix}
        1.01 & 0.01 & 0.0 \\
        0.01 & 1.02 & \alpha \\
        0.0 & 0.04 & 1.02
    \end{pmatrix},
    \quad  \mathbf{C} = 
    \begin{pmatrix}
        1 & 0 & 0 \\
        0 & 1 & 0
    \end{pmatrix},
\end{equation*}
es decir, se supone que se observan el primer y el segundo estado, siendo el tercero el que se encuentra completamente oculto. Notar que este sistema cumple ser observable en el sentido usual del control \cite{Trelat2013ControleApplications}, que es que la matriz de observabilidad del sistema
\begin{equation*}
    \mathcal{O} = \begin{pmatrix}
        C \\
        CA \\
        \vdots \\
        CA^{n-1}
    \end{pmatrix},
\end{equation*}
tenga rango completo, donde $n$ es la dimensión del estado, en este caso $n=3$. Se corrobora que el cálculo de la matriz de observabilidad en este caso entrega
\[
\mathcal{O} =
\begin{pmatrix}
1 & 0 & 0 \\
0 & 1 & 0 \\
1.01 & 0.01 & 0 \\
0.01 & 1.02 & \alpha \\
1.0202 & 0.0203 & 0.01\alpha \\
0.0203 & 0.04\alpha + 1.0405 & 2.04\alpha
\end{pmatrix}.
\]
que es una matriz de rango completo al tener columnas linealmente independientes.

En las figuras \ref{fig:kalman_vs_KKF_01}, \ref{fig:kalman_vs_KKF_001} y \ref{fig:kalman_vs_KKF_025} se muestran los resultados del filtro de Kalman y KKF para tres casos de $\alpha$ para la matriz $\mathbf{A}$, comparando con la trayectoria real sin ruido, ni de dinámica ni de observación.

Se utiliza como condición inicial real del sistema $\mathbf{x}_0 = (1,1,1)$ y un prior sobre la condición inicial $\hat{\mathbf{x}}_0$ con media $(0.8, 1.2, 0.9)$ y matriz de covarianza diagonal con entradas $(0.01, 0.01, 0.01)$. Se utilizan como variables aleatorias para la dinámica y observación un ruido normal aditivo, centrado en el origen y con matrices de covarianza
\begin{equation*}
    \mathbf{Q} = \text{diag}(0.01, 0.01, 0.01), \quad \mathbf{R} = \text{diag}(0.01, 0.01),
\end{equation*}
respectivamente.

Se simula el sistema por $30$ unidades de tiempo y para la aproximación de los operadores se utiliza $N=500$. En las tablas \ref{tab:errores_alpha_01}, \ref{tab:errores_alpha_001} y \ref{tab:errores_alpha_m025} se muestran los errores que cometen tanto el filtro de Kalman como KKF con respecto a la curva real, en norma y por cada estado.



\begin{table}[h]
    \caption{Errores del filtro de Kalman y KKF con respecto a la trayectoria real, para distintos valores de $\alpha$.}
    \begin{subtable}{.49\linewidth}
        \centering
    \caption{$\alpha = -0.1$}
    \begin{tabular}{|c|c|c|}
    \hline
    \textbf{Estado} & \textbf{Kalman} & \textbf{KKF} \\ \hline
    Estado 1 & 0.3588 & 0.3909 \\ \hline
    Estado 2 & 0.4347 & 0.4624 \\ \hline
    Estado 3 & 0.4195 & 0.2238 \\ \hline
    \end{tabular}
    \label{tab:errores_alpha_01}
    \end{subtable}
    \begin{subtable}{.49\linewidth}
        \centering
    \caption{$\alpha = 0.01$}
    \begin{tabular}{|c|c|c|}
    \hline
    \textbf{Estado} & \textbf{Kalman} & \textbf{KKF} \\ \hline
    Estado 1 & 0.3588 & 0.3912 \\ \hline
    Estado 2 & 0.4242 & 0.4264 \\ \hline
    Estado 3 & 0.7550 & 0.5928 \\ \hline
    \end{tabular}
    \label{tab:errores_alpha_001}
    \end{subtable}
    \begin{subtable}{\linewidth}
        \centering
        \caption{$\alpha = -0.25$}
        \begin{tabular}{|c|c|c|}
        \hline
        \textbf{Estado} & \textbf{Kalman} & \textbf{KKF} \\ \hline
        Estado 1 & 0.3588 & 0.3906 \\ \hline
        Estado 2 & 0.4592 & 0.5241 \\ \hline
        Estado 3 & 0.3679 & 0.2485 \\ \hline
        \end{tabular}
        \label{tab:errores_alpha_m025}
            \end{subtable}
\end{table}

Con el objetivo de visualizar la cota de error probada en el teorema \ref{teo:error_KKF_fin} se prueba ejecutar el filtro para distintos valores de $N$, manteniendo todo el resto de valores. Se prueba esto para $N \in \{10, 50, 100, 200, 300, 500, 1000, 1500, 2000\}$ y luego se ajusta una curva de la forma $C_1 \cdot N ^{-C_2}$, donde lo esperado es que $C_2$ sea, al menos, $0.5$.

En las figuras \ref{fig:linear_error_01}, \ref{fig:linear_error_001} y \ref{fig:linear_error_025} se muestran los resultados de este experimento, en donde se observa que, en efecto, el orden de decaimiento del error en función de $N$ es cercano a $1/2$, lo que cumple la cota de error vista en el teorema $\ref{teo:error_KKF_fin}$.

\begin{figure}[h!]
    \centering
    \begin{subfigure}[b]{0.49\textwidth}
        \includegraphics[width=\linewidth]{img/content/chapter4/linear_error_01.pdf}
    \caption{$\alpha = -0.1$}
    \label{fig:linear_error_01}
    \end{subfigure}
     \begin{subfigure}[b]{0.49\textwidth}
         \includegraphics[width=\linewidth]{img/content/chapter4/linear_error_001.pdf}
    \caption{$\alpha = 0.01$}
    \label{fig:linear_error_001}
    \end{subfigure}
     \begin{subfigure}[b]{0.49\textwidth}
        \includegraphics[width=\linewidth]{img/content/chapter4/linear_error_025.pdf}
    \caption{$\alpha = -0.25$}
    \label{fig:linear_error_025}
    \end{subfigure}    
    \caption{Error en norma para el estado en función de $N$ para distintos valores de $\alpha$.}
\end{figure}

\subsection{Comparación para modelos epidemiológicos con otros filtros}

Ahora se compara el filtro con otros filtros no lineales presentados en los preliminares, que vendrían siendo Extended Kalman Filter (EKF), Unscented Kalman Filter (UKF) y Particle Filters (PF), que, tal como se mencionó antes, son implementación propia para este trabajo. 

El sistema a considerar es el sistema SIR ya visto anteriormente
\begin{equation*}
    \begin{aligned}
    S_{k+1} &= S_k -\beta S_k I_k, \\
    I_{k+1} &= I_k + \beta S_k I_k - \gamma I_k, \\
    S_{k+1} &= R_k + \gamma I_k,
    \end{aligned}
\end{equation*}
en donde para esta subsección se considerarán $\beta$ y $\gamma$ conocidos y que la función de observación es
\begin{equation*}
    \mathbf{g}(S, I, R, \mathbf{w}) = I + \mathbf{w},
\end{equation*}
es decir, solo se observan los infectados.

Se estudia primero el caso en que se prueban los filtros para $\beta = 1.0$ y $\gamma = 0.3$ y ruidos normales centrados con matrices de covarianza
\begin{equation*}
    \mathbf{Q} = \text{diag}(\sigma, \sigma, \sigma), \quad \mathbf{R} = \text{diag}(\sigma),
\end{equation*}
en donde $\sigma$ variará en cada experimento para poder visualizar cómo se comporta cada filtro en presencia de mayor ruido.

Se utiliza como condición inicial real $\mathbf{x}_0 = (0.9, 0.1, 0.0)$, prior para la condición inicial una variable aleatoria normal centrada en $\hat{\mathbf{x}}_0 = (0.9, 0.05, 0.05)$ con matriz de covarianza $\mathbf{Q}_0 = (0.01, 0.01, 0.01)$. Para el EKF se utilizan diferencias finitas centradas para la aproximación del Jacobiano de la dinámica con precisión $\varepsilon=10^{-6}$, para PF se utilizaron $N_p = 5000$ partículas y para KKF se utilizó $N=1000$ como dimensión de aproximación de los operadores. En las figuras \ref{fig:nonlinear_filters_sir_sigma_01}, \ref{fig:nonlinear_filters_sir_sigma_001} y \ref{fig:nonlinear_filters_sir_sigma_0001} se observan los resultados para $\sigma \in \{0.1, 0.01, 0.001\}$, seguido de las tablas \ref{tab:errores_sigma_01}, \ref{tab:errores_sigma_001} y \ref{tabla:errores_sigma_0001} que contienen el detalle de los errores para cada estado y cada filtro.

\begin{figure}[h!]
    \centering
    \begin{subfigure}[b]{0.49\textwidth}
        \includegraphics[width=0.9\linewidth]{img/content/chapter4/nonlinear_filters_sir_sigma_01.pdf}
    \caption{$\sigma = 0.1$.}
    \label{fig:nonlinear_filters_sir_sigma_01}
    \end{subfigure}
    \begin{subfigure}[b]{0.49\textwidth}
        \includegraphics[width=0.9\linewidth]{img/content/chapter4/nonlinear_filters_sir_sigma_001.pdf}
    \caption{$\sigma = 0.01$.}
    \label{fig:nonlinear_filters_sir_sigma_001}
    \end{subfigure}
    \begin{subfigure}[b]{0.49\textwidth}
        \includegraphics[width=0.9\linewidth]{img/content/chapter4/nonlinear_filters_sir_sigma_0001.pdf}
    \caption{$\sigma = 0.001$.}
    \label{fig:nonlinear_filters_sir_sigma_0001}
    \end{subfigure}
    \caption{Comparación de resultados de las trayectorias generadas por los filtros EKF (naranja), UKF (verde), PF (rojo) y KKF (púrpura), junto con la trayectoria real (puntos azules) sin ruidos, ni de dinámica ni de observación, esto para cada uno de los estados del modelo SIR. $\sigma$ es variable para (a), (b) y (c), mientras que $\beta = 1.0$ y $\gamma = 0.3$ fijos.}
\end{figure}

\begin{table}[h!]
    \caption{Errores de los distintos filtros con respecto a la trayectoria real, para distintos valores de la diagonal de la matriz de covarianza del ruido de dinámica $\sigma$.}
    \begin{subtable}{\linewidth}
        \centering
    \caption{$\sigma = 0.1$}
    \begin{tabular}{|c|c|c|c|c|}
    \hline
    \textbf{Estado} & \textbf{EKF} & \textbf{UKF} & \textbf{PF} & \textbf{KKF} \\ \hline
    S & 0.4757 & 0.7738 & 0.9571 & 0.3330 \\ \hline
    I & 0.8665 & 0.8300 & 0.8553 & 0.3015 \\ \hline
    R & 1.1360 & 1.4221 & 1.2151 & 0.2209 \\ \hline
    \end{tabular}
    \label{tab:errores_sigma_01}
    \end{subtable}
    \begin{subtable}{\linewidth}
        \centering
    \caption{$\sigma = 0.01$}
    \begin{tabular}{|c|c|c|c|c|}
    \hline
    \textbf{Estado} & \textbf{EKF} & \textbf{UKF} & \textbf{PF} & \textbf{KKF} \\ \hline
    S & 0.1277 & 0.1328 & 0.1783 & 0.1737 \\ \hline
    I & 0.2801 & 0.2565 & 0.2779 & 0.1709 \\ \hline
    R & 0.4881 & 0.5135 & 0.4337 & 0.1329 \\ \hline
    \end{tabular}
    \label{tab:errores_sigma_001}
    \end{subtable}
    \begin{subtable}{\linewidth}
        \centering
    \caption{$\sigma = 0.001$}
    \begin{tabular}{|c|c|c|c|c|}
    \hline
    \textbf{Estado} & \textbf{EKF} & \textbf{UKF} & \textbf{PF} & \textbf{KKF} \\ \hline
    S & 0.0417 & 0.0528 & 0.0493 & 0.2320 \\ \hline
    I & 0.1023 & 0.0985 & 0.1022 & 0.1996 \\ \hline
    R & 0.3070 & 0.2489 & 0.2983 & 0.1710 \\ \hline
    \end{tabular}
    \label{tabla:errores_sigma_0001}
    \end{subtable}
\end{table}

Se observa que KKF logra superar en error a todos los otros filtros con creces en el caso en que $\sigma \in \{ 0.1, 0.01\}$, aunque para el caso en donde hay menor ruido los resultados son bastante similares. Esto mostraría que el filtro tiene un mejor desempeño en comparación con los otros filtros en escenarios de mayor incertidumbre.

Se muestra ahora el caso en que $\sigma = 0.01$ fijo y varía el parámetro $\beta \in \{0.6, 0.9, 1.5\}$, en donde el objetivo es ver cómo cambian los resultados en función de $\beta$, que es en algún sentido el parámetro que cuantifica la no linealidad del sistema. Dado que $\gamma$ en realidad representa una relación lineal, se deja fijo en $\gamma = 0.3$. Se utilizan los mismos valores para todo el resto de configuraciones. Los resultados se pueden apreciar en las figuras \ref{fig:nonlinear_filters_sir_beta_06}, \ref{fig:nonlinear_filters_sir_beta_09} y \ref{fig:nonlinear_filters_sir_beta_15}, y en las tablas \ref{tab:errores_beta_gamma_06}, \ref{tab:errores_beta_gamma_09} y \ref{tab:errores_beta_gamma_15}.

Además, se midió el error en norma de los filtros para $50$ valores de $\beta$ entre $0.1$ y $2.5$ equiespaciados, manteniendo todo el resto de parámetros sin cambios. El resultado se puede observar en la figura \ref{fig:nonlinear_filters_sir_error_beta}, en donde KKF tiene un mejor desempeño a nivel general.

\subsection{Estimación de parámetros de modelos epidemiológicos}

A continuación se prueba la metodología para estimación de parámetros, que se compara con Markov Chain Monte Carlo, utilizando como \textit{samplers} Differential Evolution Metropolis (DEMetropolisZ) \cite{terBraak2008DifferentialChains}, que es un \textit{sampler} no basado en gradiente, y No-U-Turn Sampler (NUTS) \cite{Hoffman2014TheCarlo}, que está basado en gradiente y además es de los más utilizados cuando hay un sistema dinámico subyacente implementado en PyMC \cite{Patil2010PyMC:Python}.

Para DEMetropolisZ se utilizarán 20000 iteraciones de \textit{warm up}, es decir, no se utilizarán para la estimación y 20000 para estimación como tal. Mientras que para NUTS se utilizaron 150 y 150, respectivamente. Para la aproximación de Koopman, en los tres sistemas donde se probó, se utilizará $N=100$. Se realizaron 300 iteraciones del algoritmo \ref{alg:ParamEstim}, de las cuales la primera mitad se tomó como \textit{warm up} y la segunda mitad se considerará para la estimación. Para los 3 algoritmos, se hará el mismo procedimiento 8 veces en paralelo.

Se prueba la metodología de estimación de parámetros para el modelo SIR \eqref{eq:SIR}, esto es, lograr una estimación para $\beta$ y $\gamma$ en base a observaciones, tal como se encuentra en el algoritmo \ref{alg:ParamEstim}. Por lo que se considera el modelo con estados aumentados, que es

\begin{equation*}
    \begin{aligned}
        S_{k+1} &= S_k - \beta_k S_k I_k + \mathbf{w}_k^1 \\
        I_{k+1} &= I_k + \beta_k S_k I_k - \gamma_k I_k + \mathbf{w}_k^2 \\
        R_{k+1} &= R_k + \gamma_k I_k + \mathbf{w}_k^3 \\
        \beta_{k+1} &= \beta_k + \mathbf{w}_k^4 \\
        \gamma_{k+1} &= \gamma_k + \mathbf{w}_k^5
    \end{aligned}
\end{equation*}

Se utilizan como parámetros reales $\beta=1.3$ y $\gamma=0.5$, se utiliza como condición inicial para el estado aumentado $\mathbf{x}_0 = (0.9, 0.1, 0.0, 1.3, 0.5)$, y se entrega también como media del prior de la condición inicial $\hat{\mathbf{x}}_0 = (0.9, 0.1, 0.0, 0.1, 0.1)$, con matriz de covarianza inicial
\begin{equation*}
    \mathbf{Q}_0 = \text{diag}(0.001, 0.001, 0.001, 1, 1).
\end{equation*}

Es decir, se entrega como condición inicial para los parámetros $0.1$ y se espera que el algoritmo converja a los parámetros reales. Se considera que los estados observables son $S$ e $I$, es decir
\begin{equation*}
    \mathbf{g}(S, I, R, \beta, \gamma) = (S, I)
\end{equation*}
y se consideran ruidos normales aditivos, centrados y con matrices de covarianza, para la dinámica y la observación, respectivamente
\begin{equation*}
    \mathbf{Q} = \text{diag}(0.001, 0.001, 0.001, 1, 1), \quad \mathbf{R} = \text{diag}(0.001, 0.001).
\end{equation*}

Se simula el sistema a $20$ unidades de tiempo y para muestrear puntos desde el estado se utiliza una variable aleatoria cuyas primeras tres entradas vienen desde una Dirichlet de parámetro $(1,1,1)$ y las otras dos desde Uniformes en $[0.5, 1.5]$ y $[0.1, 0.5]$, respectivamente. 

Se realizan $300$ iteraciones para el procedimiento expuesto en el algoritmo \ref{alg:ParamEstim}. Este procedimiento se hace $8$ veces en paralelo, por lo que hay $8$ filtros, que se denotarán cadenas, que están estimando el parámetro en paralelo. 

En la figura \ref{fig:param_estim_SIR} se muestra el resultado del filtro en la última iteración y la evolución del parámetro estimado en función de las iteraciones, todo esto solo para la primera cadena de manera ilustrativa. En la figura \ref{fig:density_param_estim_SIR} se muestran las densidades de probabilidad generadas por cada cadena, comparadas con el valor real del parámetro. Si estas se promedian, queda una densidad de probabilidad que es mucho más robusta para la estimación del parámetro, resultado que se puede observar en la figura \ref{fig:mean_density_param_estim_SIR}.



\begin{figure}[h!]
    \centering
    \begin{subfigure}[b]{0.8\textwidth}
        \includegraphics[width=\linewidth]{img/content/chapter4/nonlinear_filters_sir_params_density.pdf}
        \caption{}
        \label{fig:density_param_estim_SIR}
    \end{subfigure}
    \begin{subfigure}[b]{0.8\textwidth}
        \includegraphics[width=\linewidth]{img/content/chapter4/nonlinear_filters_sir_params_density_mean.pdf}
        \caption{}
        \label{fig:mean_density_param_estim_SIR}
    \end{subfigure}
    \caption{A la izquierda el resultado para $\beta$ y a la derecha para $\gamma$, parámetros del modelo SIR \eqref{eq:SIR}. En línea punteada vertical se encuentra el valor real del parámetro. \\
    (a) Densidades de probabilidad creadas por cada cadena solo considerando las iteraciones posteriores al \textit{warm up}. Por cada una son $8$ densidades correspondientes a cada cadena. \\
    (b) Densidades de probabilidad resultante de promediar las $8$ densidades creadas por cada cadena.}
    
\end{figure}

Se repite el mismo procedimiento con el modelo SIRS para estimar sus parámetros $\alpha$, $\beta$ y $\gamma$, que con el estado aumentado queda
\begin{equation*}
    \begin{aligned}
        S_{k+1} &= S_k - \beta_k S_k I_k + \alpha_k R_k + \mathbf{w}_k^1 \\
        I_{k+1} &= I_k + \beta_k S_k I_k - \gamma_k I_k + \mathbf{w}_k^2 \\
        R_{k+1} &= R_k + \gamma_k I_k - \alpha_k R_k + \mathbf{w}_k^3 \\
        \alpha_{k+1} &= \alpha_k + \mathbf{w}_k^4 \\
        \beta_{k+1} &= \beta_k + \mathbf{w}_k^5 \\
        \gamma_{k+1} &= \gamma_k + \mathbf{w}_k^6.
    \end{aligned}
\end{equation*}

Para el experimento se utilizaron como valores reales reales $\alpha = 0.2$, $\beta = 1.4$, $\gamma = 0.4$ y condición inicial para el resto de estados $(S_0, I_0, R_0) = (0.9, 0.1, 0.0)$. Se entrega como prior para la condición inicial una normal centrada en $\hat{\mathbf{x}}_k = (0.9, 0.1, 0.0, 0.1, 0.1, 0.1)$ con matriz de covarianza
\[
\mathbf{Q}_0 = \text{diag}(0.01, 0.01, 0.01, 10, 10, 10).
\]

Como función de observación ahora se considera que únicamente es posible observar a los infectados, es decir
\[
\mathbf{g}(S, I, R, \alpha, \beta, \gamma) = I,
\]
por lo que se tiene menos información incluso para este caso que para el anterior, en que se deseaba estimar menos parámetros.

Como distribuciones asociadas a la dinámica y la observación se utiliza un ruido normal aditivo, centrado y con matriz de covarianza y varianza, respectivamente
\[
\mathbf{Q} = \text{diag}(0.001, 0.001, 0.001, 10, 10, 10), \quad \mathbf{R} = 0.01
\]
dado que para la observación es necesaria una variable univariada.

Para muestrear desde el estado se utiliza una variable aleatoria cuyas primeras tres entradas provienen de una Dirichlet de parámetro $(1,1,1)$ y las otras tres de variables aleatorias uniformes en $[0.1, 1.0]$, $[0.1, 0.5]$ y $[0.1, 0.5]$, respectivamente. Se simula el sistema a $20$ unidades temporales.

En la figura \ref{fig:nonlinear_filters_sirs_params} se observa la solución de KKF para el sistema, esto para la última iteración del algoritmo \ref{alg:ParamEstim} y en la primera cadena de las $8$ muestreadas, mientras que en la figura \ref{fig:nonlinear_filters_sirs_params_evolution} se puede ver la evolución del parámetro a través de las iteraciones, esto también para la primera cadena.

Análogo a lo hecho antes, se puede convertir la evolución del parámetro a través de las iteraciones del algoritmo en una densidad de probabilidad, para la cual se consideran solo la segunda mitad de las iteraciones, dejando la primera mitad como \textit{warm up}. Esto se puede ver en la figura \ref{fig:nonlinear_filters_sirs_params_density}, en donde se dejan las $8$ cadenas y en la figura \ref{fig:nonlinear_filters_sirs_params_density_mean} la cadena media para cada uno de los parámetros.

El algoritmo logra acercarse mucho a los parámetros reales de cada sistema, dejando al menos el parámetro real en su intervalo de confianza de nivel $95$\%.

Por último, se prueba para el modelo SEIR, que tiene 4 estados y 4 parámetros por estimar, $\alpha$, $\beta$, $\gamma$ y $\delta$.
\begin{equation*}
    \begin{aligned}
        S_{k+1} &= S_k - \beta_k S_k I_k + \alpha_k R_k + \mathbf{w}_k^1 \\
        E_{k+1} &= E_k + \beta_k S_k I_k - \delta_k E_k + \mathbf{w}_k^2 \\
        I_{k+1} &= I_k + \delta_k E_k - \gamma_k I_k + \mathbf{w}_k^3 \\
        R_{k+1} &= R_k + \gamma_k I_k - \alpha_k R_k + \mathbf{w}_k^4 \\
        \alpha_{k+1} &= \alpha_k + \mathbf{w}_k^5 \\
        \beta_{k+1} &= \beta_k + \mathbf{w}_k^6 \\
        \gamma_{k+1} &= \gamma_k + \mathbf{w}_k^7 \\
        \delta_{k+1} &= \delta_k + \mathbf{w}_k^8.
    \end{aligned}
\end{equation*}
Se simula el sistema a $20$ unidades de tiempo suponiendo que solo se observan los susceptibles y los infectados, es decir,
\begin{equation*}
    g(S, E, I, R, \alpha, \beta, \gamma, \delta) = (S, I).
\end{equation*}

Se puede ver en las figuras \ref{fig:nonlinear_filters_seir_params_density} y \ref{fig:nonlinear_filters_seir_params_density_mean} las densidades de probabilidad que se generan al estimar parámetros. Para poder estudiar el error en las trayectorias, se \textit{samplean} 30 muestras de parámetros desde las densidades de probabilidad generadas, con lo que se obtiene error promedio, desviación estándar para el error, error mínimo y error máximo, esto para cada modelo y cada método, lo que se puede observar en las tablas \ref{tab:metricas_traj_SIR}, \ref{tab:metricas_traj_SIRS} y \ref{tab:metricas_traj_SEIR}, mientras que las trayectorias \textit{sampleadas} se pueden encontrar en las figuras \ref{fig:SIR_traj_params}, \ref{fig:SIRS_traj_params} y \ref{fig:SEIR_traj_params}.

Se observa el rendimiento de los 3 algoritmos, para los 3 sistemas, es bastante similar, solo que KKF entrega intervalos de confianza más pequeños. El aspecto más relevante, por lejos, es el hecho de que KKF es mucho más rápido que DEMetropolisZ y NUTS, siendo incluso casi 10 veces más rápido que NUTS para el modelo SEIR. Esto es un aspecto relevante a considerar dado que NUTS, muy utilizado en muchas aplicaciones de modelos epidemiológicos para estimar parámetros, usualmente demora horas para poder ejecutarse y debe llevarse a supercomputadores. Puede que esta nueva metodología sea más eficiente en este sentido y logre aligerar tareas de computo muy tediosas y disminuir las horas de espera.

\begin{table}[h!]
    \centering
    \caption{Estimación de parámetros del modelo SIR (IC 95\%).} 
    \begin{tabular}{|c|c|c|c|c|}
    \hline
    \textbf{Parámetro} & \textbf{Real} & \textbf{DEMetropolisZ} & \textbf{NUTS} & \textbf{KKF}  \\ \hline
    $\beta$ & 1.3 & 1.59 (0.96, 2.22) & 1.33 (1.31, 1.35) & 1.39 (1.35, 1.42) \\ \hline
    $\gamma$ & 0.5 & 0.69 (0.23, 1.16) & 0.45 (0.44, 0.46) & 0.47 (0.45, 0.49) \\ \hline
    \end{tabular}
    \label{tab:SIR_params}
\end{table}

\begin{table}[h!]
    \centering
    \caption{Estimación de parámetros del modelo SIRS (IC 95\%).} 
    \begin{tabular}{|c|c|c|c|c|}
    \hline
    \textbf{Parámetro}& \textbf{Real} & \textbf{DEMetropolisZ} & \textbf{NUTS} & \textbf{KKF}  \\ \hline
    $\alpha$ & 0.2 & 0.31 (-0.25, 0.88) & 0.17 (0.16, 0.18) & 0.19 (0.17, 0.21) \\ \hline
    $\beta$ & 1.3 & 1.4 (0.88, 1.92) & 1.34 (1.31, 1.36) & 1.3 (1.24, 1.36) \\ \hline
    $\gamma$ & 0.5 & 0.57 (0.22, 0.93) & 0.51 (0.50, 0.52) & 0.44 (0.40, 0.48) \\ \hline
    \end{tabular}
    \label{tab:SIRS_params}
\end{table}

\begin{table}[h!]
    \centering
    \caption{Estimación de parámetros del modelo SEIR (IC 95\%).} 
    \begin{tabular}{|c|c|c|c|c|}
    \hline
    \textbf{Parámetro}& \textbf{Real} & \textbf{DEMetropolisZ} & \textbf{NUTS} & \textbf{KKF}  \\ \hline
    $\alpha$ & 0.2 & 0.92 (0.54, 1.31) & 0.15 (0.13, 0.16) & 0.09 (0.08, 0.10) \\ \hline
    $\beta$ & 1.3 & 2.0 (1.96, 2.03) & 1.27 (1.2, 1.35) & 0.99 (0.95, 1.03) \\  \hline
    $\gamma$ & 0.4 & 0.55 (0.44, 0.66) & 0.47 (0.44, 0.51)
 & 0.37 (0.36, 0.38)\\ \hline
    $\delta$ & 0.5 & 0.37 (0.25, 0.50) & 0.39 (0.37, 0.40) & 0.64 (0.62, 0.66)\\ \hline
    \end{tabular}
    \label{tab:SEIR_params}
\end{table}

\begin{table}[h!]
    \centering
    \begin{tabular}{|c|c|c|c|}
    \hline
    \textbf{Métrica} & \textbf{DEMetropolisZ} & \textbf{NUTS} & \textbf{KKF} \\
    \hline
    Error promedio & 0.730 & 0.170 & 0.190 \\
    \hline
Desviación estándar & 0.369 & 0.015 & 0.023 \\
    \hline
Error mínimo & 0.212 & 0.137 & 0.156 \\
    \hline
Error máximo & 2.029 & 0.203 & 0.240 \\
    \hline
    \end{tabular}
    \caption{Métricas para trayectorias \textit{sampleadas} del modelo SIR.}
    \label{tab:metricas_traj_SIR}
\end{table}

\begin{table}[h!]
    \centering
    \begin{tabular}{|c|c|c|c|}
\hline
\textbf{Métrica} & \textbf{DEMetropolisZ} & \textbf{NUTS} & \textbf{KKF} \\
\hline
Error promedio & 1.150 & 0.160 & 0.260 \\
\hline
Desviación estándar & 0.654 & 0.015 & 0.110 \\
\hline
Error mínimo & 0.450 & 0.131 & 0.030 \\
\hline
Error máximo & 3.590 & 0.192 & 0.511 \\
\hline
\end{tabular}
    \caption{Métricas para trayectorias \textit{sampleadas} del modelo SIRS.}
    \label{tab:metricas_traj_SIRS}
\end{table}

\begin{table}[h!]
    \centering
    \begin{tabular}{|c|c|c|c|}
\hline
\textbf{Métrica} & \textbf{DEMetropolisZ} & \textbf{NUTS} & \textbf{KKF} \\
\hline
Error promedio & 1.120 & 0.190 & 0.240 \\
\hline
Desviación estándar & 0.280 & 0.079 & 0.013 \\
\hline
Error mínimo & 0.612 & 0.082 & 0.215 \\
\hline
Error máximo & 1.749 & 0.464 & 0.269 \\
\hline
\end{tabular}
    \caption{Métricas para trayectorias \textit{sampleadas} del modelo SEIR.}
    \label{tab:metricas_traj_SEIR}
\end{table}

\begin{table}[h!]
    \centering
    \caption{Tiempos de ejecución para los distintos métodos y modelos, para estimación de parámetros, en segundos.} 
    \begin{tabular}{|c|c|c|c|}
    \hline
    \textbf{Modelo} & \textbf{DEMetropolisZ} & \textbf{NUTS} & \textbf{KKF}  \\ \hline
    \textbf{SIR} & 104.1 & 112.00 & 28.3 \\ \hline
    \textbf{SIRS} & 120.6 & 237.7 & 31.5 \\ \hline
    \textbf{SEIR} & 127.8 & 508.6 & 73.6 \\ \hline
    \end{tabular}
    \label{tab:ex_times}
\end{table}

% Capítulo 5: Discusiones y conclusiones
\chapter{Discusiones, conclusiones y trabajo futuro}
Los resultados expuestos en las secciones anteriores se pueden generalizar a tiempo continuo, entendiendo al filtro de Kalman en tiempo continuo como límite de filtros de Kalman en tiempo discreto. Esto ya fue probado en dimensión finita por \cite{Shaid1999TheFilter, Kelly2014Well-posednessTime, Lange2022DerivationCoefficients}.

Se considera la dinámica a tiempo continuo en un horizonte finito de tiempo $t \in [0, T]$
\begin{align*}
    \mathbf{x}'(t) & = \mathbf{f}(\mathbf{x}(t), \mathbf{w}(t)) \\
    \mathbf{y}(t) & = \mathbf{g}(\mathbf{x}(t), \mathbf{v}(t))
\end{align*}
en donde $\mathbf{w}: \R \to \R^{n_\mathbf{w}}$, $\mathbf{v}: \R \to \R^{n_\mathbf{v}}$ son variables aleatorias tales que $\E[\mathbf{w}(t)] = 0$, $\E[\mathbf{v}(t)] = 0$ y tienen segundo momento finito en el sentido de que
\begin{equation*}
    \E[ \mathbf{w}(t)^\top \mathbf{w}(t) ] = \mathbf{Q}(t), \quad \E[ \mathbf{v}(t)^\top \mathbf{v}(t) ] = \mathbf{R}(t)
\end{equation*}
para ciertas funciones a valores matriciales $\mathbf{Q}: \R \to \R^{n_\mathbf{w} \times n_\mathbf{w}}$, $\mathbf{R}: \R \to \R^{n_\mathbf{v} \times n_\mathbf{v}}$.

Se consideran las siguientes dinámicas discretas, parametrizadas por el paso de tiempo $\Delta t$
\begin{align*}
    \mathbf{x}_{k+1} & = \Tilde{\mathbf{f}} (\mathbf{x}_k, \mathbf{w}_k) := \mathbf{x}_k + \Delta t \cdot \mathbf{f}(\mathbf{x}_k, \mathbf{w}_k )\\
    \mathbf{y}_k & = \mathbf{g}(\mathbf{x}_k, \mathbf{v}_k)
\end{align*}
en donde 
\begin{equation*}
    t_k = k\Delta t, \,  \mathbf{x}_k := \mathbf{x}(t_k), \,  \mathbf{y}_k := \mathbf{y}(t_k), \, \mathbf{w}_k := \mathbf{w}(t_k), \, \mathbf{v}_k := \mathbf{v}(t_k).
\end{equation*}

Se recuerda de secciones anteriores entonces el operador de Koopman asociado a $\Tilde{\mathbf{f}}$, que en este contexto quedará parametrizado por $\Delta t$,
\[
(\U_{\Delta t} \psi)(\mathbf{x}) = \E [\psi (\Tilde{\mathbf{f}} (\mathbf{x}, \cdot))] = \E [\psi (\mathbf{x} + \Delta t \cdot \mathbf{f} (\mathbf{x}, \cdot))].
\]
Esta definición motiva la del operador de Koopman en tiempo continuo.

\begin{defn}[Operador de Koopman estocástico, tiempo continuo]
    Se define el operador $\Tilde{\U}_{\Delta t}: \H_\X \to \H_\X$ mediante
    \[
    (\Tilde{\U}_{\Delta t} \psi)(\mathbf{x}_0) = \E \left [\psi \left ( \mathbf{x}_0 + \int_0^{\Delta t} \mathbf{f}(\mathbf{x}(s), \mathbf{w}(s)) ds \right ) \right ]
    \]
\end{defn}

Intuitivamente, este operador desplaza la dinámica continua en un tiempo $\Delta t$. 

\begin{prop}[Phillip et al. \cite{Philipp2024ErrorOperator}]
    $(\Tilde{\U}_{\Delta t})_{\Delta t \geq 0}$ es un $C_0$-semigrupo cuyo generador
    \[
    \Tilde{\mathcal{L}} \psi = \lim_{\Delta t \searrow 0} \frac{\Tilde{\U}_{\Delta t}\psi - \psi}{\Delta t}
    \]
    está bien definido en un denso de $\H_\X$, es cerrado y disipativo.
\end{prop}

\begin{lema}
    Sea $k$ el kernel de Matérn de parámetro $\nu = 1/2$ y ancho de banda $\gamma$, es decir,
    \[
    k(x, y) = \text{exp} \left ( -\frac{\| x- y \|}{\gamma} \right ).
    \]
    Sea $\H_\X$ su espacio de Hilbert asociado, luego si $\psi \in \H_\X$ se tiene que $\psi$ es $1/2$-Hölder de constante $\sqrt{2/\gamma} \| \psi \|_{\H_\X}$, esto es,
    \[
    |\psi(x) - \psi(y)| \leq \sqrt{\frac{2}{\gamma}} \| \psi \|_{\H_\X} \sqrt{\| x - y\|}, \quad \forall x, y \in \X.
    \]
    \label{lemma:matern_lipschitz}
\end{lema}
\begin{proof}
    Sean $\psi \in \H_\X$, $x, y \in \X$. Primero, por propiedad reproduciente notar que se tiene
    \[
    | \psi(x) - \psi(y) |^2 = | \langle k(x, \cdot), \psi \rangle - \langle k(y, \cdot), \psi \rangle  |^2.
    \]
    Luego, por bilinealidad del producto interno
    \[
    | \psi(x) - \psi(y) |^2 = | \langle k(x, \cdot) - k(y, \cdot), \psi \rangle  |^2.
    \]
    Por desigualdad de Cauchy-Schwarz se tiene
    \[
    \begin{aligned}
        | \psi(x) - \psi(y) |^2 &\leq \| \psi \|^2_{\H_\X} \| k(x, \cdot) - k(y, \cdot) \|^2_{\H_\X} \\
        &\leq \| \psi \|^2_{\H_\X} \langle k(x, \cdot) - k(y, \cdot), k(x, \cdot) - k(y, \cdot) \rangle \\
        &\leq \| \psi \|^2_{\H_\X} ( k(x, x) + k(y, y) - 2k(x,y) ).
    \end{aligned}
    \]
    Dado que 
    \[
    k(x, y) = \text{exp} \left ( -\frac{\| x- y \|}{\gamma} \right ),
    \]
    se tiene que $k(x,x) = k(y,y) = 1$, con lo que
    \[
    | \psi(x) - \psi(y) |^2 \leq 2 \| \psi \|^2_{\H_\X} \left ( 1 - \text{exp} \left ( -\frac{\| x- y \|}{\gamma} \right ) \right ).
    \]
    Utilizando que $1 - e^{-x} \leq x$ para $x \geq 0$, se llega a
    \[
    | \psi(x) - \psi(y) |^2 \leq \frac{2}{\gamma} \| \psi \|^2_{\H_\X} \| x - y \|,
    \]
    lo que implica que
    \[
    | \psi(x) - \psi(y) | \leq \sqrt{\frac{2}{\gamma}} \| \psi \|_{\H_\X} \sqrt{\| x - y \|}
    \]
    obteniendo el resultado.
\end{proof}

Esto tendrá como consecuencia una cota de error para el error que se comete al aproximar el operador de Koopman continuo por uno de dinámica discreta. 

\begin{prop}
    Bajo las hipótesis del lema \ref{lemma:matern_lipschitz}, se tiene que
    \[
    \| \Tilde{\U}_{\Delta t} - \U_{\Delta t} \| \leq 2 \sqrt{\frac{\| \mathbf{f} \|_\infty \Delta t }{\gamma}} 
    \]
    \label{prop:aprox_koop_cont_disc}
\end{prop}

\begin{proof}
    Sean $\psi \in \H_\X$, $\mathbf{x}_0 \in \X$, entonces
    \[
    \begin{aligned}
        \| (\Tilde{\U}_{\Delta t}\psi)(\mathbf{x}_0) - (\U_{\Delta t} \psi)(\mathbf{x}_0) \| &= \left  \| \E \left [\psi \left ( \mathbf{x}_0 + \int_0^{\Delta t} \mathbf{f}(\mathbf{x}(s), \mathbf{w}(s)) ds \right ) \right ] - \E [\psi(\mathbf{x}_0 + \Delta t \cdot \mathbf{f}(\mathbf{x}(t), \mathbf{w}(t)))] \right \| \\
        &\leq \E \left [ \left \| \psi \left ( \mathbf{x}_0 + \int_0^{\Delta t} \mathbf{f}(\mathbf{x}(s), \mathbf{w}(s)) ds \right ) - \psi(\mathbf{x}_0 + \Delta t \cdot \mathbf{f}(\mathbf{x}(t), \mathbf{w}(t))) \right \| \right ] \\
        &\leq \E \left [ \sqrt{\frac{2}{\gamma}} \| \psi \|_{\H_\X} \sqrt{ \left \| \mathbf{x}_0 + \int_0^{\Delta t} \mathbf{f}(\mathbf{x}(s), \mathbf{w}(s)) ds - \mathbf{x}_0 - \Delta t \cdot \mathbf{f}(\mathbf{x}(t), \mathbf{w}(t)))\right \|} \right ] \\
        &= \E \left [ \sqrt{\frac{2}{\gamma}} \| \psi \|_{\H_\X} \sqrt{ \left \| \int_0^{\Delta t} \mathbf{f}(\mathbf{x}(s), \mathbf{w}(s)) ds - \Delta t \cdot \mathbf{f}(\mathbf{x}(t), \mathbf{w}(t)))\right \|} \right ] \\
        &= \E \left [ \sqrt{\frac{2}{\gamma}} \| \psi \|_{\H_\X} \sqrt{ \left \| \int_0^{\Delta t} \mathbf{f}(\mathbf{x}(s), \mathbf{w}(s)) ds - \Delta t \cdot \mathbf{f}(\mathbf{x}(t), \mathbf{w}(t)) \right \|} \right ] \\
        &= \E \left [ \sqrt{\frac{2}{\gamma}} \| \psi \|_{\H_\X} \sqrt{ \left \| \int_0^{\Delta t} \mathbf{f}(\mathbf{x}(s), \mathbf{w}(s)) ds - \int_0^{\Delta t} \mathbf{f}(\mathbf{x}(t), \mathbf{w}(t)) ds \right \|} \right ] \\
        &\leq \E \left [ \sqrt{\frac{2}{\gamma}} \| \psi \|_{\H_\X} \sqrt{ 2 \| \mathbf{f} \|_\infty \Delta t } \right ] \\
        &= 2 \sqrt{\frac{\| \mathbf{f} \|_\infty \Delta t}{\gamma}} \| \psi \|_{\H_\X}  
     \end{aligned}
    \]
    Al ser esta cota uniforme en $\mathbf{x}_0$ se concluye que
    \[
    \| \Tilde{\U}_{\Delta t}\psi - \U_{\Delta t} \psi \| \leq 2 \sqrt{\frac{\| \mathbf{f} \|_\infty \Delta t}{\gamma}} \| \psi \|_{\H_\X}\] 
    con lo que
    \[
    \| \Tilde{\U}_{\Delta t} - \U_{\Delta t} \| \leq 2 \sqrt{\frac{\| \mathbf{f} \|_\infty \Delta t}{\gamma}}. \] 
\end{proof}

Primero, se deduce la dinámica continua del embedding
\[
\mu_{k+1} = \U^*_{\Delta t} \mu_k + \zeta_k = \E [X^+ | X = \mathbf{x}_k] + \Phi_\X (\mathbf{x}_{k+1}) - \E [X^+ | X = \mathbf{x}_k]  = \Phi_\X (\mathbf{x}_{k+1})
\]
con lo que
\[
\frac{\mu_{k+1} - \mu_k}{\Delta t} = \frac{\U^*_{\Delta t} \mu_k - \mu_k + \zeta_k}{\Delta t} = \frac{\Tilde{\U}^*_{\Delta t} \mu_k - \mu_k + \Tilde{\U}^*_{\Delta t} \mu_k - \U^*_{\Delta t} \mu_k + \zeta_k}{\Delta t}
\]
haciendo $\Delta t \searrow 0$ se obtiene
\[
\mu'(t) = \Tilde{\mathcal{L}}^* \mu(t) + \Tilde{\zeta} (t)
\]
con $\mu(t) = \Phi_\X (\mathbf{x}(t))$ y $\Tilde{\zeta}(t)$ algún proceso centrado con operador de covarianza $\mathcal{Q}(t)$.
Entonces se genera el siguiente sistema a tiempo continuo
\begin{equation*}
    \begin{aligned}
        \mu'(t) &= \Tilde{\mathcal{L}}^* \mu(t) + \Tilde{\zeta}(t) \\
        \mathbf{y}(t) &= \mathcal{G}^* \mu(t) + \nu (t)
    \end{aligned}
\end{equation*}
siendo $\nu$ un proceso centrado con función de covarianza $\mathcal{R}:[0, T] \to \R^{p \times p}$ que se supondrá invertible.

Luego, tiene el siguiente filtro de Kalman-Bucy asociado

\begin{equation*}
    \hat{\mu}'(t) = \Tilde{\mathcal{L}}^*\hat{\mu}(t)  + \K(t) \big( \mathbf{y}(t) - \mathcal{G}^*\hat{\mu}(t) \big),
\end{equation*}
\begin{equation*}
    \mathcal{P}'(t) = \Tilde{\mathcal{L}}^*\mathcal{P}(t) + \mathcal{P}(t)\Tilde{\mathcal{L}} - \mathcal{P}(t)\mathcal{G}\mathcal{R}^{-1}\mathcal{G}^*\mathcal{P}(t) + \mathcal{Q}(t).
\end{equation*}
\begin{equation*}
    \mathcal{K}(t) = \mathcal{P}(t)\mathcal{G}\mathcal{R}^{-1}(t).
\end{equation*}

Para $\Delta t > 0$ se propone aproximar la solución por la solución del siguiente sistema

\begin{equation*}
    \hat{\mu}'_{N, \Delta t}(t) = \Tilde{\mathcal{L}}_{N, \Delta t}^* \hat{\mu}_{N}(t)  + \K_{N, \Delta t}(t) \big( \mathbf{y}(t) - \mathcal{G}_N^*\hat{\mu}_{N, \Delta t}(t) \big),
\end{equation*}
\begin{equation*}
    \mathcal{P}_{N, \Delta t}'(t) = \Tilde{\mathcal{L}}_{N, \Delta t}^*\mathcal{P}_{N, \Delta t}(t) + \mathcal{P}_{N, \Delta t}(t)\Tilde{\mathcal{L}}_{N, \Delta t} - \mathcal{P}_{N, \Delta t}(t)\mathcal{G}_N\mathcal{R}_N^{-1}\mathcal{G}_N^*\mathcal{P}_{N, \Delta t}(t) + \mathcal{Q}_N(t).
\end{equation*}
\begin{equation*}
    \mathcal{K}_{N, \Delta t}(t) = \mathcal{P}_{N, \Delta t}(t)\mathcal{G}_N\mathcal{R}_N^{-1}(t).
\end{equation*}
\begin{equation*}
    \Tilde{\mathcal{L}}_{N, \Delta t} = \frac{\U_N - I}{\Delta t}
\end{equation*}

\begin{prop}
    Bajo las hipótesis del lema y del teorema \ref{teo:error_koop}, tomando $\gamma = N^{3/4}$, $\Delta t = N^{-1/4}$, se tiene que, para $\delta \in (0, 1)$, con probabilidad $1-\delta$ existe una constante $\Tilde{C}$ tal que
    \[
    \| \Tilde{\mathcal{L}}_{N, \Delta t} - \Tilde{\mathcal{L}}\| \leq \Tilde{C} N^{-1/4}
    \]
\end{prop}
\begin{proof}
    La demostración pasa por 3 aproximaciones intermedias
    \[
    \begin{aligned}
        \| \Tilde{\mathcal{L}} - \Tilde{\mathcal{L}}_{N, \Delta t}\| &= \left \| \Tilde{\mathcal{L}} - \left ( \frac{\Tilde{\U}_{\Delta t} - I}{\Delta t} \right ) + \left ( \frac{\Tilde{\U}_{\Delta t} - I}{\Delta t} \right ) - \left ( \frac{\U_{\Delta t} - I}{\Delta t} \right ) + \left ( \frac{\U_{\Delta t} - I}{\Delta t} \right ) - \left ( \frac{\U_N - I}{\Delta t} \right )  \right \| \\
        &\leq \left \| \Tilde{\mathcal{L}} - \left ( \frac{\Tilde{\U}_{\Delta t} - I}{\Delta t} \right ) \right \| + \left \| \left ( \frac{\Tilde{\U}_{\Delta t} - I}{\Delta t} \right ) - \left ( \frac{\U_{\Delta t} - I}{\Delta t} \right ) \right \| + \left \| \left ( \frac{\U_{\Delta t} - I}{\Delta t} \right ) - \left ( \frac{\U_N - I}{\Delta t} \right )  \right \| \\
        & \leq C \Delta t + \frac{1}{\Delta t} \left \|  \Tilde{\U}_{\Delta t} - \U_{\Delta t}  \right \| + \frac{1}{\Delta t} \left \| \U_{\Delta t} - \U_N \right \|.
    \end{aligned}
    \]
    Aplicando la proposición \ref{prop:aprox_koop_cont_disc} se tiene
        \[
    \begin{aligned}
        \| \Tilde{\mathcal{L}} - \Tilde{\mathcal{L}}_{N, \Delta t}\|
        & \leq C \Delta t + \frac{1}{\Delta t} \sqrt{\frac{\| \mathbf{f} \|_\infty \Delta t}{\gamma}} + \frac{1}{\Delta t} C N^{-1/2} \\
        & = C \Delta t + \frac{1}{\sqrt{\Delta t}} \sqrt{\frac{\| \mathbf{f} \|_\infty}{\gamma}} + \frac{1}{\Delta t} C N^{-1/2}.
    \end{aligned}
    \]
    Utilizando $\gamma = N^{3/4}$ y $\Delta t = N^{-1/4}$ se obtiene
    \[
    \begin{aligned}
        \| \Tilde{\mathcal{L}} - \Tilde{\mathcal{L}}_{N, \Delta t}\|
        & = C N^{-1/4} + \frac{1}{\sqrt{N^{-1/4}}} \sqrt{\frac{\| \mathbf{f} \|_\infty}{N^{3/4}}} + \frac{1}{N^{-1/4}} C N^{-1/2} \\
        & \leq C N^{-1/4} + \sqrt{\frac{\| \mathbf{f} \|_\infty}{N^{1/2}}} + C N^{-1/4} \\
        & \leq C N^{-1/4} + \sqrt{\| \mathbf{f} \|_\infty} N^{-1/4} + C N^{-1/4} \\
        & = \Tilde{C} N^{-1/4}
    \end{aligned}
    \]
\end{proof}

Entonces, se puede ver que
\[
\begin{aligned}
    \| \mathcal{P}_{N, \Delta t}(t) - \mathcal{P}(t) \| &= \left \| \mathcal{P}_{N, \Delta t}(0) - \mathcal{P}(0) + \int_0^t (\mathcal{P}'_{N, \Delta t}(s) - \mathcal{P}'(s))ds \right \| \\
    &\leq \| \mathcal{P}_{N, \Delta t}(0) - \mathcal{P}(0) \| + \int_0^t \| \mathcal{P}'_{N, \Delta t}(s) - \mathcal{P}'(s) \| ds \\
    &\leq C_1 N^{-1/2} + \int_0^t C_2(s) N^{-1/2} \|  \mathcal{P}_{N, \Delta t}(s) - \mathcal{P}(s) \| ds
\end{aligned}
\]
entonces por Gronwall-Bellman
\[
    \| \mathcal{P}_{N, \Delta t}(t) - \mathcal{P}(t) \| \leq C_1 N^{-1/2} \text{exp} \left ( N^{-1/4} \int_0^t C_2 (s) ds \right )
\]
con esta cota y haciendo lo análogo para $\hat{\mu}(t)$ se obtiene
\[
    \| \hat{\mu}_{N, \Delta t}(t) - \hat{\mu}(t) \| \leq C_3 N^{-1/2} \text{exp} \left ( N^{-1/4} \int_0^t C_4 (s) ds \right )
\]

\begin{appendixd}
    \chapter{Resultados numéricos capítulo 3}

\section{kEDMD para modelos epidemiológicos}

\begin{figure}[htbp]
    \centering
    \begin{subfigure}[b]{0.45\textwidth}
        \centering
        \includegraphics[width=\textwidth]{img/content/chapter3/SIR1.pdf}
        \caption{$\beta=1$, $\gamma = 0.3$}
        \label{fig:SIR1}
    \end{subfigure}
    \hfill
    \begin{subfigure}[b]{0.45\textwidth}
        \centering
        \includegraphics[width=\textwidth]{img/content/chapter3/SIR2.pdf}
        \caption{$\beta=0.5$, $\gamma = 0.1$}
        \label{fig:SIR2}
    \end{subfigure}
    \hfill
    \begin{subfigure}[b]{0.45\textwidth}
        \centering
        \includegraphics[width=\textwidth]{img/content/chapter3/SIR3.pdf}
        \caption{$\beta=1.5$, $\gamma = 0.6$}
    \end{subfigure}
    \caption{Ilustración de los tres casos de $\beta$ y $\gamma$ elegidos para la comparación entre el sistema SIR original y el sistema linealizado por Koopman a 200 puntos \textit{sampleados} de una variable aleatoria Dirichlet. En forma de puntos se dejan los valores reales que toma el sistema y en línea continua los valores entregados por el sistema linealizado, que se consideran como predicción.}
    \label{fig:Comp_traj_SIR}
\end{figure}

\begin{figure}[h]
    \centering
    \begin{subfigure}[b]{0.32\textwidth}
        \centering
        \includegraphics[width=\textwidth]{img/content/chapter3/SIR1Errors.pdf}
        \caption{$\beta=1.0$, $\gamma=0.3$}
    \end{subfigure}
    \hfill
    \begin{subfigure}[b]{0.32\textwidth}
        \centering
        \includegraphics[width=\textwidth]{img/content/chapter3/SIR2Errors.pdf}
        \caption{$\beta=0.5$, $\gamma=0.1$}
    \end{subfigure}
    \hfill
    \begin{subfigure}[b]{0.32\textwidth}
        \centering
        \includegraphics[width=\textwidth]{img/content/chapter3/SIR3Errors.pdf}
        \caption{$\beta=1.5$, $\gamma=0.6$}
    \end{subfigure}
    \caption{Ilustración de los tres casos del modelo SIR elegido para la evolución en función de $N$ de la diferencia en norma entre el sistema lineal original y el sistema linealizado por Koopman a $N$ puntos,  \textit{sampleados} de una variable aleatoria normal. En forma de puntos se deja la evolución observada del error y en línea continua la mejor curva de la forma $C \cdot N^{a}$, donde $a$ es el exponente que se deja en la leyenda.}
    \label{fig:ErrorSIR}
\end{figure}

\begin{figure}[h]
    \centering
    \begin{subfigure}[b]{0.45\textwidth}
        \centering
        \includegraphics[width=\textwidth]{img/content/chapter3/SIRS1.pdf}
        \caption{$\alpha=0.1$, $\beta = 1.0$, $\gamma = 0.3$.}
        \label{fig:SIRS1}
    \end{subfigure}
    \hfill
    \begin{subfigure}[b]{0.45\textwidth}
        \centering
        \includegraphics[width=\textwidth]{img/content/chapter3/SIRS2.pdf}
        \caption{$\alpha=0.1$, $\beta = 1.0$, $\gamma = 0.3$.}
        \label{fig:SIRS2}
    \end{subfigure}
    \hfill
    \begin{subfigure}[b]{0.45\textwidth}
        \centering
        \includegraphics[width=\textwidth]{img/content/chapter3/SIRS3.pdf}
        \caption{$\alpha=0.05$}
    \end{subfigure}
    \caption{Ilustración de los tres casos de $\alpha$, $\beta$ y $\gamma$ elegidos para la comparación entre el sistema SIRS y el sistema linealizado por Koopman a 1000 puntos \textit{sampleados} de una variable aleatoria Dirichlet. En forma de puntos se dejan los valores reales que toma el sistema y en línea continua los valores entregados por el sistema linealizado, que se consideran como predicción.}
    \label{fig:Comp_traj_SIRS}
\end{figure}
\begin{figure}[h]
    \centering
    \begin{subfigure}[b]{0.32\textwidth}
        \centering
        \includegraphics[width=\textwidth]{img/content/chapter3/SIRS1Errors.pdf}
        \caption{$\alpha=0.1$, $\beta=1$, $\gamma=0.3$}
    \end{subfigure}
    \hfill
    \begin{subfigure}[b]{0.32\textwidth}
        \centering
        \includegraphics[width=\textwidth]{img/content/chapter3/SIRS2Errors.pdf}
        \caption{$\alpha=0.3$, $\beta=1$, $\gamma=0.3$}
    \end{subfigure}
    \hfill
    \begin{subfigure}[b]{0.32\textwidth}
        \centering
        \includegraphics[width=\textwidth]{img/content/chapter3/SIRS3Errors.pdf}
        \caption{$\alpha=0.05$}
    \end{subfigure}
    \caption{Ilustración de los tres casos del modelo SIRS elegido para la evolución en función de $N$ de la diferencia en norma entre el sistema lineal original y el sistema linealizado por Koopman a $N$ puntos,  \textit{sampleados} de una variable aleatoria normal. En forma de puntos se deja la evolución observada del error y en línea continua la mejor curva de la forma $C \cdot N^{a}$, donde $a$ es el exponente que se deja en la leyenda.}
    \label{fig:ErrorSIRS}
\end{figure}


\chapter{Resultados numéricos capítulo 4}

\section{Comparación con el Filtro de Kalman}

\begin{figure}[h!]
    \centering
    \begin{subfigure}[b]{0.49\textwidth}
        \centering
        \includegraphics[width=0.75\linewidth]{img/content/chapter4/kalman_kkkf_01.pdf}
    \caption{$\alpha = -0.1$.}
    \label{fig:kalman_vs_KKF_01}
    \end{subfigure}
    \begin{subfigure}[b]{0.49\textwidth}
        \centering \includegraphics[width=0.75\linewidth]{img/content/chapter4/kalman_kkkf_001.pdf}
    \caption{$\alpha = 0.01$.}
    \label{fig:kalman_vs_KKF_001}
    \end{subfigure}
    \begin{subfigure}[b]{0.49\textwidth}
        \centering \includegraphics[width=0.75\linewidth]{img/content/chapter4/kalman_kkkf_025.pdf}
    \caption{$\alpha = -0.25$.}
    \label{fig:kalman_vs_KKF_025}
    \end{subfigure}
    \caption{Comparación de Kalman con KKF para distintos valores de $\alpha$. En azul el resultado del sistema simulado sin ruidos ni de dinámica ni observación, en naranja la solución del filtro de Kalman y en verde la de KKF.}
\end{figure}

\section{Comparación con otros filtros para modelos epidemiológicos}

\begin{figure}[h!]
    \centering
    \begin{subfigure}[b]{0.49\textwidth}
        \includegraphics[width=\linewidth]{img/content/chapter4/nonlinear_filters_sir_beta_06.pdf}
    \caption{$\beta = 0.6$.}
    \label{fig:nonlinear_filters_sir_beta_06}
    \end{subfigure}
    \begin{subfigure}[b]{0.49\textwidth}
        \includegraphics[width=\linewidth]{img/content/chapter4/nonlinear_filters_sir_beta_09.pdf}
    \caption{$\beta = 0.9$.}
    \label{fig:nonlinear_filters_sir_beta_09}
    \end{subfigure}
    \begin{subfigure}[b]{0.49\textwidth}
        \includegraphics[width=\linewidth]{img/content/chapter4/nonlinear_filters_sir_beta_15.pdf}
    \caption{$\beta = 1.5$.}
    \label{fig:nonlinear_filters_sir_beta_15}
    \end{subfigure}
    \caption{Comparación de resultados de las trayectorias generadas por los filtros EKF (naranja), UKF (verde), PF (rojo) y KKF (púrpura), junto con la trayectoria real (puntos azules) sin ruidos, ni de dinámica ni de observación, esto para cada uno de los estados del modelo SIR. $\beta$ varía en cada caso para (a), (b) y (c), mientras que $\gamma = 0.3$ y $\sigma = 0.01$ fijos.}
\end{figure}

\begin{table}[h!]
    \caption{Errores para distintos valores de $\beta$, parámetro que representa la no linealidad del sistema. Esto para $\gamma = 0.3$ y $\sigma = 0.01$ fijos.}
    \begin{subtable}{\linewidth}
        \centering
    \caption{Errores para $\beta = 0.6$ y $\gamma = 0.3$}
    \begin{tabular}{|c|c|c|c|c|}
    \hline
    \textbf{Estado} & \textbf{EKF} & \textbf{UKF} & \textbf{PF} & \textbf{KKF} \\ \hline
    S & 0.2442 & 0.2379 & 0.3305 & 0.1629 \\ \hline
    I & 0.2904 & 0.2815 & 0.2903 & 0.1352 \\ \hline
    R & 0.4874 & 0.5503 & 0.4732 & 0.1210 \\ \hline
    \end{tabular}
    \label{tab:errores_beta_gamma_06}
    \end{subtable}
    \begin{subtable}{\linewidth}
        \centering
    \caption{Errores para $\beta = 0.9$ y $\gamma = 0.3$}
    \begin{tabular}{|c|c|c|c|c|}
    \hline
    \textbf{Estado} & \textbf{EKF} & \textbf{UKF} & \textbf{PF} & \textbf{KKF} \\ \hline
    S & 0.1486 & 0.1533 & 0.2195 & 0.1672 \\ \hline
    I & 0.2808 & 0.2590 & 0.2798 & 0.1628 \\ \hline
    R & 0.4866 & 0.5246 & 0.4907 & 0.1300 \\ \hline
    \end{tabular}
    \label{tab:errores_beta_gamma_09}
    \end{subtable}
    \begin{subtable}{\linewidth}
        \centering
    \caption{Errores para $\beta = 1.5$ y $\gamma = 0.3$}
    \begin{tabular}{|c|c|c|c|c|}
    \hline
    \textbf{Estado} & \textbf{EKF} & \textbf{UKF} & \textbf{PF} & \textbf{KKF} \\ \hline
    S & 0.0885 & 0.2296 & 0.2292 & 0.1906 \\ \hline
    I & 0.2834 & 0.2788 & 0.2816 & 0.1934 \\ \hline
    R & 0.4671 & 0.5295 & 0.4042 & 0.1435 \\ \hline
    \end{tabular}
    \label{tab:errores_beta_gamma_15}
    \end{subtable}
\end{table}

\begin{figure}[h!]
    \centering
    \includegraphics[width=0.9\linewidth]{img/content/chapter4/nonlinear_filters_sir_error_beta.pdf}
    \caption{Error en norma de EKF (azul), UKF (naranja), PF (verde) y KKF (rojo) para distintos valores de $\beta$ entre $0.1$ y $2.5$. Eje y en escala logarítmica.}
    \label{fig:nonlinear_filters_sir_error_beta}
\end{figure}

\newpage

\section{Comparación con métodos de Markov Chain Monte Carlo}


\begin{figure}[h!]
    \centering
    \begin{subfigure}[b]{0.49\textwidth}
    \centering
         \includegraphics[width=0.8\linewidth]{img/content/chapter4/nonlinear_filters_sir_params.pdf}
    \caption{}
    \end{subfigure}
   \begin{subfigure}[b]{0.49\textwidth}
   \centering
       \includegraphics[width=0.8\linewidth]{img/content/chapter4/nonlinear_filters_sir_params_evolution.pdf}
       \caption{}
   \end{subfigure}
    \caption{Resultado para modelo SIR \eqref{eq:SIR}. \\
    (a) Resultado de KKF para estimación de parámetros, primera cadena. \\
    (b) Evolución de la estimación de los parámetros a través de las iteraciones del algoritmo de estimación, primera cadena. De color rojo, el régimen de \textit{warm up}, mientras que en naranja se encuentran que sí serán consideradas para la estimación. En azul se encuentra el valor real del parámetro.}
    \label{fig:param_estim_SIR}
\end{figure}

\begin{figure}[h!]
    \centering
    \begin{subfigure}[b]{0.8\textwidth}
        \centering\includegraphics[width=0.8\linewidth]{img/content/chapter4/nonlinear_filters_sir_params_density.pdf}
    \caption{Densidades de las 8 cadenas.}
    \label{fig:nonlinear_filters_sir_params_density}
    \end{subfigure}
        
    \begin{subfigure}[b]{0.8\textwidth}
        \centering \includegraphics[width=0.8\linewidth]{img/content/chapter4/nonlinear_filters_sir_params_density_mean.pdf}
    \caption{Densidad promedio entre las 8 cadenas.}
    \label{fig:nonlinear_filters_sir_params_density_mean}
    \end{subfigure}
    \caption{Densidades de estimación para los parámetros del modelo SIR.}
\end{figure}

\begin{figure}[h!]
    \centering
    \begin{subfigure}[b]{0.49\textwidth}
         \centering
         \includegraphics[height=0.9\linewidth]{img/content/chapter4/nonlinear_filters_sirs_params.pdf}
         \caption{}
         \label{fig:nonlinear_filters_sirs_params}
    \end{subfigure}
    \begin{subfigure}[b]{0.49\textwidth}
         \centering \includegraphics[height=0.9\linewidth]{img/content/chapter4/nonlinear_filters_sirs_params_evolution.pdf}
         \caption{}
         \label{fig:nonlinear_filters_sirs_params_evolution}
    \end{subfigure}
    \caption{Resultado para modelo SIR con pérdida de inmunidad \eqref{eq:SIRS}. \\
    (a) Resultado de KKF para estimación de parámetros, primera cadena. \\
    (b) Evolución de la estimación de los parámetros a través de las iteraciones del algoritmo de estimación, primera cadena. De color rojo, el régimen de \textit{warm up}, mientras que en naranja se encuentran que sí serán consideradas para la estimación. En azul se encuentra el valor real del parámetro.}
    \label{fig:SIR_inmun}
\end{figure}

\begin{figure}[h]
    \centering
    \begin{subfigure}[b]{0.8\textwidth}
        \centering \includegraphics[width=0.8\linewidth]{img/content/chapter4/nonlinear_filters_sirs_params_density.pdf}
    \caption{Densidades de las 8 cadenas.}
    \label{fig:nonlinear_filters_sirs_params_density}
    \end{subfigure}
    \begin{subfigure}[b]{0.8\textwidth}
        \centering \includegraphics[width=0.8\linewidth]{img/content/chapter4/nonlinear_filters_sirs_params_density_mean.pdf}
    \caption{Densidad promedio entre las 8 cadenas.}
    \label{fig:nonlinear_filters_sirs_params_density_mean}
    \end{subfigure}
    \caption{Densidades de estimación para los parámetros del modelo SIR con pérdida de inmunidad.}
\end{figure}

\begin{figure}[h]
    \centering
    \begin{subfigure}[b]{0.49\textwidth}
         \includegraphics[height=\linewidth]{img/content/chapter4/nonlinear_filters_seir_params.pdf}
         \caption{}
         \label{fig:nonlinear_filters_seir_params}
    \end{subfigure}
    \begin{subfigure}[b]{0.49\textwidth}
         \includegraphics[height=\linewidth]{img/content/chapter4/nonlinear_filters_seir_params_evolution.pdf}
         \caption{}
         \label{fig:nonlinear_filters_seir_params_evolution}
    \end{subfigure}
    \caption{Resultado para modelo SEIR \eqref{eq:SEIR}. \\
    (a) Resultado de KKF para estimación de parámetros, primera cadena. \\
    (b) Evolución de la estimación de los parámetros a través de las iteraciones del algoritmo de estimación, primera cadena. De color rojo, el régimen de \textit{warm up}, mientras que en naranja se encuentran que sí serán consideradas para la estimación. En azul se encuentra el valor real del parámetro.}
    \label{fig:SEIR}
\end{figure}

\begin{figure}[h]
    \centering
    \begin{subfigure}[b]{0.8\textwidth}
        \centering \includegraphics[width=0.9\linewidth]{img/content/chapter4/nonlinear_filters_seir_params_density.pdf}
    \caption{Densidades de las 8 cadenas.}
    \label{fig:nonlinear_filters_seir_params_density}
    \end{subfigure}
    \begin{subfigure}[b]{0.8\textwidth}
        \centering \includegraphics[width=0.9\linewidth]{img/content/chapter4/nonlinear_filters_seir_params_density_mean.pdf}
    \caption{Densidad promedio entre las 8 cadenas.}
    \label{fig:nonlinear_filters_seir_params_density_mean}
    \end{subfigure}
    \caption{Densidades de estimación para los parámetros del modelo SEIR.}
\end{figure}

\begin{figure}[h]
    \centering
    \begin{subfigure}[b]{0.49\linewidth}
        \centering \includegraphics[height=0.65\linewidth]{img/content/chapter4/DEMetropolis_sir_params_trace.pdf}
        \caption{DEMetropolisZ, 20000 iteraciones de estimación y 20000 de \textit{warm up}.}
        \label{fig:DEMetropolis_sirs_params_trace}
    \end{subfigure}
    \begin{subfigure}[b]{0.49\linewidth}
        \centering \includegraphics[height=0.65\linewidth]{img/content/chapter4/NUTS_sir_params_trace.pdf}
        \caption{NUTS, 150 iteraciones de estimación y 150 de \textit{warm up}.}
        \label{fig:NUTS_sir_params_trace}
    \end{subfigure}
    \caption{Evolución de los parámetros de SIR para una cadena de MCMC. Se muestran las iteraciones posteriores a las de \textit{warm up}, en naranjo el valor estimado del parámetro y en línea punteada el valor real.}
    \label{fig:MCMC_sir_params_trace}
\end{figure}

\begin{figure}[h]
    \centering
    \begin{subfigure}[b]{\linewidth}
        \centering
        \includegraphics[width=0.7\linewidth]{img/content/chapter4/DEMetropolis_sir_params_density.pdf}
        \caption{Densidad de las 8 cadenas.}
    \end{subfigure}
     \begin{subfigure}[b]{\linewidth}
        \centering
        \includegraphics[width=0.7\linewidth]{img/content/chapter4/DEMetropolis_sir_params_density_mean.pdf}
        \caption{Densidad de las 8 cadenas.}
    \end{subfigure}
    \caption{Densidad de los parámetros del modelo SIR estimados con MCMC con \textit{sampler} DEMEtropolisZ.}
\end{figure}

\begin{figure}[h]
    \centering
    \begin{subfigure}[b]{\linewidth}
        \centering
        \includegraphics[width=0.55\linewidth]{img/content/chapter4/NUTS_sir_params_density.pdf}
        \caption{Densidad de las 8 cadenas.}
    \end{subfigure}
     \begin{subfigure}[b]{\linewidth}
        \centering
        \includegraphics[width=0.55\linewidth]{img/content/chapter4/NUTS_sir_params_density_mean.pdf}
        \caption{Densidad de las 8 cadenas.}
    \end{subfigure}
    \caption{Densidad de los parámetros del modelo SIR estimados con MCMC con \textit{sampler} NUTS.}
\end{figure}

\begin{figure}[h]
    \centering
    \begin{subfigure}[b]{0.49\linewidth}
        \centering
        \includegraphics[width=\linewidth]{img/content/chapter4/DEMetropolis_sirs_params_trace.pdf}
        \caption{DEMetropolisZ, 20000 iteraciones de estimación y 20000 de \textit{warm up}.}
        \label{fig:DEMetropolisZ_sirs_params_trace}
    \end{subfigure}
    \begin{subfigure}[b]{0.49\linewidth}
        \centering
        \includegraphics[width=\linewidth]{img/content/chapter4/NUTS_sirs_params_trace.pdf}
        \caption{NUTS, 150 iteraciones de estimación y 150 de \textit{warm up}.}
        \label{fig:NUTS_sirs_params_trace}
    \end{subfigure}
    \caption{Evolución de los parámetros de SIR con pérdida de inmunidad para una cadena de MCMC. Se muestran las iteraciones posteriores a las de \textit{warm up}, en naranjo el valor estimado del parámetro y en línea punteada el valor real.}
    \label{fig:MCMC_sirs_params_trace}
\end{figure}

\begin{figure}[h]
    \centering
    \begin{subfigure}[b]{\linewidth}
        \centering
        \includegraphics[width=0.55\linewidth]{img/content/chapter4/DEMetropolis_sirs_params_density.pdf}
        \caption{Densidad de las 8 cadenas.}
    \end{subfigure}
     \begin{subfigure}[b]{\linewidth}
        \centering
        \includegraphics[width=0.55\linewidth]{img/content/chapter4/DEMetropolis_sirs_params_density_mean.pdf}
        \caption{Densidad promedio entre las 8 cadenas.}
    \end{subfigure}
    \caption{Densidad de los parámetros del modelo SIR con pérdidad de inmunidad estimados con MCMC con \textit{sampler} DEMetropolisZ.}
\end{figure}

\begin{figure}[h]
    \centering
    \begin{subfigure}[b]{\linewidth}
        \centering
        \includegraphics[width=0.55\linewidth]{img/content/chapter4/NUTS_sirs_params_density.pdf}
        \caption{Densidad de las 8 cadenas.}
    \end{subfigure}
     \begin{subfigure}[b]{\linewidth}
        \centering
        \includegraphics[width=0.55\linewidth]{img/content/chapter4/NUTS_sirs_params_density_mean.pdf}
        \caption{Densidad promedio entre las 8 cadenas.}
    \end{subfigure}
    \caption{Densidad de los parámetros del modelo SIR con pérdida de inmunidad estimados con MCMC con \textit{sampler} NUTS.}
\end{figure}

\begin{figure}[h]
    \centering
    \begin{subfigure}[b]{0.49\linewidth}
        \centering
        \includegraphics[width=\linewidth]{img/content/chapter4/DEMetropolis_seir_params_trace.pdf}
        \caption{DEMetropolisZ, 20000 iteraciones de estimación y 20000 de \textit{warm up}.}
        \label{fig:DEMetropolisZ_seir_params_trace}
    \end{subfigure}
    \begin{subfigure}[b]{0.49\linewidth}
        \centering
        \includegraphics[width=\linewidth]{img/content/chapter4/NUTS_seir_params_trace.pdf}
        \caption{NUTS, 150 iteraciones de estimación y 150 de \textit{warm up}.}
        \label{fig:NUTS_seir_params_trace}
    \end{subfigure}
    \caption{Evolución de los parámetros de SEIR para una cadena de MCMC. Se muestran las iteraciones posteriores a las de \textit{warm up}, en naranjo el valor estimado del parámetro y en línea punteada el valor real.}
    \label{fig:MCMC_seir_params_trace}
\end{figure}

\begin{figure}[h]
    \centering
    \begin{subfigure}[b]{\linewidth}
        \centering
        \includegraphics[width=0.55\linewidth]{img/content/chapter4/DEMetropolis_seir_params_density.pdf}
        \caption{Densidad de las 8 cadenas.}
    \end{subfigure}
     \begin{subfigure}[b]{\linewidth}
        \centering
        \includegraphics[width=0.55\linewidth]{img/content/chapter4/DEMetropolis_seir_params_density_mean.pdf}
        \caption{Densidad promedio entre las 8 cadenas.}
    \end{subfigure}
    \caption{Densidad de los parámetros del modelo SEIR estimados con MCMC con \textit{sampler} DEMEtropolisZ.}
\end{figure}

\begin{figure}[h]
    \centering
    \begin{subfigure}[b]{\linewidth}
        \centering
        \includegraphics[width=0.55\linewidth]{img/content/chapter4/NUTS_seir_params_density.pdf}
        \caption{Densidad de las 8 cadenas.}
    \end{subfigure}
     \begin{subfigure}[b]{\linewidth}
        \centering
        \includegraphics[width=0.55\linewidth]{img/content/chapter4/NUTS_seir_params_density_mean.pdf}
        \caption{Densidad promedio entre las 8 cadenas.}
    \end{subfigure}
    \caption{Densidad de los parámetros del modelo SEIR estimados con MCMC con \textit{sampler} NUTS.}
\end{figure}

\begin{figure}[h]
    \centering
    \includegraphics[width=\linewidth]{img/content/chapter4/SIR_traj_params.pdf}
    \caption{Trayectorias generadas desde parámetros \textit{sampleados} desde las densidades de probabilidad generada por los distintos métodos de estimación de parámetros, modelo SIR.}
    \label{fig:SIR_traj_params}
\end{figure}

\begin{figure}[h]
    \centering
    \includegraphics[width=\linewidth]{img/content/chapter4/SEIR_traj_params.pdf}
    \caption{Trayectorias generadas desde parámetros \textit{sampleados} desde las densidades de probabilidad generada por los distintos métodos de estimación de parámetros, modelo SEIR.}
    \label{fig:SEIR_traj_params}
\end{figure}

\begin{figure}[h]
    \centering
    \includegraphics[width=\linewidth]{img/content/chapter4/SIRS_traj_params.pdf}
    \caption{Trayectorias generadas desde parámetros \textit{sampleados} desde las densidades de probabilidad generada por los distintos métodos de estimación de parámetros, modelo SIR con pérdida de inmunidad.}
    \label{fig:SIRS_traj_params}
\end{figure}
\end{appendixd}


\bibliographystyle{plain}
\bibliography{library}

% FIN DEL DOCUMENTO
\end{document}